\section{Article 11 - Les élections}\label{article-11---les-uxe9lections}

\subsection{Politique générale}\label{politique-guxe9nuxe9rale}

\begin{enumerate}
 \item
  Les procédures électorales décrites dans le Manuel des Politiques de la FCÉG sont valides pour l\textquotesingle élection en Assemblée générale ou par le Conseil d\textquotesingle administration pour les conseillers, les membres de l\textquotesingle exécutif national et les ambassadeur.drice.s.
 \item
  Les procédures électorales doivent inclure une stipulation selon laquelle, pour être élu, un candidat doit obtenir la majorité absolue des votes, les abstentions étant exclues.
\end{enumerate}

\subsection{Éligibilité}\label{uxe9ligibilituxe9}

\begin{enumerate}
 \item
  Pour être éligible, le candidat ou la candidate à un poste de l\textquotesingle exécutif national doit être membre d\textquotesingle un membre actif au moment de sa candidature.
 \item
  Il n\textquotesingle y a pas de condition d\textquotesingle éligibilité spécifique pour le candidat ou la candidate à un poste de conseiller ou conseillère.
 \item
  Aucun(e) étudiant(e) ne peut occuper plus d\textquotesingle un poste au sein de l\textquotesingle équipe des responsables de la FCÉG à un moment donné.
\end{enumerate}
