\section{Article 13 - Les dispositions financières et légales}\label{article-13---les-dispositions-financiuxe8res-et-luxe9gales}

\subsection{L'audit externe}\label{laudit-externe}

\subsubsection{\texorpdfstring{La Fédération }{La Fédération }}\label{la-fuxe9duxe9ration}

\begin{enumerate}
 \item
  La Fédération doit effectuer une vérification externe annuelle.
 \item
  L\textquotesingle entité d\textquotesingle audit doit être approuvée chaque année lors de l\textquotesingle assemblée générale annuelle.
 \item
  Le coût de cette révision financière doit être estimé au moment de leur approbation, le coût total engagé doit être rapporté à chaque réunion des membres.
\end{enumerate}

\subsubsection{Les activités}\label{les-activituxe9s}

\begin{enumerate}
 \item
  Toutes les activités de la FCÉG doivent faire l\textquotesingle objet d\textquotesingle un audit externe, réalisé par l\textquotesingle organisation qui a géré les comptes financiers de l\textquotesingle activité.
 \item
  Il s\textquotesingle agira soit de la FCÉG, soit d\textquotesingle un organisme incorporé approuvé par le Conseil d\textquotesingle administration.
 \item
  Les frais d\textquotesingle audit de la FCÉG, décrits au point Article 13.1.3, ne s\textquotesingle appliquent qu\textquotesingle aux écoles d\textquotesingle accueil qui utilisent des comptes bancaires de la FCÉG pour leur activité.
\end{enumerate}

\subsection{L'audit par les membres}\label{laudit-par-les-membres}

\begin{enumerate}
 \item
  Lors du Sommet du développement des associations en ingénierie, un membre actif peut demander à auditer les livres de la Fédération pour l\textquotesingle année en cours.
 \item
  Le membre doit rendre compte à la prochaine Assemblée générale annuelle.
 \item
  Si plus d\textquotesingle un membre fait une demande, l\textquotesingle Assemblée générale ou le Conseil d\textquotesingle administration nomme le membre responsable de l\textquotesingle audit.
\end{enumerate}

\subsection{Les remboursements}\label{les-remboursements}

\subsubsection{Ordinaires}\label{ordinaires}

\begin{enumerate}
 \item
  Les directeur.trice.s et les commissaires seront remboursés de tous les coûts, frais et dépenses qu\textquotesingle ils supportent ou engagent dans le cadre de toute action, poursuite ou procédure intentée contre eux, ou en ce qui concerne tout acte, fait, matière ou chose quelconque, fait, fait ou permis par eux, dans l\textquotesingle exécution des fonctions de leur poste ou en ce qui concerne toute responsabilité de ce genre.
\end{enumerate}

\subsubsection{Extraordinaires}\label{extraordinaires}

\begin{enumerate}
 \item
  Le Conseil d\textquotesingle administration peut autoriser le remboursement des directeur.trice.s, des membres de l\textquotesingle exécutif national et des commissaires de tous les autres coûts, charges et dépenses qu\textquotesingle ils supportent ou engagent dans ou en relation avec les affaires de celui-ci, sauf les coûts, charges ou dépenses occasionnés par leur propre négligence ou défaut délibéré.
\end{enumerate}

\subsection{Société de gestion}\label{sociuxe9tuxe9-de-gestion}

\begin{enumerate}
 \item
  Les directeur.trice.s ont le pouvoir d\textquotesingle autoriser des dépenses au nom de la Fédération de temps à autre et peuvent déléguer par résolution à un ou des officiers de la Fédération le droit d\textquotesingle employer et de payer les salaires des employés.
 \item
  Les directeur.trice.s ont le pouvoir de conclure une convention de fiducie avec une société de fiducie dans le but de créer un fonds de fiducie dans lequel le capital et les intérêts peuvent être mis à la disposition pour promouvoir l\textquotesingle intérêt de la Fédération conformément aux termes que le Conseil d\textquotesingle administration ou les membres peuvent prescrire.
\end{enumerate}

\subsection{Dons}\label{dons}

\begin{enumerate}
 \item
  Le Conseil d\textquotesingle administration prendra les mesures qu\textquotesingle il jugera nécessaires pour permettre à la Fédération d\textquotesingle acquérir, d\textquotesingle accepter, de solliciter ou de recevoir des legs, des dons, des subventions, des règlements, des legs, des dotations et des dons de toute nature que ce soit dans le but de promouvoir les objectifs de la Fédération.
\end{enumerate}

\subsection{Procuration}\label{procuration}

\begin{enumerate}
 \item
  Le Conseil d\textquotesingle administration peut donner à la Fédération le pouvoir de représentation à tout courtier en valeurs mobilières enregistré aux fins du transfert et de la négociation de tout stock, obligation et autre valeur mobilière de la Fédération.
\end{enumerate}

\subsection{Les contrats}\label{les-contrats}

\subsubsection{Les exigences de signature}\label{les-exigences-de-signature}

\begin{enumerate}
 \item
  Tout contrat, document ou instrument écrit doit être signé par le/la vice-président.e de finance et de l\textquotesingle administration.
 \item
  Tous les contrats ainsi signés lieront la Fédération sans aucune autorisation ou formalité supplémentaire si la valeur n\textquotesingle excède pas 1000 \$.
 \item
  Les contrats, documents ou instruments écrits dépassant 1000 \$ et nécessitant la signature de la Fédération doivent être signés par deux membres de l\textquotesingle exécutif national.
 \item
  Tous les contrats, documents et instruments écrits ainsi signés lieront la Fédération sans aucune autorisation ou formalité supplémentaire si la valeur n\textquotesingle excède pas 10 000 \$.
 \item
  Pour un contrat d\textquotesingle une valeur supérieure à 10 000 \$, une résolution autorisant sa signature doit être adoptée par le Conseil d\textquotesingle administration.
 \item
  Tout contrat, document ou instrument écrit nécessitant la signature de la Fédération pour une activité ou un service doit être signé par son responsable d'activité, ou son délégué, nécessitant également l\textquotesingle approbation du Conseil d\textquotesingle administration en cas de dépassement de 10 000 \$.
\end{enumerate}

\subsubsection{L'agent de signature spécial}\label{lagent-de-signature-spuxe9cial}

\begin{enumerate}
 \item
  Le Conseil d\textquotesingle administration aura le pouvoir, de temps à autre par résolution, de nommer un commissaire de la Fédération pour signer des contrats, documents et instruments écrits spécifiques.
\end{enumerate}
