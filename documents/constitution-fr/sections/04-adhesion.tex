\section{Article 4 - L\textquotesingle adhésion}\label{article-4---ladhuxe9sion}

\subsection{Membre régulier (société d'ingénierie)}\label{membre-ruxe9gulier-sociuxe9tuxe9-dinguxe9nierie}

\begin{enumerate}
 \item
  Un membre régulier de la Fédération est une société représentant les étudiants de premier cycle en génie dans un établissement d\textquotesingle enseignement supérieur canadien offrant un programme de génie accrédité, ou en cours d\textquotesingle accréditation, par le Bureau canadien d\textquotesingle accréditation en génie, qui représente les préoccupations des étudiants auprès de leur faculté et de l\textquotesingle administration de leur établissement d\textquotesingle enseignement supérieur et qui a reçu l\textquotesingle appui de l\textquotesingle Assemblée générale tel que spécifié à l\textquotesingle article 4.3.2.
\end{enumerate}

\subsection{Membre régulier coopératif}\label{membre-ruxe9gulier-coopuxe9ratif}

\begin{enumerate}
 \item
  Plusieurs sociétés peuvent postuler pour co-représenter les étudiants de premier cycle en génie de leur établissement, comme si elles étaient éligibles à être un membre régulier ordinaire tel que défini à la Article 4.1, lorsqu\textquotesingle il n\textquotesingle y a aucune société unique répondant à la description.
 \item
  Un membre régulier coopératif est identique à un membre régulier ordinaire à tous égards, sauf que le représentant votant doit être confirmé par le/la président.e de chaque société coopérative.
\end{enumerate}

\subsection{Demande d'adhésion}\label{demande-dadhuxe9sion}

\begin{enumerate}
 \item
  Pour devenir membre régulier, une organisation doit soumettre une demande écrite au/à la vice-président.e de finance et de l\textquotesingle administration, qui doit immédiatement informer les membres d\textquotesingle une telle demande.
 \item
  L\textquotesingle Assemblée des membres doit alors prendre une décision concernant l\textquotesingle admission de l\textquotesingle organisation, et ce, au plus tard lors de la réunion annuelle suivant la soumission de la demande par l\textquotesingle organisation.
\end{enumerate}

\subsection{\texorpdfstring{Statut d\textquotesingle observateur ou d'observatrice (société d'ingénierie) }{Statut d\textquotesingle observateur ou d'observatrice (société d'ingénierie) }}\label{statut-dobservateur-ou-dobservatrice-sociuxe9tuxe9-dinguxe9nierie}

\begin{enumerate}
 \item
  Un observateur de la Fédération est une société représentant les étudiants de premier cycle en génie dans un établissement d\textquotesingle enseignement supérieur canadien offrant un programme de génie accrédité, ou en cours d\textquotesingle accréditation, par le Bureau canadien d\textquotesingle accréditation en génie, qui représente les préoccupations des étudiants auprès de leur faculté et de l\textquotesingle administration de leur établissement d\textquotesingle enseignement supérieur, et qui a reçu le soutien du Conseil d\textquotesingle administration tel que spécifié dans les statuts.
 \item
  Le statut d\textquotesingle observateur est valide pendant un an après l\textquotesingle approbation d\textquotesingle une demande, après quoi la société en question doit choisir de présenter une demande pour devenir membre régulier conformément à l\textquotesingle article 4.3, présenter une demande de prolongation unique ou se retirer de l\textquotesingle adhésion, conformément aux statuts.
\end{enumerate}
