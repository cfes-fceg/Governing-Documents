\section{Chapter 3 - Les dispositions}\label{chapter-3---les-dispositions}

\subsection{Les directives principes}\label{les-directives-principes}

\begin{enumerate}
 \item
  La FCÉG soutient la croissance et la communication au sein de ses membres afin de garantir leur bien-être moral, intellectuel, culturel, académique, social et économique.
 \item
  La Fédération représente et promeut les étudiants en génie du premier cycle au Canada sur le plan national et international.
 \item
  La FCÉG ne s\textquotesingle alignera sur aucun parti politique.
\end{enumerate}

\subsection{La mission}\label{la-mission}

\begin{enumerate}
 \item
  Le but de la FCÉG est de solliciter, représenter, organiser et échanger des points de vue, des informations et des activités pertinentes pour les objectifs des membres au niveau national afin de garantir leur bien-être moral, intellectuel, culturel, académique, social et économique, et d\textquotesingle améliorer la qualité et la portée de l\textquotesingle éducation en génie au Canada.
 \item
  Les moyens par lesquels cela sera accompli sont les suivants:

  \begin{enumerate}
   \item
    Organiser et accueillir des activités, réunions, assemblées, expositions et compétitions liées à cet objectif,
   \item
    S\textquotesingle affilier à toute organisation poursuivant les mêmes objectifs que ceux de ladite corporation,
   \item
    Acquérir par achat, location ou autrement, posséder et exploiter les actifs, biens mobiliers et immobiliers nécessaires à la réalisation des objectifs susmentionnés,
   \item
    Fournir à ses membres des services de toutes sortes liés aux objectifs susmentionnés,
   \item
    Créer, soutenir, participer à des mouvements, campagnes d\textquotesingle opinion ou autres organisations liées à ces objectifs chaque fois que possible,
   \item
    Encourager les sociétés membres et associées à promouvoir la création et l\textquotesingle échange d\textquotesingle informations pertinentes pour tous les membres et autres parties intéressées,
   \item
    Promouvoir l\textquotesingle interaction des étudiants en génie canadiens avec des groupes d\textquotesingle intérêt spécifiques (sociétés, associations, gouvernements et autres organismes) sur des questions nationales et internationales de préoccupation sociale, économique, politique, juridique et humaine pertinentes pour le génie,
   \item
    Présenter des informations pertinentes aux gouvernements et autres organismes compétents,
   \item
    Coopérer avec toutes les associations de génie reconnues par la corporation dans les domaines d\textquotesingle intérêt commun,
   \item
    Promouvoir l\textquotesingle image publique des étudiants en génie au Canada,
   \item
    Faciliter et développer des activités et des services, y compris mais sans s\textquotesingle y limiter, le Congrès canadien sur le leadership en ingénierie, la Compétition canadienne d\textquotesingle ingénierie, le Congrès sur la diversité en ingénierie et le Sommet du développement des associations en ingénierie de la FCÉG.
  \end{enumerate}
\end{enumerate}

\subsection{Les régions}\label{les-ruxe9gions}

\begin{enumerate}
 \item
  La FCÉG est composée de 4 régions. Elles sont les suivantes :

  \begin{enumerate}
   \item
    L'Atlantique : Les provinces de l\textquotesingle Île-du-Prince-Édouard, du Nouveau-Brunswick, de la Nouvelle-Écosse et de Terre-Neuve-et-Labrador ;
   \item
    Le Québec : La province de Québec ;
   \item
    L'Ontario : La province de l\textquotesingle Ontario ;
   \item
    L'Ouest : Les provinces de la Colombie-Britannique, de l\textquotesingle Alberta, de la Saskatchewan, du Manitoba, ainsi que les Territoires du Yukon, du Nunavut et des Territoires du Nord-Ouest.
  \end{enumerate}
\end{enumerate}

\subsection{Les langues officielles}\label{les-langues-officielles}

\begin{enumerate}
 \item
  Toutes les communications adressées aux membres seront effectuées dans l\textquotesingle une ou l\textquotesingle autre des langues officielles, ou dans les deux, à la discrétion du membre.
 \item
  Les demandes de traduction d\textquotesingle une langue officielle à l\textquotesingle autre peuvent être faites par n\textquotesingle importe quelle organisation membre.
 \item
  De plus, toutes les publications de la Fédération seront disponibles simultanément dans les deux langues officielles du Canada.
 \item
  En cas de divergence entre les versions anglaise et française de cette constitution, la version anglaise prévaut.
 \item
  En cas de divergence entre les versions anglaise et française des règles et règlements ou des résolutions, le document original prévaut sur la traduction en tant que document officiel de la Fédération.
\end{enumerate}

\subsection{Invalidité}\label{invalidituxe9}

\begin{enumerate}
 \item
  Les dispositions invalides ou illégales de cette constitution ou de tout autre règlement que la Fédération adoptera par la suite n\textquotesingle invalideront pas l\textquotesingle intégralité de cette constitution, de ces règlements ou de ces règlements, et ils resteront valides comme si ces dispositions invalides ou illégales n\textquotesingle avaient jamais été incluses.
\end{enumerate}

\subsection{Les titres}\label{les-titres}

\begin{enumerate}
 \item
  Les titres insérés dans cette constitution et dans tous les autres documents de la Fédération adoptés par la suite sont inclus uniquement pour faciliter la lecture et ne devraient pas affecter l\textquotesingle interprétation de cette constitution et des documents.
\end{enumerate}

\subsection{L'interprétation}\label{linterpruxe9tation}

\begin{enumerate}
 \item
  Dans cette constitution et dans tous les autres documents de la Fédération adoptés par la suite, sauf indication contraire du contexte, les mots au singulier incluront le pluriel, et vice versa, et les références aux personnes incluront les entreprises et les sociétés.
\end{enumerate}

\subsection{Le Manuel des politiques}\label{le-manuel-des-politiques}

\begin{enumerate}
 \item
  En complément à cette constitution et aux règlements qui y sont contenus, des politiques additionnelles et des meilleures pratiques peuvent être consultées dans le Manuel des politiques de la FCÉG.
\end{enumerate}

\subsection{Le Cahier des positions}\label{le-cahier-des-positions}

\begin{enumerate}
 \item
  Les positions officielles prises par la FCÉG peuvent être consultées dans le Cahier des positions tel qu\textquotesingle indiqué dans le Manuel des politiques.
\end{enumerate}
