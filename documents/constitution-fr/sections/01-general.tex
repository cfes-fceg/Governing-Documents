\section{Article 1 - Général}\label{article-1---guxe9nuxe9ral}

\subsection{Le nom de la corporation}\label{le-nom-de-la-corporation}

\begin{enumerate}
 \item
  Le nom de la corporation sera "la Fédération canadienne des étudiant.e.s en génie", désignée ci-après par "la FCÉG".
\end{enumerate}

\subsection{Le siège social}\label{le-siuxe8ge-social}

\begin{enumerate}
 \item
  Le siège social de la FCÉG sera situé au bureau d\textquotesingle Ingénieurs Canada, 55 rue Metcalfe - Bureau 300, Ottawa, Ontario, K1P 6L9.
\end{enumerate}

\subsection{Le sceau}\label{le-sceau}

\begin{enumerate}
 \item
  Le Sceau, dont une impression est apposée ci-dessous, sera le Sceau de la Fédération.
\end{enumerate}

\subsection{L'année fiscale}\label{lannuxe9e-fiscale}

\begin{enumerate}
 \item
  L'année fiscale de la FCÉG se terminera le 30 avril de chaque année.
\end{enumerate}

\subsection{Les abréviations et logo}\label{les-abruxe9viations-et-logo}

\subsubsection{La Fédération canadienne des étudiant.e.s en génie}\label{la-fuxe9duxe9ration-canadienne-des-uxe9tudiant.e.s-en-guxe9nie-1}

\begin{enumerate}
 \item
  Les abréviations FCÉG et CFES ainsi que le logo ci-dessous sont utilisés pour désigner la Fédération canadienne des étudiant.e.s en génie.
 \item
  Dans cette constitution, le terme "fédération" désigne la Fédération canadienne des étudiant.e.s en génie
\end{enumerate}

\subsubsection{La Compétition canadienne en ingénierie}\label{la-compuxe9tition-canadienne-en-inguxe9nierie}

\begin{enumerate}
 \item
  Les abréviations CCI et CEC sont utilisées pour désigner la Compétition canadienne en ingénierie
 \item
  Le logo de la Compétition canadienne en ingénierie doit incorporer, en partie ou en totalité, le logo de la FCÉG tel qu\textquotesingle établi à l\textquotesingle article 1.5.1.
\end{enumerate}

\subsubsection{Le Congrès canadien sur le leadership en ingénierie}\label{le-congruxe8s-canadien-sur-le-leadership-en-inguxe9nierie}

\begin{enumerate}
 \item
  L\textquotesingle abréviation CCLI est utilisée pour désigner le Congrès canadien sur le leadership en ingénierie de la Fédération canadienne des étudiant.e.s en génie.
 \item
  Le logo du Congrès canadien sur le leadership en ingénierie de la Fédération canadienne des étudiant.e.s en génie doit incorporer, en partie ou en totalité, le logo de la FCÉG tel qu\textquotesingle établi à l\textquotesingle article 1.5.1.
\end{enumerate}

\subsubsection{Le Congrès sur la diversité en ingénierie}\label{le-congruxe8s-sur-la-diversituxe9-en-inguxe9nierie}

\begin{enumerate}
 \item
  L\textquotesingle abréviation CDI est utilisée pour désigner le Congrès sur la diversité en ingénierie de la Fédération canadienne des étudiant.e.s en génie.
 \item
  Le logo du Congrès sur la diversité en ingénierie de la Fédération canadienne des étudiant.e.s en génie doit incorporer, en partie ou en totalité, le logo de la FCÉG tel qu\textquotesingle établi à l\textquotesingle article 1.5.1.
\end{enumerate}

\subsubsection{Le Sommet du développement des associations en ingénierie}\label{le-sommet-du-duxe9veloppement-des-associations-en-inguxe9nierie}

\begin{enumerate}
 \item
  L\textquotesingle abréviation FCÉG SDAI est utilisée pour désigner le Sommet du développement des associations en ingénierie de la Fédération canadienne des étudiant.e.s en génie.
 \item
  Le logo du Sommet du développement des associations en ingénierie de la Fédération canadienne des étudiant.e.s en génie doit incorporer, en partie ou en totalité, le logo de la FCÉG tel qu\textquotesingle établi à l\textquotesingle article 1.5.1.
\end{enumerate}

\subsubsection{Le Congrès sur le développement durable en ingénierie}\label{le-congruxe8s-sur-le-duxe9veloppement-durable-en-inguxe9nierie}

\begin{enumerate}
 \item
  L\textquotesingle abréviation CDDI est utilisée pour désigner le Congrès sur le développement durable en ingénierie de la Fédération canadienne des étudiant.e.s en génie.
 \item
  Le logo du Congrès sur le développement durable en ingénierie de la Fédération canadienne des étudiant.e.s en génie doit incorporer, en partie ou en totalité, le logo de la FCÉG tel qu\textquotesingle établi à l\textquotesingle article 1.5.1.
\end{enumerate}
