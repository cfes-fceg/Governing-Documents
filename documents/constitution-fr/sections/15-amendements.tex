\section{Article 15 - Les amendements}\label{article-15---les-amendements}

\subsection{Les amendements substantiels}\label{les-amendements-substantiels}

\begin{enumerate}
 \item
  La constitution de la Fédération peut être abrogée ou modifiée par résolution à une assemblée générale où pas moins des deux tiers (2/3) des membres actifs présents doivent être d\textquotesingle accord.
\end{enumerate}

\subsection{Les amendements grammaticaux}\label{les-amendements-grammaticaux}

\begin{enumerate}
 \item
  Les amendements à cette constitution qui ne sont que des mesures correctives visant à garantir une traduction précise et une structure grammaticale peuvent être adoptés par le Conseil d\textquotesingle administration à la majorité des deux tiers (2/3), à condition que ces amendements ne modifient pas le concept ou l\textquotesingle esprit du point en question.
\end{enumerate}

\subsection{Les amendements temporaires}\label{les-amendements-temporaires}

\begin{enumerate}
 \item
  Le Conseil d\textquotesingle administration peut prescrire des règles et règlements conformes à cette constitution concernant la gestion et le fonctionnement de la Fédération comme ils le jugent opportun, à condition que ces règles et règlements n\textquotesingle aient force et effet que jusqu\textquotesingle à la prochaine assemblée annuelle des membres de la Fédération, où ils seront confirmés, et en l\textquotesingle absence de confirmation lors de cette assemblée annuelle des membres, ils cesseront d\textquotesingle avoir force et effet à partir de ce moment.
 \item
  Toute modification adoptée par le Conseil d\textquotesingle administration de cette manière doit être communiquée à tous les membres dans les 2 jours ouvrables.
\end{enumerate}
