\section{Les modifications}\label{les-modifications}

\subsection{Les modifications majeures}\label{les-modifications-majeures}

\begin{enumerate}
 \item
  Les statuts de la Fédération peuvent être abrogés ou modifiés par résolution de l\textquotesingle assemblée générale annuelle où pas moins des deux tiers (2/3) des membres actifs présents doivent être d\textquotesingle accord.
\end{enumerate}

\subsection{Les modifications grammaticales}\label{les-modifications-grammaticales}

\begin{enumerate}
 \item
  Les amendements à ces statuts qui ne sont que des mesures correctives visant à garantir une traduction précise et une structure grammaticale peuvent être adoptés par le Conseil d\textquotesingle administration à la majorité des deux tiers (2/3), à condition que ces amendements ne modifient pas le concept ou l\textquotesingle esprit du point en question.
\end{enumerate}

\subsection{Les modifications temporaires}\label{les-modifications-temporaires}

\begin{enumerate}
 \item
  Le Conseil d\textquotesingle administration peut prescrire des règles et des règlements conformes à ces statuts relatifs à la gestion et au fonctionnement de la Fédération comme il le juge opportun, à condition que ces règles et règlements n\textquotesingle aient force et effet que jusqu\textquotesingle à la prochaine assemblée annuelle des membres de la Fédération, où ils seront confirmés, et en l\textquotesingle absence de confirmation lors de cette assemblée annuelle des membres, ils cesseront d\textquotesingle avoir force et effet à partir de ce moment.
 \item
  Toutes les modifications adoptées par le Conseil d\textquotesingle administration de cette manière doivent être communiquées à tous les membres dans un délai de 10 jours ouvrables.
\end{enumerate}

