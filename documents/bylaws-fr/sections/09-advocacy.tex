\section{Le plaidoyer}\label{le-plaidoyer}

\subsection{Le Cahier des positions}\label{le-cahier-des-positions}

\begin{enumerate}
 \item
  Le Cahier des positions de la Fédération canadienne des étudiant.e.s en génie contient les positions officielles prises par la FCÉG.
 \item
  Les positions contenues dans le document peuvent être ajoutées ou modifiées par une résolution des trois quarts (3⁄4) adoptée à cette fin.
 \item
  Les positions peuvent également être retirées par une résolution majoritaire adoptée à cette fin.
 \item
  Les modifications au document nécessitent deux lectures, et des amendements à la modification peuvent être présentés lors de la deuxième lecture.
 \item
  Des modifications peuvent être apportées en exception à ces exigences avec l\textquotesingle approbation de 90\% de tous les membres présents.
 \item
  L\textquotesingle exécutif national est responsable d\textquotesingle informer les membres de la procédure entourant le document des positions avant le Sommet du développement des associations en ingénierie, ainsi que de mener des recherches préliminaires pour s\textquotesingle assurer que les positions potentielles sont généralement bénéfiques pour tous les membres, et d\textquotesingle informer les membres de toute implication associée avant le vote.
\end{enumerate}

