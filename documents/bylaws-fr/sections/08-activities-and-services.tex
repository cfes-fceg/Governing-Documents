\section{Les activités et les services}\label{les-activituxe9s-et-les-services}

\subsection{Les accords d'activités}\label{les-accords-dactivituxe9s}

\begin{enumerate}
 \item
  Les accords d\textquotesingle activités sont des contrats qui spécifient la nature de la relation entre la FCÉG, le responsable d\textquotesingle activité et la société membre hôte qui a remporté l\textquotesingle enchère pour organiser une activité.
 \item
  Les accords d\textquotesingle activités sont en vigueur pour les activités suivantes de la Fédération :

  \begin{enumerate}
   \item
    Le Congrès canadien sur le leadership en ingénierie (CCLI)
   \item
    Le Sommet du développement des associations en ingénierie (SDAI)
   \item
    La Compétition canadienne d\textquotesingle ingénierie (CCI)
   \item
    Le Congrès sur la diversité en ingénierie (CDI)
   \item
    Le Congrès sur le développement durable en ingénierie (CDDI)
   \item
    Le \emph{Lean Six Sigma}
  \end{enumerate}
 \item
  Normalement, ils doivent être signés au moins un (1) an avant l\textquotesingle événement en question. Ils doivent, au minimum, inclure les dispositions suivantes :

  \begin{enumerate}
   \item
    Le nom de activité
   \item
    L\textquotesingle objectif de activité
   \item
    Les responsabilités du/de la responsable d\textquotesingle activité
   \item
    Les responsabilités de la Fédération envers le/la responsable d\textquotesingle activité
   \item
    Les responsabilités de la société membre hôte
   \item
    Le processus de modification
  \end{enumerate}
\end{enumerate}

\subsection{Les responsables d'activité}\label{les-responsables-dactivituxe9}

\begin{enumerate}
 \item
  Les responsables d'activité sont les individus ayant la responsabilité exclusive de tout ce qui est lié à l\textquotesingle activité concernée, y compris, mais sans s\textquotesingle y limiter, les finances, la signature de documents et les opérations.
 \item
  Les responsables d'activité doivent être membre de la société hôte.
\end{enumerate}

\subsection{La participation dans les activités et services}\label{la-participation-dans-les-activituxe9s-et-services}

\begin{enumerate}
 \item
  Généralement, seuls les membres réguliers en règle et les membres observateurs peuvent envoyer des étudiant.e.s de leur établissement qui sont soit inscrits dans un programme d\textquotesingle ingénierie accrédité, soit inscrits dans un programme en cours d\textquotesingle accréditation, aux activités organisées par la Fédération et bénéficier des services qu\textquotesingle elle offre, les membres observateurs pouvant envoyer toute équipe admissible à la CCI, et au maximum deux délégués à toutes les autres activités.
 \item
  D\textquotesingle autres étudiant.e.s représenté.e.s par une société membre peuvent participer aux activités organisées par la Fédération s\textquotesingle ils ont été choisis selon la section 6.4.
 \item
  Le Conseil d\textquotesingle administration a le droit d\textquotesingle inviter toute autre personne à toute activité ou à bénéficier de tout service offert par la Fédération et uniquement sur cette invitation toute autre personne peut assister aux activités ou bénéficier des services.
\end{enumerate}

\subsection{Les responsabilités des sociétés membres comme services}\label{les-responsabilituxe9s-des-sociuxe9tuxe9s-membres-comme-services}

\subsubsection{Les responsabilités des sociétés participantes}\label{les-responsabilituxe9s-des-sociuxe9tuxe9s-participantes}

\begin{enumerate}
 \item
  Désigner un/une chef.fe de délégation
 \item
  Informer l'activité de tout changement concernant le/la chef.fe de délégation
 \item
  S\textquotesingle assurer que l\textquotesingle inscription préalable (si applicable) est remplie en temps voulu.
\end{enumerate}

\subsubsection{Les responsabilités des délégué.e.s principal.aux.alesprincipales}\label{les-responsabilituxe9s-des-duxe9luxe9guuxe9.e.s-principal.aux.alesprincipales}

\begin{enumerate}
 \item
  S'assurer que tous les délégué.e.s s\textquotesingle inscrivent à l'activité dans les délais impartis
 \item
  Cosigner le Code de conduite pour tous les délégué.e.s de leur société
 \item
  Gérer la délégation de leur société membre lors de l'activité
 \item
  Soumettre un plan de sécurité émotionnelle de la délégation à leur ambassadeur.drice (la manière dont ce plan est soumis sera déterminée par les ambassadeur.drices) avant l'activité, détaillant les soutiens en santé mentale disponibles pour leur délégation et les méthodes qui seront mises en place pour soutenir leurs délégué.e.s
\end{enumerate}

