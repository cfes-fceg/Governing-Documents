\section{Le Comité d'intervention des incidents}\label{le-comituxe9-dintervention-des-incidents}

\subsection{L'objectif}\label{lobjectif-2}

\begin{enumerate}
 \item
  Le formulaire de rapport d\textquotesingle incident sera la manière dont le Comité d\textquotesingle intervention des incidents reçoit les plaintes pour violations du Code de conduite.
 \item
  Afin de garantir que les activités de la FCÉG sont un espace sûr et accueillant pour tous, un comité d\textquotesingle intervention en cas d\textquotesingle incident sera formé pour enquêter et tenter de résoudre tout problème signalé.
 \item
  L\textquotesingle objectif du comité d\textquotesingle intervention en cas d\textquotesingle incident sera de veiller à ce qu\textquotesingle un environnement sûr et inclusif soit maintenu lors des activités de la FCÉG.
\end{enumerate}

\subsection{Le Formulaire de rapport d'incident}\label{le-formulaire-de-rapport-dincident}

\subsubsection{La structure}\label{la-structure}

\begin{enumerate}
 \item
  Le formulaire de rapport d'incident sera un formulaire créé par le Comité d\textquotesingle intervention des incidents et rendu accessible à tous les délégué.e.s lors d\textquotesingle une activité de la FCÉG pendant l\textquotesingle activité et jusqu\textquotesingle à trois (3) jours après.
 \item
  Le formulaire comprendra les champs suivants :

  \begin{enumerate}
   \item
    Le nom de l\textquotesingle individu qui signale
   \item
    L'école de l\textquotesingle individu qui signale
   \item
    Le/les nom.s de.s l\textquotesingle individu.s ou de l\textquotesingle organisation impliqué.e.s dans l\textquotesingle incident (ou descriptions)
   \item
    La date et heure de l\textquotesingle incident
   \item
    Les détails de l\textquotesingle incident
   \item
    Le/les témoin.s de l\textquotesingle incident, le cas échéant
   \item
    Les coordonnées (optionnelles)
   \item
    L'indication si le rapporteur accepte d\textquotesingle être contacté par le CII
   \item
    Une description de toute relation personnelle ou de tout conflit d\textquotesingle intérêts potentiel avec les membres du CII, en ce qui concerne à la fois l\textquotesingle individu qui signale et toute personne impliquée dans la plainte
   \item
    Une brève description de la procédure du CII, qui précise que tous les rapports soumis par le biais du formulaire de rapport d\textquotesingle incident iront au/à la président.e du CII et que si les plaignant.e.s souhaitent soumettre des plaintes en dehors du formulaire, cela doit être fait par le biais des coordonnées figurant sur ce formulaire.
  \end{enumerate}
\end{enumerate}

\subsubsection{La confidentialité des rapports}\label{la-confidentialituxe9-des-rapports}

\begin{enumerate}
 \item
  Tout contenu permettant d\textquotesingle identifier les rapports d\textquotesingle incidents, ainsi que les informations d\textquotesingle identification découvertes lors de l\textquotesingle enquête, seront gardés confidentiels par les membres du Comité d\textquotesingle intervention des incidents (CII), uniquement sous réserve de la section 10.3.6. : La publication des informations.
\end{enumerate}

\subsubsection{L'échéance de rapport d'incident}\label{luxe9chuxe9ance-de-rapport-dincident}

\begin{enumerate}
 \item
  Si un rapport est soumis entre 8h et 20h, le Comité d\textquotesingle intervention des incidents commencera à examiner le rapport dans les quatre heures.
 \item
  Si un rapport est soumis en dehors de ces horaires, mais durant l'activité, le CII commencera à examiner le rapport le lendemain midi.
 \item
  Le CII n\textquotesingle est pas autorisé à travailler pendant les repas, et donc tout rapport ouvert pendant ce temps voit son délai prolongé.
 \item
  Si un rapport est soumis moins de quatre (4) heures avant la fin de l'activité, ou dans les 72 heures après minuit du dernier jour de l'activité, le CII commencera à examiner le rapport dans les 36 heures suivantes.
\end{enumerate}

\subsection{Le Comité d'intervention des incidents}\label{le-comituxe9-dintervention-des-incidents-1}

\subsubsection{La structure}\label{la-structure-1}

\begin{enumerate}
 \item
  Un Comité d\textquotesingle intervention des incidents (CII) sera formé pour chaque activité de la FCÉG.
\end{enumerate}

\subsubsection{Les membres du comité}\label{les-membres-du-comituxe9}

\begin{enumerate}
 \item
  Un CII sera composé d\textquotesingle un maximum de six (6) membres, comprenant:

  \begin{enumerate}
   \item
    Jusqu\textquotesingle à deux (2) membres non votants du Conseil d\textquotesingle administration de la FCÉG qui agiront en tant que président.e du CII.
   \item
    Le.s président.e.s du CII proposé.e.s est/sont chargé.e.s de choisir un comité du CII et de le présenter au conseil d\textquotesingle administration pour approbation.
   \item
    Le comité peut être composé d\textquotesingle une des combinaisons qui suit:

    \begin{enumerate}
     \item
      Quatre (4) membres du comité organisateur, choisis par les responsables de l\textquotesingle activité en consultation avec le Conseil d\textquotesingle administration de la FCÉG, dont la seule tâche pendant l\textquotesingle activité sera le travail du CII.
      \begin{enumerate}
       \item
        Ces membres du CO peuvent, dans la mesure du possible, conserver leurs responsabilités habituelles.
       \item
        Une embauche séparée au CO (1-4 personnes) menée par le.s responsable.s d\textquotesingle activité peut être envisagée si le.s responsable.s d\textquotesingle activité exprime.nt des préoccupations raisonnables quant à la charge de travail.
       \item
        Les individus seront les bienvenus pour assister aux sessions si cela est applicable à l\textquotesingle activité.
      \end{enumerate}

     \item
      Quatre (4) délégué.e.s membres non votants sélectionnés par le Conseil d\textquotesingle administration de la FCÉG.
    \end{enumerate}
  \end{enumerate}

 \item
  Le CCLI peut avoir deux CII distinctes, chacune composée d\textquotesingle un.e président.e du CII et de quatre délégué.e.s sans droit de vote, pour un total de deux président.e.s du CII et de huit délégué.e.s sans droit de vote.
 \item
  Dans le cas de deux président.e.s du CII, ils/elles partageront les responsabilités tout au long de l\textquotesingle activité.
 \item
  Pour chaque cas, seul l\textquotesingle un des deux président.e.s agira en tant que président.e votant.e tandis que l\textquotesingle autre agira en tant que président.e consultant.e sans droit de vote.
 \item
  Le/la président.e consultant.e peut assister à toutes les entrevues et discussions de cas, mais n\textquotesingle a pas de vote final.
 \item
  Les président.e.s alterneront en tant que président.e votant.e entre les cas pour équilibrer la charge de travail.
 \item
  Les coprésident.e.s peuvent répartir les cas entre eux comme ils/elles le jugent approprié.
 \item
  Les membres du CII:

  \begin{enumerate}
   \item
    Participera à des présentations de la FCÉG avec les conseiller.ères nationaux.ales de la FCÉG.
   \item
    Suivra une formation en premiers soins psychologiques payée par la FCÉG.
   \item
    Ne comptera pas plus d\textquotesingle un membre du Conseil d\textquotesingle administration de la FCÉG.
   \item
    Sera ratifiée par le Conseil d\textquotesingle administration de la FCÉG au plus tard deux (2) semaines avant le début de l\textquotesingle activité.
   \item
    Sera maintenue confidentielle, sauf si nécessaire pour que le CII puisse accomplir ses tâches, pendant la durée de l'activité.
  \end{enumerate}
 \item
  Lors de la sélection des membres du CII, le Conseil d\textquotesingle administration de la FCÉG fera un effort pour s\textquotesingle assurer que la composition du CII reflète une gamme diversifiée de perspectives et d\textquotesingle origines.
 \item
  De plus, le Conseil d\textquotesingle administration s\textquotesingle efforcera de veiller à ce que les membres ne siègent pas à plusieurs CII consécutifs lorsque cela est possible.
 \item
  Pour atténuer les risques psychologiques associés à ce rôle, les membres du CII seront tenus de soumettre des plans de sécurité émotionnelle au Conseil d\textquotesingle administration de la FCÉG au plus tard trois (3) jours avant le début de l\textquotesingle activité.
 \item
  Les plans de sécurité émotionnelle doivent inclure, au minimum, ce qui suit :

  \begin{enumerate}
   \item
    Une reconnaissance du travail émotionnel associé au rôle et une volonté de participer à ce travail.
   \item
    Un aperçu des soutiens professionnels disponibles pour le membre du CII.
   \item
    Un aperçu des soutiens personnels disponibles pour le membre du CII.
   \item
    Les coordonnées des personnes identifiées par le membre du CII comme soutiens émotionnels clés.
  \end{enumerate}
 \item
  En cas de détresse émotionnelle extrême, les membres du CII peuvent utiliser les soutiens professionnels et personnels décrits dans leurs plans de sécurité émotionnelle pour discuter des préoccupations liées à leur travail en tant que membre du CII.
 \item
  Les membres du CII ne sont pas autorisés à discuter de leur rôle de membre du CII avec les autres délégués présents et sont soumis à la section 10.3.6 pour toutes les autres questions relatives à leur rôle.
 \item
  Le comité organisateur, en consultation avec le CII et l\textquotesingle exécutif national, est censé recueillir des ressources en santé mentale afin de les avoir disponibles pour l\textquotesingle activité afin de diriger les délégués vers elles pour un soutien professionnel.
 \item
  Il peut s\textquotesingle agir de services offerts par l\textquotesingle activité ou de ressources externes pouvant être offertes dans la ville ou la province de l\textquotesingle activité.
 \item
  S\textquotesingle il n\textquotesingle y a pas de ressources en santé mentale disponibles, le/la président.e du CII doit en informer le Conseil d\textquotesingle administration de la FCÉG afin que d\textquotesingle autres précautions puissent être prises.
\end{enumerate}

\subsubsection{L'autorité du comité}\label{lautorituxe9-du-comituxe9}

\begin{enumerate}
 \item
  Les décisions prises par le CII seront considérées comme finales pendant la durée de l\textquotesingle activité, et peuvent aller d\textquotesingle un simple avertissement verbal à l\textquotesingle expulsion de l'activité.
 \item
  Après la conclusion de l\textquotesingle activité, le/la défendeur.e peut faire appel des décisions prises par le CII.
 \item
  Les délégué.e.s ne peuvent pas être expulsé.e.s par une autre autorité de la FCÉG.
 \item
  Le CII ne peut pas prendre de décisions affectant la capacité du/de la délégué.e à contribuer à la FCÉG à l\textquotesingle avenir, comme le bannissement conditionnel ou complet des futures activités, mais peut recommander ces actions au comité d\textquotesingle appel.
\end{enumerate}

\subsubsection{La procédure de réponse du comité}\label{la-procuxe9dure-de-ruxe9ponse-du-comituxe9}

\begin{enumerate}
 \item
  Toutes les plaintes soumises via le Formulaire de rapport d'incident seront d\textquotesingle abord examinées par le/la président.e du CII, et celui-ci examinera le formulaire pour déterminer si certains membres du CII ne doivent pas avoir accès à la plainte en raison d\textquotesingle un conflit d\textquotesingle intérêts comme le prévoit l\textquotesingle article 10.3.5.
 \item
  Une fois cela déterminé, le/la président.e du CII partagera les réponses du formulaire avec tous les membres concernés du CII et commencera son enquête.
 \item
  Si une plainte est soumise à un autre membre du CII en raison d\textquotesingle un conflit d\textquotesingle intérêts avec le/la président.e du CII, la membre auquel la plainte a été soumise devra effectuer le même processus avant de partager la plainte avec les autres membres du CII.
 \item
  Pour s\textquotesingle assurer que le CII a une image complète de l\textquotesingle incident et de son impact sur les parties concernées, il devrait contacter les parties suivantes et discuter de l\textquotesingle incident et des réponses appropriées :

  \begin{enumerate}
   \item
    Le/les défendeur.e.s
   \item
    La/les personne.s affectée.s
   \item
    La personne ayant fait le signalement, le cas échéant
   \item
    Tout témoin(s) de l\textquotesingle incident, si absolument nécessaire
  \end{enumerate}
 \item
  Remarque : Le/la/les défendeur.e.s et la/les personne.s affecté.e.s seront invité.e.s à avoir une personne supplémentaire pour les soutenir lors de toutes les conversations.
 \item
  Après ces discussions, le CII utilisera son meilleur jugement sur la réponse appropriée ou s\textquotesingle il faut procéder à une enquête supplémentaire.
 \item
  Toutes les décisions du CII nécessitent un vote à la majorité, et le/la président.e du CII ne votera qu\textquotesingle en cas d\textquotesingle égalité.
 \item
  Toute communication officielle doit être faite par le/la président.e du CII, avec la connaissance et l\textquotesingle approbation de tous les membres du CII, sauf dans les cas où le/la président.e du CII a un conflit d\textquotesingle intérêts comme décrit à la section 10.3.5.
 \item
  Dans ce cas, le CII désignera l\textquotesingle un de ses membres pour faire les communications officielles concernant la plainte en question.
 \item
  Les réponses à toute communication officielle doivent être partagées avec tous les membres du CII.
 \item
  Le CII doit s\textquotesingle assurer qu\textquotesingle il y a au moins deux membres du CII présents pour toute conversation avec les personnes impliquées dans l\textquotesingle incident..
 \item
  Le CII devrait orienter les personnes impliquées vers les ressources de santé mentale mentionnées à la section 8.4.2.4.
\end{enumerate}

\subsubsection{Les conflits d'intérêt}\label{les-conflits-dintuxe9ruxeat}

\begin{enumerate}
 \item
  Un conflit d\textquotesingle intérêts est défini comme une relation personnelle forte positive ou négative entre un membre du CII et le/la défendeur.e, la victime, ou d\textquotesingle autres personnes impliquées dans l\textquotesingle incident, qui influence de manière injuste leurs actions ou décisions.
 \item
  Un membre du CII doit se retirer du CII s\textquotesingle il/elle estime qu\textquotesingle il a un conflit d\textquotesingle intérêts réel ou perçu.
 \item
  Si quelqu\textquotesingle un pense qu\textquotesingle un membre du CII a un conflit d\textquotesingle intérêt, il doit être présenté au CII.
 \item
  Si un membre du CII estime qu\textquotesingle un conflit d\textquotesingle intérêt réel ou perçu existe, le membre approprié sera retiré du CII .
 \item
  Si trois membres ou plus du CII doivent se retirer en raison d\textquotesingle un conflit d\textquotesingle intérêt, les membres restants du CII seront responsables de choisir un remplaçant de manière temporaire ou permanente, selon le cas, et de notifier le Conseil d\textquotesingle appel du changement.
 \item
  Les membres restants du CII doivent informer les délégués des activités des changements de membres s\textquotesingle il y a des changements permanents, et ces changements doivent être reflétés sur le formulaire de réponse aux incidents.
 \item
  Dans le cas de deux président.e.s du CII, si l\textquotesingle un/une des président.e.s a un conflit d\textquotesingle intérêt, l\textquotesingle autre président.e doit prendre en charge le cas sans consultation de l\textquotesingle autre président.e.
\end{enumerate}

\subsubsection{La publication d'informations}\label{la-publication-dinformations}

\begin{enumerate}
 \item
  Le CII est tenu de garder confidentielles de manière permanente toutes les informations d\textquotesingle identification de toute personne impliquée autre que le/la défendeur.e, sauf dans les situations suivantes :

  \begin{enumerate}
   \item
    Le CII ou toute personne impliquée dans l\textquotesingle incident décide que la police doit être informée, ou ;
   \item
    Le CII a obtenu le consentement écrit de toutes les personnes impliquées pour divulguer les informations.
  \end{enumerate}
 \item
  Le CII est tenu de garder confidentielles toutes les informations pouvant identifier le/la défendeur.e, sauf dans les situations suivantes :

  \begin{enumerate}
   \item
    Le CII ou toute personne impliquée dans l\textquotesingle incident décide que la police doit être informée, ou ;
   \item
    Le CII a obtenu le consentement écrit de toutes les personnes impliquées pour divulguer les informations, ou ;
   \item
    Le CII recommande que le Conseil d\textquotesingle administration informe l\textquotesingle école du/de la défendeur.e, c\textquotesingle est-à-dire le bureau du/de la doyen.ne, de la nature de la plainte.
  \end{enumerate}
 \item
  Le CII recommande que le Conseil d\textquotesingle administration limite ou mette fin à la participation d\textquotesingle une personne aux activités de la FCÉG.
 \item
  Si le Conseil d\textquotesingle administration (après la finalisation de l\textquotesingle appel) décide de maintenir cette recommandation, la société membre doit être informée du statut de la personne, mais pas des motifs de la décision.
 \item
  Si le CII décide de prendre des mesures à l\textquotesingle encontre d\textquotesingle un/une délégué.e, il doit en informer le/la délégué.e principal.e.
 \item
  Si la mesure est prise à l\textquotesingle encontre d\textquotesingle un/une délégué.e principal.e d'une délégation, l\textquotesingle ambassadeur.drice du/de la délégué.e doit en être informé.e.
 \item
  Si un membre du CII enfreint ces règles de confidentialité, il peut être passible de sanctions par le Conseil d\textquotesingle administration de la FCÉG.
\end{enumerate}

\subsubsection{Les meilleures pratiques}\label{les-meilleures-pratiques}

\begin{enumerate}
 \item
  Le CII sera informé par les conseiller.ères nationaux.ales sur les meilleures pratiques telles que définies dans le document sur les meilleures pratiques du Comité d\textquotesingle intervention des incidents.
 \item
  Les meilleures pratiques en matière de réponse aux incidents doivent être maintenues par l\textquotesingle exécutif national et les président.e.s du CII.
\end{enumerate}

\subsection{L'appel de décision}\label{lappel-de-duxe9cision}

\subsubsection{L'objectif}\label{lobjectif-3}

\begin{enumerate}
 \item
  Si quelqu\textquotesingle un estime qu\textquotesingle une décision prise par le Comité d\textquotesingle intervention des incidents (CII), ou que le processus suivi par le CII, était injuste, cette personne peut faire appel de la décision en informant le/la président.e du CII.
\end{enumerate}

\subsubsection{Le comité d'appel}\label{le-comituxe9-dappel}

\begin{enumerate}
 \item
  Le Conseil d\textquotesingle administration de la FCÉG entendra tous les appels résultant des décisions prises par le CII.
 \item
  Tout membre du Conseil d\textquotesingle administration qui faisait partie du CII devrait assister à la réunion où un appel est entendu pour fournir des informations, mais ne devrait pas participer à toute décision concernant l\textquotesingle appel.
 \item
  Le Conseil d\textquotesingle administration peut confirmer ou annuler les décisions prises par le CII, et peut accepter la recommandation du CII de limiter la participation d\textquotesingle un individu avec la FCÉG pour une période donnée, ou jusqu\textquotesingle à ce que certaines conditions soient remplies.
 \item
  Toutes les décisions prises par le Conseil d\textquotesingle administration sont finales.
\end{enumerate}

\subsubsection{L'échéance d'appel}\label{luxe9chuxe9ance-dappel}

\begin{enumerate}
 \item
  Les appels seront acceptés par le/la président.e du CII pendant 72 heures après la clôture de l\textquotesingle activité ou la notification d\textquotesingle une décision prise par le CII, selon la dernière occurrence.
 \item
  Une réunion sera programmée par le/la président.e du Conseil d\textquotesingle administration de la FCÉG dans les 14 jours suivant la présentation de l\textquotesingle appel pour entendre l\textquotesingle individu qui soumet l\textquotesingle appel et, séparément, le/la président.e du CII.
\end{enumerate}

\subsubsection{Le conflit d'intérêts}\label{le-conflit-dintuxe9ruxeats}

\begin{enumerate}
 \item
  Un membre du Conseil d\textquotesingle administration devrait se retirer du processus d\textquotesingle appel s\textquotesingle il croit avoir un conflit d\textquotesingle intérêts tel que défini dans la section 10.3.5.
\end{enumerate}

\subsection{L'enregistrement des informations}\label{lenregistrement-des-informations}

\begin{enumerate}
 \item
  Dans le cas où une école est informée de la décision du CII, ou qu\textquotesingle une décision de limiter ou de mettre fin à la participation d\textquotesingle un individu est prise, les éléments suivants doivent être enregistrés et conservés par les conseiller.ères nationaux.ales de la FCÉG pour les futures itérations du CII:

  \begin{enumerate}
   \item
    Le nom du/de la répondant.e.
   \item
    La décision qui a été prise.
   \item
    Les copies pertinentes de la correspondance avec l\textquotesingle école et/ou l\textquotesingle individu.
  \end{enumerate}
 \item
  Sinon, les informations suivantes doivent être enregistrées et conservées par les conseiller.ères nationaux.ales de la FCÉG pour les futurs CII:

  \begin{enumerate}
   \item
    Un résumé de haut niveau de l\textquotesingle incident comprenant les noms, les écoles, ou toute autre information d\textquotesingle identification.
   \item
    Une décision qui a été prise.
  \end{enumerate}
\end{enumerate}

