\section{Le Conseil d'administration}\label{le-conseil-dadministration}

\subsection{Élection du président du conseil d\textquotesingle administration}\label{uxe9lection-du-pruxe9sident-du-conseil-dadministration}

\begin{enumerate}
 \item
  Le Conseil d\textquotesingle administration doit choisir une personne qui sera président.e du Conseil d\textquotesingle administration pendant toute la durée du mandat de l\textquotesingle Exécutif national et qui présidera ses réunions.
 \item
  Le/la président.e du Conseil d'administration doit être inscrit auprès d'une association membre au moment de sa nomination.
 \item
  Cette personne sera proposée pour confirmation pour un mandat couvrant le reste de l\textquotesingle année fiscale lors de l\textquotesingle Assemblée générale tenue lors de le Sommet du développement des associations en ingénierie.
\end{enumerate}

\subsection{Le vice-président du conseil d\textquotesingle administration}\label{le-vice-pruxe9sident-du-conseil-dadministration}

\begin{enumerate}
 \item
  Le Conseil d\textquotesingle administration doit également désigner l\textquotesingle un de ses membres votants comme vice-président.e du Conseil d\textquotesingle administration, qui remplit les fonctions du président en son absence.
 \item
  Le/la président.e de la FCÉG n\textquotesingle est pas éligible au poste de vice-président.e du Conseil d\textquotesingle administration.
\end{enumerate}

\subsection{Responsabilités du / de la président.e du Conseil d'administration}\label{responsabilituxe9s-du-de-la-pruxe9sident.e-du-conseil-dadministration}

\begin{enumerate}
 \item
  Il/elle est chargé.e de convoquer les réunions conformément à la constitution, de distribuer les procès-verbaux pour approbation et de les soumettre, une fois approuvés, pour qu'ils soient affichés dans le site web.
 \item
  En plus de ses responsabilités à l'égard du Conseil, le/la président.e préside également l'assemblée générale. Dans le cadre de cet exercice, le/la président.e du Conseil d'administration est chargé de ce qui suit :

  \begin{enumerate}
   \item
    Veiller, en collaboration avec le/la responsable des activités approprié, à ce qu'on ait prévu le temps et les ressources nécessaires au déroulement de toutes les délibérations.
   \item
    Veiller à ce que les membres soient informés du moment des réunions, comme l'exige la constitution.
   \item
    Veiller à ce que les membres soient adéquatement préparés pour une réunion en leur remettant le procès-verbal de la réunion précédente et d'autres documents pertinents.
   \item
    Veiller à la mise en place d'un processus permettant aux membres de présenter des propositions et des contre-propositions.
   \item
    Veiller à ce qu'un ordre du jour préliminaire soit distribué comme demandé dans le cadre de chaque assemblée générale.
  \end{enumerate}
 \item
  En l'absence du/de la président.e du Conseil d'administration, c'est le/la vice-président.e du Conseil d'administration qui s'acquitte des tâches énumérées ci-dessus et de celles qui autrement sont confiées au/à la président.e du Conseil.
\end{enumerate}

\subsection{Les réunions}\label{les-ruxe9unions}

\subsubsection{Les réunions régulières}\label{les-ruxe9unions-ruxe9guliuxe8res}

\begin{enumerate}
 \item
  Les réunions du Conseil d'administration sont tenues à un moment et à un lieu déterminés par les administrateurs, pourvu qu'un avis verbal ou écrit de cinq (5) jours en soit donné à chaque membre.
 \item
  Le Conseil doit tenir au moins six (6) réunions par année.
 \item
  Chaque directeur.trice présent à une réunion dispose du droit d'exercer un (1) vote.
 \item
  Si tous les directeur.trice.s de la Fédération y consentent, un ou plusieurs d'entre eux peuvent participer à une réunion du Conseil en utilisant un moyen de communication technologique, comme le téléphone, pour permettre à tous les participants de s'entendre.
 \item
  Les directeur.trice.s participant à une telle réunion par ce moyen sont réputés être présents.
 \item
  Les directeur.trice.s n'ont pas le droit de voter par procuration.
\end{enumerate}

\subsubsection{Les réunions d'urgence}\label{les-ruxe9unions-durgence}

\begin{enumerate}
 \item
  Il est possible de convoquer une réunion d'urgence du Conseil d'administration pour des besoins particuliers ou une urgence qui exigent une intervention immédiate, pourvu que les deux tiers des directeur.trice.s acceptent de convoquer la réunion.
\end{enumerate}

\subsubsection{Le quorum}\label{le-quorum}

\begin{enumerate}
 \item
  Pour qu'il y ait quorum à une réunion du Conseil d'administration, il faut que les deux tiers (⅔) des directeur.trice.s soient présents.
\end{enumerate}

\subsection{Le vote électronique}\label{le-vote-uxe9lectronique}

\begin{enumerate}
 \item
  Le Conseil d'administration peut adopter une résolution par voie électronique si tous les directeur.trice.s habilités à voter sur cette résolution dans le cadre d'une réunion votent en sa faveur.
 \item
  La résolution est aussi valide que si elle avait été adoptée au cours d'une réunion en personne.
\end{enumerate}

\subsection{Les pouvoirs de nomination}\label{les-pouvoirs-de-nomination}

\subsubsection{Les comités}\label{les-comituxe9s}

\begin{enumerate}
 \item
  Le Conseil d'administration peut constituer un comité où les membres exerceront leurs fonctions au gré du Conseil.
\end{enumerate}

\subsubsection{Les mandataires et les employés}\label{les-mandataires-et-les-employuxe9s}

\begin{enumerate}
 \item
  Le Conseil d'administration peut nommer des mandataires et engager des employés lorsque cela semble nécessaire; au moment de leur nomination, ces personnes exerceront le pouvoir et exécuteront les tâches comme prescrit par le Conseil.
 \item
  Le Conseil d'administration établira pour tous les employés de la Fédération une rémunération raisonnable par résolution, comme le permet le budget.
\end{enumerate}

\subsection{Les pouvoirs d'administration}\label{les-pouvoirs-dadministration}

\begin{enumerate}
 \item
  Les directeur.trice.s de la Fédération peuvent administrer ses affaires, pour tout ce qui la concerne, établir en son nom tout type de contrat qu'elle a le droit de passer ou en demander l'établissement et en règle générale, sauf dans les cas ci-dessous prévus, exercer tous ces pouvoirs et accomplir toute autre activité que la Fédération, aux termes de ses Statuts, est autorisée à exercer et à accomplir.
\end{enumerate}

\subsection{La rémunération}\label{la-ruxe9munuxe9ration}

\begin{enumerate}
 \item
  Les directeur.trice.s du Conseil occupent leur poste sans rémunération et aucun directeur.trice ne doit retirer directement ou indirectement de bénéfice du fait qu'il occupe ce poste, pourvu que les dépenses raisonnables engagées dans l'exercice de ses fonctions soient remboursées.
\end{enumerate}

\subsection{La formation en EDBA}\label{la-formation-en-edba}

\begin{enumerate}
 \item
  À chaque année, le Conseil d'administration suivra des formations certifié liés à l\textquotesingle EDBA avant le 1er août.
 \item
  Ces formations seront organisées par les conseillers nationaux de la FCÉG et enseignées par une organisation externe.
 \item
  Cette formation ou ces formations doivent porter au moins sur les sujets suivants :
  \begin{enumerate}
 \item
  Capacitisme, intellectuel et physique
 \item
  Classisme
 \item
  Consentement
 \item
  Culture de l\textquotesingle alcool
 \item
  Genre et diversité sexuelle
 \item
  Multiculturalisme, y compris la culture et la religion
 \item
  Racisme
 \item
  La vérité et de la réconciliation
 \item
  Les effets de l'intersectionnalité entre les thèmes énumérés ci-dessus
 \item
  Premier Soins Psychologiques
\end{enumerate}
\end{enumerate}

\subsection{Les postes à pourvoir}\label{les-postes-uxe0-pourvoir}

\begin{enumerate}
 \item
  Le poste d'un.e directeur.trice ou d'une directrice devient automatiquement vacant s'il démissionne, devient invalide ou décède.
 \item
  Un.e directeur.trice, officier ou commissaire sortant demeurera en fonction jusqu'à ce que la vacance de son poste soit soumise au Conseil d'administration.
\end{enumerate}

