\section{L'assemblée générale}\label{lassembluxe9e-guxe9nuxe9rale}

\subsection{L'assemblée générale annuelle}\label{lassembluxe9e-guxe9nuxe9rale-annuelle}

\begin{enumerate}
 \item
  L\textquotesingle assemblée générale annuelle (AGA) de la Fédération se tient au CCLI en janvier de chaque année.
 \item
  Cette assemblée générale comprend au moins deux sessions, avec au moins un jour entre les sessions pour rédiger des contre-propositions.
 \item
  L\textquotesingle AGA donne aux membres l\textquotesingle occasion de participer au processus décisionnel de l\textquotesingle organisation.
 \item
  Elle est présidée par le/la président.e du Conseil d\textquotesingle administration et est modérée selon la dernière édition des \emph{Rules of Order} de Robert, nouvellement révisées.
 \item
  Pour participer à ces réunions, les membres doivent fournir au président une procuration actuelle et originale, déléguant leurs droits de vote à un individu ou à un autre membre, au moins une semaine avant une assemblée générale.
\end{enumerate}

\subsection{Le Sommet du développement des associations en ingénierie}\label{le-sommet-du-duxe9veloppement-des-associations-en-inguxe9nierie}

\begin{enumerate}
 \item
  Le Sommet du développement des associations en ingénierie (SDAI) est un rassemblement de membres, en plus du Conseil d\textquotesingle administration de la FCÉG et d\textquotesingle autres membres de l\textquotesingle équipe des officiers, pour discuter des affaires de la Fédération, collaborer et réseauter.
 \item
  Le Sommet du développement des associations en ingénierie a lieu fin septembre ou début octobre, et l\textquotesingle hôte est élu au CCLI de l\textquotesingle année académique précédente.
\end{enumerate}

\subsection{Sommet canadien d'ingénierie du printemps}\label{sommet-canadien-dinguxe9nierie-du-printemps}

\begin{enumerate}
 \item
  Le sommet canadien d'ingénierie du printemps (SCIP) est un rassemblement en ligne de membres, en plus du Conseil d\textquotesingle administration de la FCÉG et d\textquotesingle autres membres de l\textquotesingle équipe des dirigeants, pour discuter des affaires de la fédération, collaborer et réseauter.
 \item
  Le sommet du printemps a lieu en mars.
\end{enumerate}

\subsection{Le bootcamp d'été (Réunions non-générales)}\label{le-bootcamp-duxe9tuxe9-ruxe9unions-non-guxe9nuxe9rales}

\begin{enumerate}
 \item
  Le bootcamp est un rassemblement en ligne de membres, en plus du Conseil d\textquotesingle administration de la FCÉG et d\textquotesingle autres membres de l\textquotesingle équipe des dirigeants, dont le but est d\textquotesingle organiser des sessions de formation et des discussions, ainsi que de réseauter.
 \item
  Le bootcamp n\textquotesingle accueillera pas d\textquotesingle assemblée générale et durera une journée.
 \item
  Le bootcamp a lieu en juin ou juillet.
\end{enumerate}

\subsection{Les Mises à jour saisonnières en ligne (MAJS-L)}\label{les-mises-uxe0-jour-saisonniuxe8res-en-ligne-majs-l}

\begin{enumerate}
 \item
  La FCÉG tiendra deux assemblées générales à distance pour combler le fossé entre les assemblées générales en personne.
 \item
  Ces réunions sont des rassemblements en ligne des membres et de l\textquotesingle équipe des dirigeants.
 \item
  L\textquotesingle objectif de ces réunions est de permettre au Conseil d\textquotesingle administration de la FCÉG et aux responsables d\textquotesingle activités de donner des mises à jour aux membres sur les progrès de leurs portfolios, projets et travaux de plaidoyer.
 \item
  Ces réunions incluront également des opportunités de retour d\textquotesingle information des membres.
 \item
  Ces deux réunions n\textquotesingle accueilleront pas d\textquotesingle assemblée générale et seront programmées pour une durée d\textquotesingle environ deux à trois heures, l\textquotesingle une ayant lieu en juillet et l\textquotesingle autre en novembre.
 \item
  L\textquotesingle organisation de ces réunions et des ordres du jour relève de la responsabilité du président du Conseil d\textquotesingle administration avec le soutien des membres appropriés de l\textquotesingle équipe des officiers.
\end{enumerate}

\subsection{Les lignes directrices pour les réunions}\label{les-lignes-directrices-pour-les-ruxe9unions}

\subsubsection{Les appels de réunion}\label{les-appels-de-ruxe9union}

\begin{enumerate}
 \item
  Une assemblée générale sera convoquée sur résolution du Conseil d\textquotesingle administration ou sur demande écrite d\textquotesingle au moins un quart (1/4) des membres actifs.
\end{enumerate}

\subsubsection{La traduction simultanée}\label{la-traduction-simultanuxe9e}

\begin{enumerate}
 \item
  À chaque assemblée générale, des services de traduction simultanée doivent être disponibles, qu\textquotesingle ils soient externalisés professionnellement ou gérés par des membres.
 \item
  Si externalisés, ces services seront fournis aux frais du comité organisateur de l\textquotesingle assemblée générale concernée.
\end{enumerate}

\subsubsection{L'ordre du jour de la réunion}\label{lordre-du-jour-de-la-ruxe9union}

\begin{enumerate}
 \item
  À chaque assemblée générale annuelle, en plus de toute autre affaire pouvant être traitée, le rapport des directeur.trice.s, l\textquotesingle état financier et le rapport des vérificateurs seront présentés.
\end{enumerate}

\subsubsection{L'avis de réunion}\label{lavis-de-ruxe9union}

\begin{enumerate}
 \item
  Un avis écrit de 30 jours sera donné à chaque membre de toute Assemblée générale.
 \item
  L\textquotesingle avis de toute réunion où des affaires spéciales seront traitées doit contenir suffisamment d\textquotesingle informations pour permettre au membre de former un jugement motivé sur la décision à prendre.
 \item
  Aucune erreur ou omission dans la remise d\textquotesingle un avis de réunion d\textquotesingle une assemblée de membres ou d\textquotesingle une assemblée de membres ajournée de la FCÉG invalidera cette réunion ni ne rendra nulles les procédures qui y ont été suivies, et tout membre peut à tout moment renoncer à l\textquotesingle avis de toute réunion et peut ratifier, approuver et confirmer toutes les procédures qui y ont été suivies.
 \item
  Aux fins d\textquotesingle envoi d\textquotesingle un avis à un membre, un.e directeur.trice ou un officier pour toute réunion ou autre, l\textquotesingle adresse du membre, du/de la directeur.trice.s. ou de l\textquotesingle officier sera celle de la dernière adresse enregistrée dans les registres de la FCÉG.
\end{enumerate}

\subsubsection{Le vote et le quorum}\label{le-vote-et-le-quorum}

\begin{enumerate}
 \item
  Chaque membre actif a droit à un (1) vote.
 \item
  Toutes les questions soumises à l\textquotesingle assemblée générale seront déterminées par la majorité des voix des membres présents et votants, sauf disposition contraire spécifiquement prévue par des résolutions ou mentionnée dans la présente constitution.
 \item
  Les deux tiers (2/3) des membres actifs constitueront le quorum lors d\textquotesingle une assemblée générale.
\end{enumerate}

\subsubsection{Les droits de parole}\label{les-droits-de-parole}

\begin{enumerate}
 \item
  Les membres actifs, les ambassadeurs.drices régionaux.ales, les conseiller.ères nationaux.ales et les représentants de l'exécutif national ont le droit de parole pendant l\textquotesingle assemblée.
\end{enumerate}

\subsubsection{L'ordre}\label{lordre}

\begin{enumerate}
 \item
  Lors de toutes les assemblées de la Fédération, le/la président.e doit appliquer les règles des assemblées délibérantes telles qu\textquotesingle elles sont définies dans les \emph{Rules of Order} de Robert, sauf dispositions contraires adoptées par la Fédération.
 \item
  Le/la président.e du Conseil d\textquotesingle administration préside les réunions de l\textquotesingle assemblée générale.
 \item
  Lors d\textquotesingle une assemblée générale, l\textquotesingle exécutif national et chaque directeur.trice ont le droit de proposer des motions sans nécessité de seconder.
\end{enumerate}

\subsubsection{La procuration}\label{la-procuration}

\begin{enumerate}
 \item
  Un membre peut, par l\textquotesingle intermédiaire d\textquotesingle une communication vérifiable par son/sa président.e, soit originale, par télécopie ou électronique, nommer un mandataire pour assister et agir lors d\textquotesingle une réunion spécifique de l\textquotesingle assemblée générale.
 \item
  L\textquotesingle avis de chaque réunion de l\textquotesingle assemblée générale doit rappeler aux membres leur capacité de nommer un mandataire.
 \item
  Tous les votes par procuration doivent être différés à un membre actif du délégué ou à un autre membre actif de la FCÉG.
\end{enumerate}

\subsubsection{Le procès-verbal}\label{le-procuxe8s-verbal}

\begin{enumerate}
 \item
  Des procès-verbaux doivent être rédigés en anglais et en français lors de toutes les réunions de l\textquotesingle assemblée générale.
 \item
  Le texte complet et approprié des procès-verbaux doit être distribué à tous les membres en anglais et en français dans les trente jours suivant la clôture de la réunion.
\end{enumerate}

\subsection{Le vote par correspondance}\label{le-vote-par-correspondance}

\subsubsection{Les articles inéligibles}\label{les-articles-inuxe9ligibles}

\begin{enumerate}
 \item
  Les éléments suivants ne peuvent pas être approuvés par un vote par correspondance :

  \begin{enumerate}
   \item
    Les modifications constitutionnelles
   \item
    L'annulation de motions
  \end{enumerate}
\end{enumerate}

\subsubsection{La procédure}\label{la-procuxe9dure}

\begin{enumerate}
 \item
  Les bulletins de vote envoyés par la poste ou toute communication officielle (en supposant que les exigences de confidentialité fixées par l'exécutif national sont respectées), ci-après appelés "bulletin de vote par correspondance", sont valides pour l\textquotesingle assemblée générale ou le Conseil d\textquotesingle administration, sauf dans le cas où la loi canadienne sur les organismes sans but lucratif interdit une telle action.
 \item
  Le Conseil d\textquotesingle administration convoquera un vote par correspondance sur demande écrite d\textquotesingle au moins un quart (1/4) des membres actifs.
 \item
  Un vote par correspondance est soumis aux mêmes directives et réglementations qu\textquotesingle une assemblée générale annuelle, que ce soit pour clarifier le quorum ou le résultat de la proposition.
 \item
  Le Conseil d\textquotesingle administration déterminera toute règle supplémentaire.
 \item
  Pour être valide, les bulletins de vote par correspondance doivent être retournés à la fédération dans les 30 jours suivant l\textquotesingle envoi de l\textquotesingle avis de vote à tous les membres actifs.
 \item
  Le Conseil d\textquotesingle administration a, entre le 35e et le 50e jour suivant l\textquotesingle envoi de l\textquotesingle avis de vote, pour envoyer à tous les membres et directeur.trices les résultats du vote par correspondance.
 \item
  Toute erreur ou omission dans l\textquotesingle envoi de l\textquotesingle avis de vote du vote par correspondance n\textquotesingle annulera pas le vote.
 \item
  Les membres actifs recevront l\textquotesingle avis de vote à leur dernière adresse connue dans les registres de la Fédération.
 \item
  Toutes les motions adoptées par un vote par correspondance sont soumises à ratification lors de la prochaine réunion de l\textquotesingle assemblée générale.
\end{enumerate}

\subsection{Les vidéos à consulter avant l'assemblée générale}\label{les-viduxe9os-uxe0-consulter-avant-lassembluxe9e-guxe9nuxe9rale}

\begin{enumerate}
 \item
  Le comité exécutif national de la FCÉG créera des vidéos avant chaque activité impliquant une session d\textquotesingle assemblée générale pour s\textquotesingle assurer que les membres sont informés des sujets pertinents à l\textquotesingle{} activité, y compris mais sans s\textquotesingle y limiter, les politiques discutées et les mandats en cours.
 \item
  La durée totale de ces vidéos ne doit pas dépasser 30 minutes, et les vidéos doivent être fournies au moins une (1) semaine avant le début de l'activité.
\end{enumerate}

