\section{L'exécutif national}\label{lexuxe9cutif-national}

\subsection{Le survol de l'exécutif national}\label{le-survol-de-lexuxe9cutif-national}

\begin{enumerate}
 \item
  L'exécutif national est l'équipe de dirigeants élus de la Fédération ; il se compose des cinq postes suivants :
  \begin{enumerate}
   \item
    Le/la président.e
   \item
    Le/la vice-président.e académique
   \item
    Le/la vice-président.e de finance et de l'administration
   \item
    Le/la vice-président.e externe
   \item
    Le/la vice-président.e des services
   \item
    Le/la vice-président.e des communications
  \end{enumerate}
 \item
  Voici quelques-unes des tâches et des responsabilités de l'exécutif national

  \begin{enumerate}
   \item
    Imposer le respect de la constitution et du Manuel des politiques de la Fédération
   \item
    Mettre sur pied et diriger les initiatives de la Fédération
   \item
    S'assurer que tous les mandats sont complétés
   \item
    Encadrer les commissaires selon les limites de leurs portefeuilles respectifs, et assumer leurs responsabilités au cas où elles ne seraient pas remplies
   \item
    S'assurer d'entretenir la communication avec les écoles membres en :

    \begin{enumerate}
     \item
      veillant à ce que toutes les communications officielles soient bilingues et non-genrées
     \item
      veillant à ce que la mise à jour sur les tâches effectuées par tous les officiers, commissaires et directeur.trice.s soit dressée, puis distribuée aux membres au moins tous les deux mois
     \item
      veillant à ce qu'un rapport sur les résultats soit rédigé pour chacune des réunions où chacun des événements extérieurs auxquels un officier, un commissaire ou un.e directeur.trice a pu assister
    \end{enumerate}
   \item
    Prendre contact avec des organismes gouvernementaux lorsque nécessaire
   \item
    Faire l'embauche des groupes de travails pour l'exécutif national entrant au plus tard le 30 avril
   \item
    La rédaction d'un document de transition sommaire au plus tard le 30 avril ; ce document sera stocké dans les archives électroniques de la FCEG
  \end{enumerate}
\end{enumerate}

\subsection{Le/la président.e}\label{lela-pruxe9sident.e}

\begin{enumerate}
 \item
  Voici quelques-unes des tâches et des responsabilités du/de la président.e :
  \begin{enumerate}
   \item
    S'assurer que le mandat de chaque vice-président.e est mené à bien
   \item
    Coordonner les activités du Conseil d'administration avec le/la président.e du Conseil d'administration
   \item
    Faciliter le développement de la Fédération et celui de ses membres
   \item
    S'assurer que la planification à long terme de la Fédération est lancée, faisant le point une fois par année à l'intention des membres
   \item
    Entretenir des voies de communication avec toutes les parties prenantes de la Fédération
   \item
    Veiller à ce que les plans d'actions soient complétés par chacun des membres de l'exécutif national; ces plans d'action doivent être distribués au Conseil d'administration avant le Sommet du développement des associations en ingénierie (SDAI).
   \item
    Préparer un document de transition avant le SCIP.
   \item
    Veiller à ce que chaque membre de l\textquotesingle exécutif national prépare un document de transition avant le SCIP.
   \item
    Faire la promotion des prix, rassembler les prix, remettre les prix et former un comité des prix dans le cadre du gala annuel du CCLI
   \item
    Assurer la communication avec Ingénieurs Canada et assister à son assemblée générale annuelle ainsi qu'aux réunions de son conseil, dans la mesure du possible.
   \item
    Assister aux activités, sinon s'y faire représenter de manière convenable :

    \begin{enumerate}
     \item
      L'assemblée générale de BEST
     \item
      La réunion du président de BEST
    \end{enumerate}
  \end{enumerate}
\end{enumerate}

\subsection{Le/la vice-président.e académique (VPA)}\label{lela-vice-pruxe9sident.e-acaduxe9mique-vpa}

\begin{enumerate}
 \item
  Voici quelques-unes des tâches et des responsabilités du/de la VPA:
  \begin{enumerate}
   \item
    Acquérir une compréhension des initiatives académiques, de l'expérience de travail intégrée, du contenu des programmes d'études, des normes et des exigences d'agrément, de même que de tout changement qui influerait sur la formation des ingénieurs.
   \item
    Déterminer les besoins académiques, moraux, intellectuels, culturels, sociaux et économiques de étudiants en génie, contribuer à l'opération de changements si cela s'avère nécessaire et offrir une force de levier à l'échelle nationale dans les négociations entre les représentants étudiants et l'administration des société membres et d'autres organisations impliquées.
   \item
    Communiquer l'information et les enjeux pertinents à l'éducation en génie ou aux étudiant en génie au Canada aux société membres et aux Doyennes et doyens d'ingénierie au Canada (DDIC).
   \item
    Faciliter l'échange d'information académique important ou des actions de plaidoyer entre les sociétés membres et les organisations nationales d'éducation.
   \item
    Diriger les politiques et les procédures qui découlent du document de position, présenter les positions convenables au nom de l'exécutif national et mener des recherches sur l'incidence d'éventuelles positions, celles exposées par l'exécutif national comme celles exposées par des membres, et présenter ces conclusions aux membres.
   \item
    Diriger le groupe de travail sur la défense des intérêts, notamment organiser et présider les réunions, établir les objectifs et les directives ainsi que superviser le travail des responsables régionaux de la défense des intérêts.
   \item
    Envoyer un questionnaire aux membres au début de chaque mandat afin d'établir des priorités annuelles, et de partager ces résultats avec les membres pour encourager une collaboration entre les sociétés membres.
   \item
    Créer et garder des Lettres de support lorsque demandé par les membres
  \end{enumerate}
 \item
  Pour soutenir le/la VPA, il est recommandé que le commissaire suivant soit disponible, sous sa direction, et qu'il lui fasse rapport:

  \begin{enumerate}
   \item
    Le/la commissaire des données
  \end{enumerate}
\end{enumerate}

\subsection{Le/la vice-président.e de finance et de l'administration (VPFA)}\label{lela-vice-pruxe9sident.e-de-finance-et-de-ladministration-vpfa}

\begin{enumerate}
 \item
  Voici quelques-unes des tâches et des responsabilités du/de la VPFA :
  \begin{enumerate}
   \item
    Agir à titre de trésorier de la FCÉG en :

    \begin{enumerate}
     \item
      émettant les paiements
     \item
      déposant les fonds qui entrent
     \item
      suivant tous les actifs financiers
     \item
      tenant un registre complet et exact de tous les dossiers financiers des exercices précédents
     \item
      réalisant et tenant des états financiers historiques (l'état des résultats, le bilan et l'état des flux de trésorerie)
    \end{enumerate}
   \item
    Assurer la passation opportune des pouvoirs de signature sur les comptes de la FCÉG à la fin de l'année fiscale.
   \item
    Encaisser les chèques représentant les frais d'adhésion au plus un mois après leur réception; envoyer un avis indiquant que le chèque a été encaissé dans les deux semaines suivant l'encaissement
   \item
    Dresser une liste de renseignements concernant la perception des frais d'adhésion auprès des membres pour l'exercice en cours et le précédent
   \item
    Faire le comptage du nombre d'étudiants représentés par les membres actifs tels que spécifiés dans la section 6.1, et ce, à chaque année
   \item
    Soumettre un projet de budget détaillé à l'exécutif national entrant

    \begin{enumerate}
     \item
      Offrir des modifications au budget une fois que l'équipes des responsables est établi pour la prochaine année
    \end{enumerate}
   \item
    Étayer les revenus et les dépenses à l'aide de documents applicables, y compris des reçus, des factures, des procès-verbaux approuvant les dépenses et tout autre document au besoin
   \item
    Composer avec tout ce qui touche l'incorporation, y compris le dépôt de la déclaration annuelle auprès de Corporations Canada chaque année, en utilisant le numéro 0233129, et joindre toute révision à la constitution le cas échéant
   \item
    S'assurer que les marques de commerce actuelles sont protégées
   \item
    Si 75\% de la FVO est dispensé par le SDAI de l'année du budget, le/la VPFA doit redistribuer les fonds du budget et s'assurer qu'il soit approuvé durant l'AG du SDAI, pour que le FVO puisse continuer à être utilisé durant l'année du budget.
   \item
    Publier à l'intention du Conseil d'administration le procès-verbal des réunions de l'exécutif national dans les 30 jours qui suivent.
   \item
    Avant la réunion du printemps, le/la VPFA doit évaluer les sociétés membres de chaque responsable ainsi que le lieu de l'activité et proposer des changements, si nécessaire, pour s'assurer d'avoir assez d'argent dans leurs fonds.
   \item
    Afin d'assurer la responsabilité et la transparence, le VPFA doit compléter les livrables financiers énoncés à la section 8.10 des règlements administratifs.
  \end{enumerate}
\end{enumerate}

\subsubsection{Le/la vice-président.e sortant.e de finance et de l'administration}\label{lela-vice-pruxe9sident.e-sortant.e-de-finance-et-de-ladministration}

\begin{enumerate}
 \item
  Les tâches et responsabilités du/de la vice-président.e sortant.e de finance et de l\textquotesingle administration sont les suivantes
  \begin{enumerate}
   \item
    Travailler aux côtés du/de la vice-président.e des finances et de l\textquotesingle administration au cours des premiers mois de son mandat.
   \item
    Finaliser les états financiers pour l\textquotesingle année fiscale de l\textquotesingle exécutif national sortant.
   \item
    Veiller à l\textquotesingle achèvement de l\textquotesingle audit pour l\textquotesingle exercice financier de l\textquotesingle exécutif national sortant.
  \end{enumerate}
\end{enumerate}

\subsection{Le/la vice-président.e externe (VPE)}\label{lela-vice-pruxe9sident.e-externe-vpe}

\begin{enumerate}
 \item
  Voici quelques-unes des tâches et des responsabilités du/de la VPE:
  \begin{enumerate}
   \item
    Servir de personne-ressource principale de la FCÉG auprès de nos partenaires nationaux et internationaux, en plus d'assurer la liaison avec les partenaires actuels et futurs au nom de la Fédération
   \item
    Étudier les nouveaux projets et les idées de coopération avec des partenaires nationaux et internationaux, existants et à venir
   \item
    S'assurer que tous les partenaires et ceux qui se préparent à le devenir sont invités à tous les événements pertinents
   \item
    Consulter les représentants des membres de la FCÉG lorsqu'il ou elle est confrontée à une décision importante
   \item
    S'assurer que tous les membres de la FCÉG sont au courant des occasions à l'échelle internationale et des avantages de la coopération internationale
   \item
    Faire le point régulièrement sur les projets, les initiatives et les travaux en cours importants des partenaires, ainsi que sur les résultats de ces partenariats
   \item
    Participer à des réunions et à d'autres événements précisés dans les ententes de partenariat, sinon trouver le représentant qui convient le mieux, et rédiger un rapport sur les résultats dans un délai de deux mois
   \item
    Agir à titre de personne-ressource principale pour \emph{EngiQueers} Canada et Ingénieurs Canada, coordonner sa participation aux événements de la FCÉG, de même qu'aux projets collaboratifs et au plaidoyer.
   \item
    Plaidoyer auprès des organizations externe selon notre Cahier des positions et en collaboration avec le/la VPE
   \item
    Développer et maintenir des contacts avec d'autres organisations éducatifs en génie national ou régional
  \end{enumerate}
 \item
  Pour soutenir le/la VPE, il est recommandé que les commissaires suivant soient disponible, sous sa direction, et qu'ils lui fasse rapport:

  \begin{enumerate}
   \item
    Le/la commissaire aux relations d'entreprise
  \end{enumerate}
\end{enumerate}

\subsection{Le/la vice-président.e des services (VPS)}\label{lela-vice-pruxe9sident.e-des-services-vps}

\begin{enumerate}
 \item
  Les tâches et responsabilités du/de la vice-président.e des services sont les suivantes:
  \begin{enumerate}
   \item
    Planifier, programmer et organiser des réunions de contrôle mensuelles avec les responsables d\textquotesingle activités.
   \item
    S\textquotesingle assurer que les accords d\textquotesingle activité sont signés avant l\textquotesingle acceptation de l\textquotesingle offre.
   \item
    Répondre aux demandes des responsables d\textquotesingle activité dans un délai raisonnable et les aider à diriger leurs questions vers les membres appropriés de l\textquotesingle équipe de la FCÉG.
   \item
    Apporter un soutien logistique et moral aux activités de la FCÉG, principalement aux responsables d\textquotesingle activité.
   \item
    Veiller à ce que les responsabilités définies dans les accords d\textquotesingle activité soient remplies en se référant au document sur les meilleures pratiques du/de la VPS et en respectant les délais indiqués dans l\textquotesingle accord.
   \item
    Former le commissaire à la logistique aux fonctions du/de la VPS et le soutenir dans ses fonctions.
   \item
    Développer en permanence le module de formation sur l\textquotesingle accessibilité en collaboration avec le commissaire du leadership et du développement.
   \item
    Encourager les sociétés membres à se porter candidates et à accueillir des activités de la FCÉG
   \item
    Travailler avec les conseillers nationaux pour s\textquotesingle assurer que le comité de réponse aux incidents (CRI) est mis en œuvre lors de chaque activité de la FCÉG.
   \item
    Distribuer un formulaire de feedback des délégués sur l\textquotesingle activité à chaque responsable d\textquotesingle activité au plus tard dix jours après la fin de chaque activité de la FCÉG.Distribuer un formulaire de feedback sur les enseignements tirés par le comité d\textquotesingle organisation au plus tard dix jours après la fin de chaque activité de la FCÉG.

    \begin{enumerate}
     \item
      Les résultats doivent être communiqués au Conseil d\textquotesingle administration.
    \end{enumerate}
  \end{enumerate}
\end{enumerate}

\subsection{Le/la vice-président.e des communications (VPC)}\label{lela-vice-pruxe9sident.e-des-communications-vpc}

\begin{enumerate}
 \item
  Voici quelques-unes des tâches et des responsabilités du/de la VPC :

  \begin{enumerate}
   \item
    Tenir à jour des archives de fichiers originaux relatifs au logo, aux bannières, aux articles de marque et aux modèles
   \item
    Maintenir en vigueur et faire respecter les directives sur le logo et l'image de marque de la FCÉG
   \item
    Aider les organisateurs des activités, des programmes et des services à réaliser des modèles de conception destinés aux documents ou au matériel promotionnel, au besoin
   \item
    Demander des soumissions en ce qui a trait aux articles de marque destinés à la FCÉG ou aux responsables des activités, au besoin
   \item
    Mettre à jour le profil de la Fédération dans les médias sociaux
   \item
    Diriger le groupe de travail des médias et du marketing, notamment en organisant et en présidant les réunions, en établissant les grands objectifs et les orientations, ainsi qu'en supervisant le travail de ses membres
   \item
    Diriger le groupe de travail du bilinguisme, notamment en organisant et en présidant les réunions, en établissant les grands objectifs et les orientations, ainsi qu'en supervisant le travail de ses membres
   \item
    Faciliter les échanges de points de vue entre associations membres tout au long de l'année, y compris dans le cadre du Mois national du génie
   \item
    Publier le numéro annuel de la revue \emph{Projet}
   \item
    Envoi de publications mensuelles (format à leur discrétion) aux sociétés d\textquotesingle ingénierie membres afin de les tenir au courant des activités de la FCÉG.
   \item
    Préparer un plan d\textquotesingle action sur les efforts et les améliorations du bilinguisme conformément au Plan stratégique.

    \begin{enumerate}
     \item
      L\textquotesingle état d\textquotesingle avancement du plan d\textquotesingle action doit être partagé avec les membres deux fois par an lors du SDAI et SCIP.
    \end{enumerate}
  \end{enumerate}
 \item
  Pour soutenir le/la VPC, il est recommandé que les commissaires suivant soit disponible, sous sa direction, et qu'il lui fasse rapport:

  \begin{enumerate}
   \item
    Le/la commissaire des médias et du marketing
   \item
    Le/la commissaire au bilinguisme
  \end{enumerate}
\end{enumerate}

\subsection{Les réunions}\label{les-ruxe9unions-1}

\begin{enumerate}
 \item
  Les réunions de l'exécutif national sont tenues à un moment et à un lieu déterminés par ses membres, à condition qu'un avis verbal ou écrit de deux (2) jours en soit donné à chaque membre.
 \item
  Pour qu'il y ait quorum à une réunion de l'exécutif national, il faut que les quatre (4) membres soient présents.
 \item
  Chaque membre présent à une réunion dispose du droit d'exercer (1) un vote.
 \item
  Si tous les membres du Conseil national y consentent, un ou plusieurs dirigeants peuvent participer à une réunion en utilisant un moyen de communication technologique, comme le téléphone, permettant à tous les participants de s'entendre.
 \item
  Les membres de l'exécutif national participant à une telle réunion par ce moyen sont réputés être présents
\end{enumerate}

\subsection{Le renvoi}\label{le-renvoi}

\begin{enumerate}
 \item
  Un membre de l\textquotesingle exécutif national peut être révoqué par une résolution du Conseil d\textquotesingle administration adoptée par les deux tiers (2/3) des administrateurs présents à une réunion convoquée à cet effet ou par une résolution adoptée par les deux tiers (2/3) des membres de l\textquotesingle assemblée générale présents à une réunion convoquée à cet effet.
\end{enumerate}

\subsection{Les postes à pourvoir}\label{les-postes-uxe0-pourvoir-1}

\begin{enumerate}
 \item
  Si un poste de dirigeant devient vacant, cette vacance doit être communiquée aux membres dans un délai de cinq (5) jours.
 \item
  Le Conseil d'administration peut, par voie de nomination, pourvoir au poste vacant en suivant des procédures conformes à la présente constitution ; il faut annoncer aux membres une période de mise en candidature qui restera ouverte au moins deux (2) semaines.
\end{enumerate}

\subsection{La rémunération}\label{la-ruxe9munuxe9ration-1}

\begin{enumerate}
 \item
  Les membres de l'exécutif national occupent leur poste sans rémunération et aucun d'entre eux ne doit retirer directement ou indirectement de bénéfice du fait qu'il occupe ce poste, pourvu que les dépenses raisonnables engagées dans l'exercice de ses fonctions soient remboursées.
\end{enumerate}

