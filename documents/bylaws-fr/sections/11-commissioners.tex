\section{Les commissaires}\label{les-commissaires}

\subsection{La nomination et la ratification}\label{la-nomination-et-la-ratification}

\begin{enumerate}
 \item
  Le Conseil d\textquotesingle administration, sur recommandation de l'exécutif national entrant, nommera les commissaires, leur titre et leurs responsabilités, une période de nomination étant communiquée aux membres et restant ouverte pendant un minimum de deux (2) semaines.
\end{enumerate}

\subsection{Le mandat}\label{le-mandat}

\begin{enumerate}
 \item
  Le mandat des commissaires est le même que l\textquotesingle année fiscale de la FCÉG.
\end{enumerate}

\subsection{Le renvoi}\label{le-renvoi-3}

\begin{enumerate}
 \item
  Un commissaire de la Fédération peut être révoqué par résolution du Conseil d\textquotesingle administration adoptée par les deux tiers (2/3) des directeur.trices présent.e.s lors d\textquotesingle une réunion convoquée à cet effet ou par résolution adoptée par les deux tiers (2/3) des membres présents lors d\textquotesingle une réunion de l\textquotesingle assemblée générale convoquée à cet effet.
\end{enumerate}

\subsection{Les postes à pourvoir}\label{les-postes-uxe0-pourvoir-4}

\begin{enumerate}
 \item
  En cas de vacance, celle-ci doit être communiquée aux membres dans un délai de cinq (5) jours.
 \item
  Le Conseil d\textquotesingle administration, sur recommandation de l\textquotesingle exécutif national, peut nommer un/une commissaire par intérim jusqu\textquotesingle à la fin du mandat.
 \item
  Une période de nomination doit être annoncée aux membres et rester ouverte pendant un minimum de deux (2) semaines.
\end{enumerate}

\subsection{La rémunération}\label{la-ruxe9munuxe9ration-2}

\begin{enumerate}
 \item
  Les commissaires serviront sans rémunération et aucun commissaire ne recevra directement ou indirectement de bénéfice de l\textquotesingle occupation du poste, à condition que les dépenses raisonnables engagées dans l\textquotesingle exercice des fonctions du commissaire soient remboursées.
\end{enumerate}

