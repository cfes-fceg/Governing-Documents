\section{Les membres}\label{les-membres}

\subsection{Les cotisations}\label{les-frais-dadhuxe9sion}

\subsubsection{Objectif}\label{les-frais-dadhuxe9sion-objectif}

\begin{enumerate}
 \item
  Le paiement de la cotisation à la FCÉG maintient le statut de membre actif d\textquotesingle une association d'étudiants en ingénierie, ce qui comprend :

  \begin{enumerate}
   \item
    L\textquotesingle éligibilité à la participation aux activités et services de la FCÉG
   \item
    Le droit de présenter des motions lors des réunions de l\textquotesingle assemblée générale
   \item
    Le droit de vote aux réunions de l\textquotesingle assemblée générale
  \end{enumerate}
 \item La cotisation de chaque école est calculée en fonction du nombre d'étudiant.e.s à temps plein inscrit.e.s au premier cycle en génie, en plus de tout.e étudiant.e ayant opté pour l\textquotesingle adhésion à la FCÉG comme spécifié à l\textquotesingle article 6.4, augmentant chaque année par la suite en fonction de l\textquotesingle indice des prix à la consommation (IPC).
 \item Les cotisations seront perçues au cours de l'année scolaire en cours, en se basant sur les effectifs de l'année scolaire précédente, afin d'établir le budget pour l'année à venir.
 \item Le processus de perception des cotisations pour chaque école doit suivre l’échéancier de perception décrit à la section 8.1.2.
\end{enumerate}

\subsubsection{Échéancier de perception}\label{luxe9chuxe9ancier-de-perception}
\begin{enumerate}
 \item La collecte des effectifs doit être lancée au plus tard le 1er octobre.
 \item Les effectifs doivent être communiqués au/à la VPFA par chaque école membre au plus tard le 1er novembre.
 \item Les cotisations seront déterminées lors de l'assemblée générale du CCLI.
 \item Les factures relatives aux cotisations doivent être envoyées aux membres au plus tard le 31 janvier.
 \item Les cotisations sont dues 30 jours après l'émission de la facture.
 \item Lors de la dernière réunion du conseil d'administration précédant le SCIP, les membres qui n'ont pas payé leurs cotisations peuvent être considérés comme des « membres réguliers pas en règle » au cas par cas.

\end{enumerate}

\subsubsection{Cotisations impayées}\label{les-frais-dadhuxe9sion-impayuxe9s}
\begin{enumerate}
\item L'école doit n'aura pas le droit de vote jusqu'à ce que les frais d'adhésion soient payés.
\item Une association étudiante en génie qui n'a pas participé au SDAI ou au CCLI, et qui n'a pas payé sa cotisation à temps doit être contactée par l'ambassadeur.drice régional.e concerné.e afin d'évaluer son avenir en tant que membre de la FCEG.
\item Les effectifs des membres doivent être communiqués à la FCÉG avant le début de la compétition régionale d'ingénierie pour que les membres puissent participer à la CCI.
\end{enumerate}

\subsection{Les types de membres}\label{les-types-de-membres}

\subsubsection{Les membres réguliers en règle}\label{les-membres-ruxe9guliers-en-ruxe8gle}

\begin{enumerate}
 \item
  Un membre régulier en règle de la Fédération est défini comme ayant payé ses frais d\textquotesingle adhésion annuels comme indiqué à la section 6.1.
 \item
  Avoir payé tous les frais de délégation pour les étudiants inscrits à des activités affiliées à la FCÉG, qu\textquotesingle ils participent ou non à l\textquotesingle activité.
 \item
  Seuls les membres réguliers en règle ont le droit de vote à la Fédération.
\end{enumerate}

\subsubsection{Les membres réguliers pas en règle}\label{les-membres-ruxe9guliers-pas-en-ruxe8gle}

\begin{enumerate}
 \item
  Un membre régulier pas en règle n\textquotesingle a pas le droit de vote à la Fédération.
 \item
  Les étudiant.e.s représenté.e.s par des membres non en règle peuvent être admissibles à participer aux activités de la FCÉG ou à utiliser les services de la FCÉG, à condition que le membre ait contacté le Conseil d\textquotesingle administration avec l\textquotesingle intention de payer ses frais d\textquotesingle adhésion, et que l\textquotesingle organe directeur approprié pour l\textquotesingle activité ou le service approuve leur participation.
\end{enumerate}

\subsubsection{Les membres observateurs}\label{les-membres-observateurs}

\begin{enumerate}
 \item
  Un membre observateur n\textquotesingle a pas le droit de vote à la Fédération.
\end{enumerate}

\subsection{Le statut des membres}\label{le-statut-des-membres}

\subsubsection{La suspension des membres}\label{la-suspension-des-membres}

\begin{enumerate}
 \item
  Le Conseil d\textquotesingle administration peut, par résolution adoptée par les deux tiers (2/3) des directeur.trices présent.e.s lors d\textquotesingle une réunion convoquée à cet effet, suspendre tout membre qui enfreint une résolution ou un règlement de la Fédération ou dont la conduite ou les activités sont jugées préjudiciables au bien-être ou au fonctionnement de la Fédération, qui n\textquotesingle a pas payé sa cotisation pour l\textquotesingle année en cours tel que prévu à la politique 8.1.1. ou qui n\textquotesingle a pas payé sa cotisation de délégué pour une activité.
 \item
  Un membre ainsi suspendu sera réputé être un "membre régulier pas en règle".
 \item
  Dans les deux jours ouvrables suivant la suspension d\textquotesingle un membre, le/la président.e informera tous les membres de la suspension, en fournissant les motifs.
 \item
  Les membres qui sont considérés comme n\textquotesingle étant pas en règle parce qu\textquotesingle ils n\textquotesingle ont pas payé leur cotisation ou les frais de délégation pour une activité de la FCÉG sont réintégrés dès qu\textquotesingle ils ont payé l\textquotesingle intégralité des frais en suspens.
\end{enumerate}

\subsubsection{L'expulsion des membres}\label{lexpulsion-des-membres}

\begin{enumerate}
 \item
  Une assemblée générale peut, par résolution adoptée par les deux tiers (2/3) des membres présents lors d\textquotesingle une assemblée des membres, exclure tout membre qui enfreint la constitution de la FCÉG, ou dont la conduite ou les activités sont jugées préjudiciables au bien-être ou au fonctionnement de la FCÉG.
 \item
  Les membres observateurs seront automatiquement exclus s\textquotesingle ils ne demandent pas à devenir membres réguliers conformément à la constitution, à demander une prolongation unique selon l\textquotesingle article 6.3.4, ou à retirer leur adhésion selon l\textquotesingle article 6.3.5 dans les 30 jours suivant la fin de leur statut d\textquotesingle observateur.
\end{enumerate}

\subsubsection{La demande pour le statut de membre observateur}\label{la-demande-pour-le-statut-de-membre-observateur}

\begin{enumerate}
 \item
  Pour prolonger le statut de membre observateur, une organisation doit soumettre une demande écrite au/à la VPFA, qui doit informer immédiatement le Conseil d\textquotesingle administration d\textquotesingle une telle demande.
 \item
  Le Conseil d\textquotesingle administration doit alors prendre une décision concernant l\textquotesingle admission de l\textquotesingle organisation, et ce au plus tard lors de la réunion du Conseil d\textquotesingle administration suivant la soumission de la demande de l\textquotesingle organisation.
 \item
  Après une décision du Conseil d\textquotesingle administration, les membres seront immédiatement informés.
 \item
  Une société ne peut demander le statut de membre observateur que si, à un moment donné au cours des 12 derniers mois, elle n\textquotesingle a été ni membre régulier ni membre observateur.
\end{enumerate}

\subsubsection{La demande de prolongation du statut de membre observateur}\label{la-demande-de-prolongation-du-statut-de-membre-observateur}

\begin{enumerate}
 \item
  Un observateur peut demander une prolongation unique à la fin de sa première année en tant qu\textquotesingle observateur.
 \item
  Pour prolonger la période d'observation, une organisation doit soumettre une demande écrite au/à la VPFA, qui doit immédiatement informer le Conseil d\textquotesingle administration d\textquotesingle une telle demande.
 \item
  La demande doit inclure ce qui suit :

  \begin{enumerate}
   \item
    Une lettre du/de la président.e de la société expliquant les raisons de la décision de prolonger le statut d\textquotesingle observateur et évaluant les progrès de la FCÉG sur les points à améliorer mis en avant dans la lettre initiale de demande de statut de membre observateur
   \item
    Preuve de soutien à la prolongation de la part de la société d\textquotesingle ingénieurs par le biais d\textquotesingle une motion adoptée dans les procès-verbaux d\textquotesingle une réunion publique.
   \item
    Il est recommandé qu\textquotesingle un membre de l\textquotesingle exécutif national de la FCÉG ou du Conseil d\textquotesingle administration soit invité à cette réunion pour représenter la FCÉG.
  \end{enumerate}
 \item
  Une prolongation du statut d\textquotesingle observateur est valable un an après l\textquotesingle approbation d\textquotesingle une demande de prolongation, après quoi la société en question doit choisir de devenir membre conformément à l\textquotesingle article 4.3 ou de retirer son adhésion conformément à l\textquotesingle article 6.3.5.
\end{enumerate}

\subsubsection{La procédure de retrait du statut de membre}\label{la-procuxe9dure-de-retrait-du-statut-de-membre}

\begin{enumerate}
 \item
  Tout membre peut se retirer de la Fédération en remettant une démission écrite au/à la VPFA.
 \item
  Le retrait prend effet le premier jour du mois suivant la réception de l\textquotesingle avis.
\end{enumerate}

\subsubsection{La période de grâce}\label{la-puxe9riode-de-gruxe2ce}

\begin{enumerate}
 \item
  Le membre retiré peut immédiatement revenir au statut de membre qu\textquotesingle il détenait avant le retrait sans avoir besoin de passer par le processus formel de demande d\textquotesingle adhésion, à condition que d\textquotesingle ici le Sommet du développement des associations en ingénierie de l\textquotesingle année fiscale suivant le retrait, le membre :

  \begin{enumerate}
   \item
    Informe le Conseil d\textquotesingle administration avec une preuve écrite de soutien de la part de la société d\textquotesingle ingénieurs (une motion adoptée dans les procès-verbaux d\textquotesingle une réunion publique).
   \item
    Paye les cotisations pour être à jour des frais
   \item
    Avant le retrait était un membre régulier en règle ou un membre régulier pas en règle.
   \item
    N\textquotesingle a pas bénéficié d\textquotesingle une période de grâce au cours des 12 derniers mois.
  \end{enumerate}
\end{enumerate}

\subsubsection{L'appel d'une suspension ou d'une expulsion}\label{lappel-dune-suspension-ou-dune-expulsion}

\begin{enumerate}
 \item
  Un membre peut demander le remboursement de ses cotisations avant le 1er juin de cette année fiscale.
 \item
  Si un remboursement est accordé, le membre est suspendu conformément à l\textquotesingle article 6.3.1 jusqu\textquotesingle à ce qu\textquotesingle il paie sa cotisation complète pour l\textquotesingle année financière.
 \item
  Le Conseil d\textquotesingle administration peut, par résolution adoptée par les deux tiers (2/3) des administrateurs présents lors d\textquotesingle une réunion convoquée à cet effet, accorder un remboursement intégral des frais.
 \item
  Le Conseil d\textquotesingle administration peut demander des documents, y compris des documents financiers, au membre afin d\textquotesingle évaluer correctement le besoin d\textquotesingle un remboursement.
\end{enumerate}

\subsection{Inscription des membres étudiant.e.s}\label{inscription-des-membres-uxe9tudiant.e.s}

\begin{enumerate}
 \item
  Les sociétés membres peuvent décider d\textquotesingle opter pour tous les étudiant.e.s qu\textquotesingle elles représentent et qui ne sont pas inscrits dans un programme de génie accrédité ou inscrits dans un programme en cours d\textquotesingle accréditation dans l\textquotesingle adhésion à la FCÉG.
 \item
  Les sociétés qui souhaitent adhérer d\textquotesingle autres étudiants.e.s à la FCÉG doivent soumettre cette demande au/à la VPFA avant le 1er juillet de l\textquotesingle année précédant la date d\textquotesingle échéance de la cotisation d\textquotesingle adhésion.
 \item
  La responsabilité de la FCÉG envers les membres optés de cette manière se limite à fournir l\textquotesingle accès aux activités décrites à l'article 12 de la constitution.
 \item
  Si une demande n\textquotesingle est pas soumise avant cette date, les étudiants.es qui ne sont ni inscrits dans un programme de génie accrédité ni inscrits dans un programme en cours d\textquotesingle accréditation ne seront pas autorisés à participer aux activités de la FCÉG, à bénéficier des services qu\textquotesingle elle offre ou à être éligibles aux postes de l\textquotesingle exécutif national et des commissaires à moins qu\textquotesingle une exception ne soit accordée par le Conseil d\textquotesingle administration.
\end{enumerate}

