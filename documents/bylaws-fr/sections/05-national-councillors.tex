\section{Les conseiller.ères nationaux.ales}\label{les-conseiller.uxe8res-nationaux.ales}

\subsection{L'objectif}\label{lobjectif-1}

\begin{enumerate}
 \item
  La Fédération élit annuellement deux (2) conseiller.ères nationaux.ales pour siéger en tant que membres du Conseil d\textquotesingle administration et soutenir sa mission.
\end{enumerate}

\subsection{Les responsabilités}\label{les-responsabilituxe9s-1}

\begin{enumerate}
 \item
  En plus de celles énoncées dans la constitution, les conseiller.ères nationaux.ales ont les responsabilités suivantes :

  \begin{enumerate}
   \item
    Conseiller la direction nationale entre les réunions de l\textquotesingle assemblée générale et les réunions du Conseil d\textquotesingle administration
   \item
    Soutenir l'exécutif national, les commissaires, les ambassadeurs.drices régionaux.ales et les membres avec des connaissances institutionnelles
   \item
    Aider les ambassadeurs.drices régionaux.ales à impliquer les membres lorsque cela est nécessaire
  \end{enumerate}
 \item
  Au début de leur mandat en tant que conseillers ou conseillères nationaux, les deux (2) personnes occupant le poste établissent, en collaboration avec l'exécutif national, des domaines de responsabilité relatifs pour maximiser le bénéfice de leurs postes pour la Fédération et éviter la duplication du travail.
 \item
  Les conseiller.ères nationaux.ales sont responsables de fournir une expérience diversifiée au Conseil d\textquotesingle administration, à la Fédération et aux membres, et de conseiller là où cela est souhaité.
\end{enumerate}

\subsection{Le renvoi}\label{le-renvoi-2}

\begin{enumerate}
 \item
  Les conseiller.ères peuvent être révoqué.e.s par résolution du Conseil d\textquotesingle administration adoptée par les deux tiers (2/3) des administrateurs présents à une réunion convoquée à cet effet ou par une résolution adoptée par les deux tiers (2/3) des membres présents à une réunion de l\textquotesingle assemblée générale convoquée à cet effet.
\end{enumerate}

\subsection{Les postes à pourvoir}\label{les-postes-uxe0-pourvoir-3}

\begin{enumerate}
 \item
  Si un poste devient vacant, cette vacance doit être communiquée aux membres dans les cinq (5) jours.
 \item
  Le Conseil d\textquotesingle administration peut nommer un/une conseiller.ère par intérim jusqu\textquotesingle à la fin du mandat.
 \item
  Une période de nomination doit être annoncée aux membres et rester ouverte pendant au moins deux (2) semaines.
\end{enumerate}

