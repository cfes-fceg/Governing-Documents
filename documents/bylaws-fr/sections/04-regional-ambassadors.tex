\section{Les ambassadeurs.drices régionaux.ales}\label{les-ambassadeurs.drices-ruxe9gionaux.ales}

\subsection{L'objectif}\label{lobjectif}

\begin{enumerate}
 \item
  Les ambassadeurs.drices régionaux.ales sont choisi.es par leur région respective pour se faire représenter au Conseil d'administration.
 \item
  On compte un/une ambassadeur.drice par région; il/elle a des tâches à accomplir tant à l'échelle nationale que régionale.
\end{enumerate}

\subsection{Les responsabilités}\label{les-responsabilituxe9s}

\begin{enumerate}
 \item
  Les ambassadeurs.drices régionaux.ales sont chargé.e.s de présenter le point de vue de leur région respective auprès de la Fédération, ainsi que de faciliter la communication entre leur association régionale respective et la FCEG et de remplir tout autre mandat qui leur est confié par l'assemblée générale ou le Conseil d'administration.
\end{enumerate}

\subsection{Les accords régionaux}\label{les-accords-ruxe9gionaux}

\begin{enumerate}
 \item
  La FCÉG doit maintenir en vigueur des contrats avec ses entités juridiques régionales comme le définit la constitution.
 \item
  Ces contrats sont connus sous le nom d'ententes régionales.
 \item
  Celles-ci devraient faire l'objet d'une révision chaque année, laquelle devrait porter tout au moins sur les points suivants :

  \begin{enumerate}
   \item
    Responsabilités des ambassadeurs.drices régionaux.ales : l'accord régional doit stipuler que ce sont les documents constitutifs de l'organisme régional compétent et de la FCÉG qui établissent les fonctions de l'ambassadeur.drice de la région visée; l'entente régionale elle-même n'aborde pas du tout le sujet des fonctions
   \item
    Élection des ambassadeurs.drices régionaux.ales : l'accord régionale doit mettre en place les procédures d'élection de l'ambassadeur.drice de la région visée, y compris l'échéance pour la ratification par l'assemblée générale ou le Conseil d'administration de la FCÉG
   \item
    Compétitions régionales d'ingénierie: l'accord régional doit exiger du/de la président.e du comité organisateur de la région concernée qu'il participe à titre consultatif à la Compétition canadienne d'ingénierie (CCI).
  \end{enumerate}
\end{enumerate}

\subsection{Les tâches et responsabilités à l'égard de la FCÉG}\label{les-tuxe2ches-et-responsabilituxe9s-uxe0-luxe9gard-de-la-fcuxe9g}

\begin{enumerate}
 \item
  Voici quelques-unes des tâches et des responsabilités des ambassadeurs.drices régionaux.ales:

  \begin{enumerate}
   \item
    Entretenir des voies de communication transparentes entre la FCÉG et les membres de leur région; faire valoir le point de vue de leur région dans le cadre des réunions du Conseil d'administration
   \item
    Faciliter le recours aux votes par procuration au besoin
   \item
    Encourager les membres de leur région à participer aux événements de la FCÉG, sinon à les organiser
   \item
    Encourager les étudiant.e.s à poser leur candidature à un poste au sein de la FCÉG
   \item
    Aider le/la VPFA à percevoir les frais d'adhésion
   \item
    Aider le/la commissaire à l'administration à fournir au VPS les coordonnées à jour des membres de leur région
   \item
    Présenter les dirigeants de la FCÉG et mener les séances de mise à jour à l'intention des membres de leur région
   \item
    S'assurer que la FCÉG est représentée dans le cadre des activités régionales importantes, comme les AGAs, les compétitions régionales et les réunions régionales
   \item
    Diriger les réunions régionales dans le cadre du CCLI
   \item
    Rencontrer les membres des ordres constituants avant les assemblées générales afin de les tenir au courant des progrès de la Fédération dans l'exécution de ses mandats et d'informer les délégués aux attentes de l'activité. Voici les échéances mensuelles:

    \begin{enumerate}
     \item
      Août ou septembre pour se préparer au Sommet du développement des associations en ingénierie.
     \item
      Novembre pour se préparer au CCLI
    \end{enumerate}
   \item
    Rencontrer les membres des ordres constituants après les assemblées générales afin de les d'évaluer les conclusions de l'activité. Voici les échéances mensuelles :

    \begin{enumerate}
     \item
      Septembre ou octobre pour réviser le Sommet du développement des associations en ingénierie
     \item
      Janvier pour réviser le CCLI
    \end{enumerate}
   \item
    Rédiger un document de transition sommaire au plus tard le 30 avril ; ce document sera conservé dans les archives électroniques de la FCÉG
   \item
    Assurer la liaison entre leur région et les organismes de la FCÉG, comme le Groupe de travail du plaidoyer, pour aider à désigner des représentants
   \item
    La communication des mises à jour écrites sur les tâches réalisées, à une fréquence déterminée par l'exécutif national --- au moins tous les deux mois
   \item
    Exercer d'autres fonctions indiquées dans les documents constitutifs de l'organisme régional
  \end{enumerate}
\end{enumerate}

\subsection{Les ambassadeurs.drices régionaux.ales}\label{les-ambassadeurs.drices-ruxe9gionaux.ales-1}

\begin{enumerate}
 \item
  Les ambassadeurs.drices régionaux.ales occupent les quatre postes suivants, chacun étant choisi par son organisme régional respectif ; c'est le CCLI qui ratifie les choix :

  \begin{enumerate}
   \item
    L'ambassadeur.drice de l'ouest
   \item
    L'ambassadeur.drice de l'Ontario
   \item
    L'ambassadeur.drice du Québec
   \item
    L'ambassadeur.drice de l'atlantique
  \end{enumerate}
\end{enumerate}

\subsection{Les postes à pourvoir}\label{les-postes-uxe0-pourvoir-2}

\begin{enumerate}
 \item
  Si un poste devient vacant, cette vacance doit être communiquée aux membres de la région concernée et à l'organisme régional compétent dans un délai de cinq (5) jours.
 \item
  On doit pourvoir au poste d'ambassadeur/ambassadrice vacant.e en demandant aux membres de la région concernée de voter par la poste.
 \item
  C'est l'organisme régional compétent qui doit faciliter ce processus, et l'élection sera ratifiée par le Conseil d'administration.
 \item
  Il faut annoncer aux membres concernés une période de mise en candidature qui restera ouverte au moins deux (2) semaines, sauf avis contraire de l'organisme régional.
\end{enumerate}

\subsection{Le renvoi}\label{le-renvoi-1}

\begin{enumerate}
 \item
  Il est possible de destituer un.e ambassadeur.drice par résolution spéciale des membres de la région concernée.
 \item
  C'est l'organisme régional compétent qui doit faciliter ce processus.
\end{enumerate}

