\section{Article 2 - Dispositions}
\label{article-2---dispositions}

\subsection{Guiding Principles}
\label{guiding-principles}

\begin{enumerate}
 \item The Federation supports growth and communication within its members to ensure their moral, intellectual, cultural, academic, social and economic well-being.
 \item The Federation represents and promotes Canadian undergraduate engineering students on a national and international level.
 \item The Federation will not align itself with any political party.
\end{enumerate}

\subsection{Mission}
\label{mission}

\begin{enumerate}
 \item The goal of the Federation is to solicit, represent, organize and exchange views, information and activities pertinent to the goals of the Members at a national level as to ensure their moral, intellectual, cultural, academic, social and economic well-being, and to improve the quality and scope of Canadian engineering education.
 \item The means by which this will be done are:
  \begin{enumerate}
   \item To organize and host conferences, meetings, assemblies, exhibitions and competitions related to this end,
   \item To affiliate itself with any organization pursuing the same goals as those of the said corporation,
   \item To acquire via purchase, renting or otherwise, to possess and operate the assets, movables and real estate necessary for the accomplishment of the above mentioned goals,
   \item To provide to its Members services of all kinds related to the above mentioned goals,
   \item To create, support, participate in movements, opinion campaigns or other organizations relating to these goals whenever possible,
   \item To encourage Members to promote the creation and exchange of pertinent information to all Members and other interested parties,
   \item To promote the interaction of Canadian engineering students with specific interest groups (societies, associations, government and other agencies) on national and international issues of social, economic, political, legal and human concern relevant to engineering,
   \item To present information of relevance to appropriate government and other agencies,
   \item To cooperate with all engineering associations recognized by the corporation in matters of common interest,
   \item To promote the public image of engineering students in Canada,
   \item To facilitate and develop activities and services, including but not limited to the:
    \begin{enumerate}
     \item Canadian Engineering Leadership Conference
     \item Canadian Engineering Competition
     \item Conference on Diversity in Engineering
     \item Conference on Sustainability in Engineering
     \item Summit on Development of Engineering Societies
    \end{enumerate}
  \end{enumerate}
\end{enumerate}

\subsection{Regions}
\label{regions}

\begin{enumerate}
 \item The Federation comprises of 4 regions. They are:
  \begin{enumerate}
   \item Atlantic -- Provinces of Prince Edward Island, of New Brunswick, of Nova Scotia, and of Newfoundland and Labrador;
   \item Quebec -- Province of Quebec;
   \item Ontario -- Province of Ontario; and
   \item West -- Provinces of British Columbia, of Alberta, of Saskatchewan, of Manitoba, and The Territory of Yukon, Territory of Nunavut and North-West Territories.
  \end{enumerate}
\end{enumerate}

\subsection{Official Languages}
\label{official-languages}

\begin{enumerate}
 \item The Official Languages of the Federation shall be English and French.
 \item All communications to the Members shall be done in either official language, or both, at the member's discretion.
 \item Requests for translation from one Official Language to the other may be made by any Member.
 \item Furthermore, all publications of the Federation shall be available simultaneously in both Official Languages.
 \item In the case of a divergence between the English and French versions of this Constitution, the English version takes precedence.
 \item In the case of a divergence between the English and French versions of the rules and regulations or resolutions, the original document takes precedence over the translation as official document of the Federation.
\end{enumerate}

\subsection{Invalidation}
\label{invalidation}

\begin{enumerate}
 \item Invalid or illegal dispositions of this Constitution or any other rules or regulations that the Federation should henceforth adopt will not invalidate the whole of this constitution, those rules or those regulations and they shall stay valid as if those invalid or illegal dispositions had never been included.
\end{enumerate}

\subsection{Titles}
\label{titles}

\begin{enumerate}
 \item Titles inserted in this Constitution and in all other documents of the Federation hereafter passed are included only to facilitate reading and should not affect the interpretation of the Constitution and these documents.
\end{enumerate}

\subsection{Interpretation}
\label{interpretation}

\begin{enumerate}
 \item In this Constitution and in all other documents of the Federation hereafter passed, unless the context otherwise requires, words importing the singular number shall include the plural number, and vice versa, and references to persons shall include firms and corporations.
\end{enumerate}

\subsection{Bylaws}
\label{bylaws}

\begin{enumerate}
 \item Further to this Constitution and the regulations contained herein, additional policies and best practices can be found in the Bylaws of the Federation, hereafter referred to as the “Bylaws”.
\end{enumerate}

\subsection{Policy Manual}
\label{policy-manual}

\begin{enumerate}
 \item Further to this constitution and the regulations contained herein, additional policies and best practices can be found in the Policy Manual of the Federation, hereafter referred to as the "Policy Manual".
\end{enumerate}

\subsection{Document of Stances}
\label{document-of-stances}

\begin{enumerate}
 \item Official positions taken by the Federation can be found in the Document of Stances as outlined in the Policy Manual.
\end{enumerate}
