\section{Le traitement et la sécurité de l'information}\label{le-traitement-et-la-suxe9curituxe9-de-linformation}

\begin{enumerate}
 \item
  La FCÉG recueille des renseignements personnels au besoin; ces renseignements doivent rester confidentiels et sont protégés.
 \item
  Au sens de la Loi sur l'accès à l'information et la protection de la vie privée, renseignement personnel s'entend comme:

  \begin{enumerate}
   \item
    Des renseignements concernant la race, l'origine nationale ou ethnique, la couleur, la religion, l'âge, le sexe, l'orientation sexuelle, l'état matrimonial ou familial de celui-ci ou celle-ci ;
   \item
    Des renseignements concernant l'éducation, les antécédents médicaux, psychiatriques, psychologiques, criminels ou professionnels de ce particulier ou des renseignements liés à sa participation à une opération financière ;
   \item
    D'un numéro d'identification, d'un symbole ou d'un autre signe individuel qui lui est attribué ;
   \item
    De l'adresse, du numéro de téléphone, des empreintes digitales ou du groupe sanguin de ce/tte particulier.ère ;
   \item
    De ses opinions ou de ses points de vue personnels, sauf s'ils se rapportent à un.e autre particulier.ère ;
   \item
    De la correspondance ayant explicitement ou implicitement un caractère personnel et confidentiel, adressée par le particulier à une institution, ainsi que des réponses à cette correspondance originale susceptibles d'en révéler le contenu ;
   \item
    Des opinions et des points de vue d'une autre personne au sujet de ce/cette particulier.ère ;
   \item
    Du nom du/de la particulier.ère, s'il figure parmi d'autres renseignements personnels qui le concernent, ou si sa divulgation risque de révéler d'autres renseignements personnels au sujet du particulier.
  \end{enumerate}
 \item
  La FCÉG définit l'expression renseignement personnel de la même façon.
 \item
  Aucun officier de la FCÉG ne doit demander de renseignements personnels, exception faite de :

  \begin{enumerate}
   \item
    Le nom et le prénom
   \item
    L'adresse courriel
   \item
    Le numéro de téléphone
   \item
    Le statut par rapport aux études, y compris le programme, l'année d'obtention du diplôme, région de l\textquotesingle obtention d\textquotesingle un diplôme d\textquotesingle études secondaires, le nom de l'établissement d'enseignement, le statut d'étudiant étranger (le cas échéant) et les cours suivis
   \item
    La date de naissance
   \item
    Les pronoms personnels
   \item
    La langue de préférence
   \item
    Les besoins particuliers
   \item
    Le genre
   \item
    L'âge
   \item
    Le point de vue sur l'état des programmes de formation en génie dans l'établissement d'enseignement
   \item
    La révélation des expériences personnelles liées aux études
   \item
    Ethnicité
   \item
    Nationalité
  \end{enumerate}
 \item
  Si des renseignements supplémentaires se révèlent nécessaires dans un cas précis, le/la président.e peut en autoriser la collecte par une autorisation spéciale.
 \item
  Dans tous les cas, le but de la collecte de renseignements personnels et l'identité des personnes qui pourront y accéder (à l'exception des administrateurs du système de stockage infonuagique) doivent être explicitement déclarés dans le formulaire de collecte sinon au moment de la collecte.
 \item
  Les documents renfermant des renseignements personnels sont considérés comme des documents classifiés ; le stockage et le transfert de renseignements personnels sont assujettis à la Politique sur la sécurité des technologies de l'information.
\end{enumerate}

