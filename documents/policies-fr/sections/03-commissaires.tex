\section{Les commissaires}\label{les-commissaires}

\subsection{Le portrait global des postes de commissaire}\label{le-portrait-global-des-postes-de-commissaire}

\begin{enumerate}
 \item
  Les commissaires sont nommés, au besoin, par l\textquotesingle exécutif national de la FCÉG pour aider à remplir les fonctions et les objectifs de l'exécutif national.
 \item
  La liste des postes de commissaires potentiels ainsi que leurs fonctions se trouve dans le document de référence « {[}FR{]} Document de référence - Les commissaires »
 \item
  Les positions suivantes devraient être occupées, mais il est également possible que l'exécutif national crée de nouveaux postes de commissaires ou laisse certains postes vacants. Ces décisions doivent être approuvées par le Conseil d\textquotesingle administration.
 \item
  Dans le cas où un poste devient vacant, le membre de l'exécutif national responsable pour cette position assumera les tâches et les responsabilités nécessaires.
 \item
  Les postes de commissaires actuels sont les suivants:

  \begin{enumerate}
   \item
    Le/la commissaire à la gouvernance
   \item
    Le/la commissaire aux relations d'entreprise
   \item
    Le/la commissaire à l'informatique
   \item
    Le/la commissaire des relations internationales
   \item
    Le/la commissaire au leadership
   \item
    Le/la commissaire des médias et du marketing
   \item
    Le/la commissaire au bilinguisme
   \item
    Le/la commissaire aux données
   \item
    Le/la commissaire à la logistique
  \end{enumerate}

 \item
  Les responsabilités générales pour les commissaires se résument comme suit:

  \begin{enumerate}
   \item
    Compléter les tâches qui leur sont assignées par l'exécutif national;
   \item
    Contribuer aux communications de la FCÉG;
   \item
    Agir comme consultant.e ou conseiller.ère aux membres de l'exécutif national;
   \item
    Rédiger un document sommaire pour la transition au plus tard le 30 avril, pour être stocké dans les archives électroniques de la FCÉG;
   \item
    Fournir des mises à jour écrites sur les tâches réalisées, à une fréquence déterminée par l'exécutif national qui est, au minimum, à tous les deux (2) mois.
  \end{enumerate}
\end{enumerate}

\subsection{L'admissibilité}\label{ladmissibilituxe9}

\begin{enumerate}
 \item
  Pour être admissibles à un poste de commissaire, les candidat.e.s doivent être membres d'une association membre active au moment de la mise en candidature, tout en satisfaisant aux critères supplémentaires décrits par l'exécutif national pour la fonction du/de la commissaire concerné.e.
 \item
  Des exigences peuvent changer d'une année à l'autre, notamment celle de la présence effective à l'assemblée générale annuelle.
 \item
  Une personne ne peut pas détenir plus d'un poste parmi l'équipe de la FCÉG à un moment donné.
\end{enumerate}

\subsection{Les candidatures}\label{les-candidatures}

\begin{enumerate}
 \item
  Les personnes qui souhaitent occuper un poste de commissaire doivent déclarer leur intention en remplissant une demande et en la soumettant à l'exécutif national.
 \item
  Ce formulaire de demande sera mis à la disposition de tous les candidat.e.s admissible.s, et c'est l'exécutif national qui déterminera la date limite de dépôt.
\end{enumerate}

\subsection{La sélection}\label{la-suxe9lection}

\begin{enumerate}
 \item
  L'exécutif national entrant déterminera une méthode de sélection qui comprend minimalement une étape d'appel de mises en candidature.
 \item
  Les candidatures retenues seront soumises à la ratification du Conseil d\textquotesingle administration.
\end{enumerate}

\subsection{Le/la commissaire aux relations d'entreprise}\label{lela-commissaire-aux-relations-dentreprise}

\begin{enumerate}
 \item
  Le/la commissaire aux relations d'entreprise est responsable des activités de financement et de partenariat centralisées de la FCÉG.
 \item
  Ce commissaire relève directement du/de la vice-président.e externe.
 \item
  Les tâches et responsabilités du/de la commissaire aux relations d'entreprise sont les suivantes:

  \begin{enumerate}
   \item
    Assurer le respect et le maintien des ententes de partenariat qui encadrent les relations de la FCÉG
   \item
    Appuyer les responsables des activités et leur VP partenariats dans le cadre de leur initiatives respectives en matière de commandites.
   \item
    Chercher à établir d'autres partenariats pour la FCÉG ou à saisir d'autres possibilités de financement à soumettre à l'approbation du Conseil d\textquotesingle administration et des responsables des activités
   \item
    Dresser la liste de tous les partenariats de financement actifs, et la présenter aux membres dans le cadre de l'assemblée générale annuelle
   \item
    Garder et mettre à jour la base de données sur les commanditaires de la FCÉG, laquelle comprend les détails de la dernière activité, les coordonnées des commanditaires et les trousses de commandite, puis mettre ces informations à la disposition des responsables d'activités.
  \end{enumerate}
\end{enumerate}

\subsection{Le commissaire à la gouvernance}\label{le-commissaire-uxe0-la-gouvernance}

\begin{enumerate}
 \item
  Le/la commissaire à la gouvernance est responsable de maintenir les dossiers généraux de la FCÉG.
 \item
  Ce commissaire relève directement du/de la VPFA.
 \item
  Les tâches et les responsabilités du/de la commissaire à la gouvernance sont les suivantes:

  \begin{enumerate}
   \item
    Consigner les activités de contrôle des états financiers et des dossiers conformément aux lois applicables à la FCÉG.
   \item
    S'assurer que la liste des membres dressée à chaque année par le/la VPFA est à jour.
   \item
    Rédiger les procès-verbaux des assemblées générales de la CFES et désigner des rédacteurs de procès-verbaux pour les assemblées générales de la CFES auxquelles ils ne sont pas présents.
   \item
    En même temps que les procès-verbaux, distribuer aux membres un document de suivi qui décrit le déroulement des assemblées dans ses grandes lignes, y compris le titre, le numéro et l'essentiel de chaque requête, ainsi que le résultat du vote, en précisant si une requête a été remise à plus tard.
   \item
    Mettre à jour la fiche récapitulative des acronymes du CFES et veiller à ce qu\textquotesingle elle soit accessible à tous les membres.
   \item
    Rédaction des procès-verbaux des réunions du conseil d\textquotesingle administration.
   \item
    Veiller à ce que les versions les plus récentes de tous les documents juridiques et financiers soient communiquées au conseil d\textquotesingle administration.
  \end{enumerate}
\end{enumerate}

\subsection{Le/la commissaire à l'informatique}\label{lela-commissaire-uxe0-linformatique}

\begin{enumerate}
 \item
  Le/la commissaire à l'informatique entretient et administre les infrastructures numériques de la FCÉG sous tous leurs aspects.
 \item
  Ce commissaire relève directement du/de la VPFA.
 \item
  Les tâches et les responsabilités du/de la commissaire à l'informatique sont les suivantes:

  \begin{enumerate}
   \item
    Assurer l'accès au site web de la FCÉG, aux comptes de courriel, aux listes d'envois électroniques et aux archives électroniques
   \item
    Assister à la création des sites web pour les activités, les services et les programmes
   \item
    Assurer la sécurité des infrastructures numériques de la FCÉG
   \item
    Entretenir le système de stockage de fichiers de la FCÉG
   \item
    Assurer la disponibilité du site web de la FCÉG
   \item
    Offrir aux membres de la FCÉG de l'information sur les bonnes pratiques de TI.
   \item
    Activités de soutien avec des comptes de messagerie électronique pour le comité organisateur et des connexions de sous-domaines
  \end{enumerate}
\end{enumerate}

\subsection{Le/la commissaire au bilinguisme}\label{lela-commissaire-au-bilinguisme}

\begin{enumerate}
 \item
  Le/la commissaire du bilinguisme assure le caractère bilingue des opérations de la FCÉG.
 \item
  Le/la commissaire du bilinguisme relève directement du/de la VPC.
 \item
  Les tâches et les responsabilités du/de la commissaire du bilinguisme sont les suivantes:

  \begin{enumerate}
   \item
    Coordonner les services de traduction de tous les textes destinés au Conseil d\textquotesingle administration, à l'exécutif national et aux commissaires, y compris l'organisation du travail de traducteurs·trices que la FCÉG pourra engager
   \item
    Veiller à ce que les traductions des documents bilingues soient mises à jour dans les 60 jours suivant toute modification.
   \item
    Occuper la fonction de point de contact pour les membres qui ont des soucis langagiers
   \item
    Mettre en œuvre les activités et les initiatives visant à favoriser l'entente mutuelle entre les membres
   \item
    Favoriser le bilinguisme au sein de la FCÉG en faisant mieux connaître les deux langues officielles et les questions linguistiques pertinentes par l'entremise d'un groupe de travail sur le bilinguisme
   \item
    Collaborer avec les responsables d\textquotesingle activités, des services et des programmes pour gérer les services de traduction, au besoin.
  \end{enumerate}
\end{enumerate}

\subsection{Le/la commissaire des médias et du marketing}\label{lela-commissaire-des-muxe9dias-et-du-marketing}

\begin{enumerate}
 \item
  Le/la commissaire des médias et du marketing s'occupe de l'image de la FCÉG, en plus de soutenir les officiers dans les activités de mise en page et de conception.
 \item
  Le/la commissaire des médias et du marketing relève directement du/de la VPC.
 \item
  Les tâches et les responsabilités du/de la commissaire des médias et du marketing sont les suivantes:

  \begin{enumerate}
   \item
    Tenir à jour des archives de fichiers originaux relatifs au logo, aux bannières, aux articles de marque et aux modèles
   \item
    Entretenir l'architecture du site web de la FCÉG
   \item
    Maintenance et mise à jour du contenu du site web de la CFES, ce qui implique notamment de veiller à ce que les dernières informations disponibles sur les événements organisés par la CFES soient facilement accessibles sur le site web, telles que la date de l\textquotesingle événement et l\textquotesingle école hôte.
   \item
    Ces informations doivent être mises à jour au plus tard trois semaines après l\textquotesingle assemblée générale au cours de laquelle un responsable d\textquotesingle activité a été ratifié.
   \item
    Maintenir en vigueur et faire respecter les directives sur le logo et l'image de marque de la FCÉG
   \item
    Aider les responsables d'activités, des programmes et des services à réaliser des modèles de conception destinés aux documents ou au matériel promotionnel, au besoin
   \item
    Demander des soumissions en ce qui a trait aux articles de marque destinés à la FCÉG ou aux responsables d'activités, au besoin
   \item
    Mettre à jour le profil de la FCÉG dans les médias sociaux
   \item
    Obtenir, pour l'équipe des officiers, des cartes professionnelles avec le logo officiel et indiquant le titre de poste ainsi que les coordonnées au plus tard à la réunion d'été
  \end{enumerate}
\end{enumerate}

\subsection{Le/la commissaire au leadership}\label{lela-commissaire-au-leadership}

\begin{enumerate}
 \item
  Le/la commissaire au leadership est chargé.e de rassembler les documents liés au leadership et aux habiletés interpersonnelles, puis de préparer leur distribution aux membres.
 \item
  Le/la titulaire de ce poste relève directement du/de la VPS.
 \item
  Les tâches et les responsabilités du/de la commissaire au leadership sont les suivantes:

  \begin{enumerate}
   \item
    Préparer une série de séances liées au développement du leadership et des habiletés interpersonnelles pour le CCLI de la FCÉG
   \item
    Communiquer avec le Conseil d\textquotesingle administration et l'exécutif national pour avoir la certitude que le programme de développement du leadership est en harmonie avec les objectifs et la mission de la FCÉG
   \item
    Communiquer avec les responsables d\textquotesingle activités de la FCÉG afin d\textquotesingle organiser des sessions de leadership si nécessaire.
   \item
    Communiquer avec le/la responsable d'activité du CCLI pour aménager l'espace et y disposer les documents requis pour animer les séances
   \item
    S'assurer que toutes les personnes sélectionnées pour animer un séance sur le développement du leadership seront disponibles pour diriger ces séances au CCLI de la FCÉG
   \item
    Mettre à la disposition des membres, par voie numérique, les documents du dossier de développement du leadership qui seront présentés dans la cadre du CCLI de la FCÉG
   \item
    Rédiger des documents sur le développement du leadership destinés à l'usage des membres au sein de leur organisme respectif, puis les mettre à leur disposition par voie numérique
  \end{enumerate}
\end{enumerate}

\subsection{Le/la commissaire aux données}\label{lela-commissaire-aux-donnuxe9es}

\begin{enumerate}
 \item
  Le/la commissaire aux données est responsable des efforts de promotion de la FCEG par le biais du développement, de l\textquotesingle exécution et de l\textquotesingle interprétation de l\textquotesingle enquête nationale.
 \item
  Il travaille en étroite collaboration avec le vice-président académique, assure la liaison entre les différents groupes de travail et l\textquotesingle exécutif de la FCEG, et publie des rapports et de la documentation sur le développement et les résultats de l\textquotesingle enquête nationale.
 \item
  Ce poste est placé sous l\textquotesingle autorité du/de la VPA.
 \item
  Les tâches et les responsabilités du commissaire aux données sont les suivantes :

  \begin{enumerate}
   \item
    Maintenir l\textquotesingle enquête nationale annuelle
   \item
    Assurer une collecte uniforme et standardisée des données
   \item
    Distribuer l\textquotesingle enquête nationale et faire de la publicité en collaboration avec le/la VPC.
   \item
    Collecter des données et partager des analyses pertinentes
   \item
    Générer un rapport à partir des données combinées de l\textquotesingle enquête nationale collectées
   \item
    Rechercher, compiler et documenter des processus similaires dans les écoles membres et les organisations.
   \item
    Servir de ressource au Conseil d\textquotesingle administration de la FCÉG et aux écoles membres en ce qui concerne l\textquotesingle uniformité des activités et des opérations dans les autres écoles et organisations membres.
   \item
    Veiller à ce que toutes les données et enquêtes respectent les normes éthiques de collecte et participer à toute formation nécessaire à cet effet.
   \item
    Ensure the data from the national survey is backed up in a secondary location attached to vpa@cfes.ca
  \end{enumerate}
\end{enumerate}

\subsection{Le/la commissaire à la logistique}\label{lela-commissaire-uxe0-la-logistique}

\begin{enumerate}
 \item
  Le/la commissaire à la logistique est chargé d\textquotesingle organiser et de rechercher le matériel lié aux activités dans le cadre du portefeuille de services et de soutenir le/la VPS en cas de besoin, tout en maintenant une ligne de communication ouverte avec les responsables d\textquotesingle activités pour les mises à jour.
 \item
  Les tâches et responsabilités du/de la commissaire à la logistique sont les suivantes :

  \begin{enumerate}
   \item
    Diriger le groupe de travail sur la logistique aux côtés du/de la VPS.
   \item
    Encourager les sociétés membres à se porter candidates et à accueillir des activités de la FCÉG et à les promouvoir.
   \item
    La maintenance et l\textquotesingle entretien d\textquotesingle une archive de dossiers originaux relatifs aux activités et aux appels d\textquotesingle offres.
   \item
    Aider à la gestion des activités en participant aux réunions d\textquotesingle enregistrement et en examinant la documentation du Comité organisateur (CO).
   \item
    Communiquer avec le/la responsable d\textquotesingle activité désigné pour obtenir des informations sur l\textquotesingle organisation.
   \item
    Agir en tant que ressource FCÉG supplémentaire pour le/la responsable d\textquotesingle activité.
   \item
    Élaborer et mettre à disposition sous forme électronique les meilleures pratiques et les documents d\textquotesingle orientation des services.
  \end{enumerate}
\end{enumerate}

