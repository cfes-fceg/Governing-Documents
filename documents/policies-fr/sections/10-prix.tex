\section{Les prix}\label{les-prix}

\subsection{Objectif et survol}\label{objectif-et-survol}

\begin{enumerate}
 \item
  Dans le cadre de chaque CCLI, la FCÉG remet des prix à des personnes, à des groupes et à des membres choisis en reconnaissance de leur contribution à la Fédération ou à la communauté canadienne des ingénieurs.
 \item
  Tous les prix doivent être remis au CCLI, à l\textquotesingle exception du prix mentionné à la section 9.3, qui doit être remis au SCIP.

  \begin{enumerate}
   \item
    La FCÉG encourage l\textquotesingle ambassadeur.drice et/ou le/la VPX des lauréats à organiser une cérémonie en personne après la remise du prix à la SCIP, dans la mesure du possible.
  \end{enumerate}
 \item
  Un comité des prix doit être créé à la discrétion du/de la président.e, pour les prix CCLI et SCIP, et a le pouvoir d\textquotesingle attribuer tous les prix, sauf indication contraire.
\end{enumerate}

\subsection{La remise des prix au CCLI}\label{la-remise-des-prix-au-ccli}

\subsubsection{La vue d\textquotesingle ensemble}\label{la-vue-densemble}

\begin{enumerate}
 \item
  Les prix suivants sont remis chaque année; d'autres prix peuvent être remis en fonction d'une contribution exceptionnelle ou du mérite.
\end{enumerate}

\subsubsection{Le prix d'appréciation}\label{le-prix-dappruxe9ciation}

\begin{enumerate}
 \item
  Le prix d'appréciation de la FCÉG reconnaît la contribution importante d'une personne, autre qu'un.e étudiant.e, à la Fédération.
 \item
  Le prix est remis chaque année à un membre du personnel ou du corps professoral d'un établissement d'enseignement supérieur, sinon à un représentant de l'industrie ou d'un autre organisme.
\end{enumerate}

\subsubsection{Le prix du leadership}\label{le-prix-du-leadership}

\begin{enumerate}
 \item
  Le prix du leadership de la FCÉG reconnaît la contribution importante d'un.e étudiant.e en génie à l'amélioration de l'image des étudiants de premier cycle en génie du Canada.
 \item
  Le prix est décerné chaque année à un.e étudiant.e en génie qui, dans le cadre d'initiatives d'établissements d'enseignement supérieur et communautaires, a fait valoir l'utilité des étudiants en génie et de la profession pour la société.
\end{enumerate}

\subsubsection{Le prix de bienfaisance}\label{le-prix-de-bienfaisance}

\begin{enumerate}
 \item
  Le prix de bienfaisance de la FCÉG reconnaît les contributions importantes à une œuvre.
 \item
  Le prix est remis chaque année au membre qui a apporté une contribution exceptionnelle à une cause caritative au cours de l'année qui vient de se terminer
\end{enumerate}

\subsubsection{Le prix de reconnaissance pour les officiers}\label{le-prix-de-reconnaissance-pour-les-officiers}

\begin{enumerate}
 \item
  C'est l'exécutif national qui choisit le/la lauréat.e du prix de reconnaissance pour les officiers afin de souligner la contribution d'un.e ambassadeur.drice, d'un.e commissaire, du responsable d'un service ou d'un responsable d\textquotesingle activité à l'équipe des officiers.
\end{enumerate}

\subsubsection{\texorpdfstring{Le prix du président du Conseil d\textquotesingle administration }{Le prix du président du Conseil d\textquotesingle administration }}\label{le-prix-du-pruxe9sident-du-conseil-dadministration}

\begin{enumerate}
 \item
  C'est le/la président.e du Conseil d\textquotesingle administration qui choisit le/la lauréat.e du prix, soit le/la représentant.e d'une association membre qui a brillé par son enthousiasme à l'AGA.
\end{enumerate}

\subsubsection{Le prix du plaidoyer de la FCÉG}\label{le-prix-du-plaidoyer-de-la-fcuxe9g}

\begin{enumerate}
 \item
  Ce prix récompense les contributions significatives d\textquotesingle un.e étudiant.e en ingénierie aux problèmes auxquels est confrontée la communauté, que ce soit par l\textquotesingle identification, la recherche ou la promotion de la sensibilisation ou du changement.
 \item
  Le prix du plaidoyer pourrait être décerné pour le soutien apporté à l\textquotesingle une ou l\textquotesingle autre de nos positions.
\end{enumerate}

\subsubsection{Le prix de la responsabilité environnementale}\label{le-prix-de-la-responsabilituxe9-environnementale}

\begin{enumerate}
 \item
  Ce prix récompense un.e étudiant.e en ingénierie qui a fondé une initiative de gestion ou de protection de l\textquotesingle environnement ou qui y a apporté des améliorations significatives.
 \item
  Le lauréat de ce prix devra créer des solutions innovantes qui minimisent l\textquotesingle impact sur l\textquotesingle environnement et promeuvent une gestion responsable des ressources dans le domaine de l\textquotesingle ingénierie, conformément à la position de la FCÉG sur le développement durable.
\end{enumerate}

\subsection{Les prix remis au SCIP}\label{les-prix-remis-au-scip}

\subsubsection{Le prix de reconnaissance pour l'activité exceptionnelle}\label{le-prix-de-reconnaissance-pour-lactivituxe9-exceptionnelle}

\begin{enumerate}
 \item
  Le prix de reconnaissance pour l'activité exceptionnelle reconnaît une activité de la FCÉG qui s'est démarquée, et est décernée au membre ou aux membres du comité organisateur d'une activité qui s'est/se sont démarqué.s de par ses contributions à l'activité.
\end{enumerate}

