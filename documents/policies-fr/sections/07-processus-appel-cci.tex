\section{Processus d'appel de la CCI}\label{processus-dappel-de-la-cci}

\subsection{Appels concernant des changements qui touchent plus d'une CCI}\label{appels-concernant-des-changements-qui-touchent-plus-dune-cci}

\begin{enumerate}
 \item
  Tous les appels entrant dans cette catégorie doivent être interjetés dans les deux semaines suivant la décision du CCCI.
 \item
  Pour entamer la procédure d\textquotesingle appel, une lettre détaillant la décision faisant l\textquotesingle objet de l\textquotesingle appel et les motifs de celui-ci doit être soumise au président du CCCCI, qui doit ensuite en informer le CCCCI et le conseil d\textquotesingle administration de la FCEG.
 \item
  Le conseil d\textquotesingle administration de la FCEG doit répondre à l\textquotesingle appel dans un délai d\textquotesingle une semaine à compter de la notification du président du CCCCI.
 \item
  Ce type d\textquotesingle appel nécessite au moins 10 signataires de la lettre provenant de n\textquotesingle importe quelle région ou au moins 1 signataire de chaque région de la FCEG.

  \begin{enumerate}
   \item
    Les signataires de la lettre doivent être les mandataires des membres de la FCEG.
  \end{enumerate}

 \item
  Toutes les décisions prises par le conseil d\textquotesingle administration de la FCEG sont définitives.
\end{enumerate}

\subsection{Appels ayant une incidence sur la CCI actuel}\label{appels-ayant-une-incidence-sur-la-cci-actuel}

\begin{enumerate}
 \item
  Tous les appels entrant dans cette catégorie doivent être interjetés dans la semaine suivant la conclusion de la CCI.
 \item
  Pour entamer la procédure d\textquotesingle appel, une lettre détaillant la décision faisant l\textquotesingle objet de l\textquotesingle appel et les motifs de celui-ci doit être soumise au président du CCCCI, qui doit ensuite en informer le CCCCI et le conseil d\textquotesingle administration de la FCEG.
 \item
  Le conseil d\textquotesingle administration de la FCEG doit répondre à l\textquotesingle appel dans un délai d\textquotesingle une semaine à compter de la notification du président du CCCCI.
 \item
  Ce type d\textquotesingle appel doit faire l\textquotesingle objet d\textquotesingle un consensus de la majorité de l\textquotesingle équipe qui dépose l\textquotesingle appel, attesté par les signatures de 50~\% + 1 des membres de l\textquotesingle équipe sur la lettre d\textquotesingle appel.

  \begin{enumerate}
   \item
    Le délégué régional en chef peut signer la lettre, mais sa signature ne sera pas prise en compte dans le calcul du pourcentage nécessaire pour obtenir le consensus majoritaire.
  \end{enumerate}

 \item
  Toutes les décisions prises par le conseil d\textquotesingle administration de la FCEG sont définitives.
\end{enumerate}
