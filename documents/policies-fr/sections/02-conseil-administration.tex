\section{Le Conseil d\textquotesingle administration}\label{le-conseil-dadministration}

\subsection{Les tâches et responsabilités}\label{les-tuxe2ches-et-responsabilituxe9s}

\begin{enumerate}
 \item
  Les tâches et les responsabilités du Conseil d\textquotesingle administration comprennent, mais ne sont pas limitées à:

  \begin{enumerate}
   \item
    La supervision des finances et de l'administration de la FCÉG
   \item
    L'établissement d'un ordre du jour avant toute réunion
   \item
    La détermination de la méthode électorale, en cas de besoin
   \item
    L'attribution, dans le cadre de la réunion du printemps, des sommes du fonds de voyage à chaque réunion importante de la FCÉG au cours de l'année fiscale. Ces réunions incluent:

    \begin{enumerate}
     \item
      La réunion d'été
     \item
      Le Sommet du développement des associations en ingénierie (SDAI)
     \item
      La réunion d'automne
     \item
      Le Congrès canadien sur le leadership en ingénierie (CCLI)
     \item
      La réunion du printemps
    \end{enumerate}
   \item
    Le suivi du progrès des membres de l'exécutif national et des commissaires
   \item
    La considération de tout appel au comité d\textquotesingle intervention des incidents
  \end{enumerate}
\end{enumerate}

\subsection{La conduite des directeur.trice.s}\label{la-conduite-des-directeur.trice.s}

\subsubsection{La conduite}\label{la-conduite}

\begin{enumerate}
 \item
  Les membres votants du Conseil d\textquotesingle administration, en tant que directeur.trice.s de la Fédération, doivent agir avec intégrité et de bonne foi au mieux des intérêts de la Fédération dans son ensemble, avec le soin, la diligence et la compétence d'une personne raisonnablement prudente dans l'exercice de leurs fonctions.
 \item
  Afin d'assurer que les directeur.trice.s agissent de la sorte, le Conseil d\textquotesingle administration doit approuver, au plus tard le 1er juin de chaque année, un code de conduite qui, au minimum, aborde les sujets suivants:

  \begin{enumerate}
   \item
    Les conflits d'intérêt, conformes avec les énoncés ci-dessous
   \item
    La culture
   \item
    Les responsabilités
   \item
    La confidentialité
  \end{enumerate}
 \item
  Les directeur.trice.s doivent signer le Code de conduite approuvé par le Conseil d\textquotesingle administration au plus tard le 15 juin de l'année de leur mandat, et la copie signée doit être conservée dans les documents de l'organisme.
\end{enumerate}

\subsubsection{Les conflits d\textquotesingle intérêts pour les directeur.trice.s}\label{les-conflits-dintuxe9ruxeats-pour-les-directeur.trice.s}

\begin{enumerate}
 \item
  Aucun.e directeur.trice ne peut participer dans une décision, exercer un pouvoir officiel ni remplir une fonction ou un devoir officiel s'il/elle risque d'être en situation de conflit d'intérêts, soit réel ou apparent.
 \item
  Ces circonstances comprennent :

  \begin{enumerate}
   \item
    Les dossiers dont un.e directeur.trice peut tirer un avantage personnel
   \item
    Les dossiers impliquant une association membre de la Fédération dont le/la directeur.trice est lui-même membre.
  \end{enumerate}
 \item
  Les directeur.trice.s sont tenus de déclarer toute situation réelle ou potentielle de conflit d\textquotesingle intérêt.
 \item
  Dans le cas où un.e directeur.trice estime qu'un.e autre directeur.trice est en situation de conflit d'intérêt, l'affaire sera portée devant le/la président.e du Conseil d\textquotesingle administration pour qu'il·elle prenne une décision.
 \item
  Le Conseil d\textquotesingle administration possède le droit d'annuler la décision du président/de la présidente par vote majoritaire.
 \item
  Dans le cas où le/la président.e du Conseil d\textquotesingle administration se trouve dans une situation de conflit d'intérêt, le/la président.e cédera la présidence au/à la vice-président.e jusqu'à ce que la matière soit résolue.
\end{enumerate}

\subsection{Les séances à huis clos}\label{les-suxe9ances-uxe0-huis-clos}

\begin{enumerate}
 \item
  Le Conseil d\textquotesingle administration peut, au besoin, se réunir à huis clos pour discuter de dossiers confidentiels.
 \item
  Les membres votants et non votants du Conseil d\textquotesingle administration sont invités à assister à ces rencontres, ainsi que d'autres personnes invitées par une résolution adoptée à la majorité des deux tiers (⅔).
 \item
  Les dossiers susceptibles de faire l'objet de rencontres à huis clos se limites aux :

  \begin{enumerate}
   \item
    Négociations contractuelles
   \item
    Affaires personnelles
   \item
    Matières qui constituent une menace à la réputation de la Fédération
  \end{enumerate}
 \item
  Pour procéder à une séance à huis clos, le Conseil d\textquotesingle administration doit adopter une résolution à la majorité des deux tiers (⅔).
 \item
  Le Conseil d\textquotesingle administration ne peut adopter aucune résolution lorsque la séance à huis clos a commencé, mis à part une motion pour prolonger la séance, ce qui nécessite encore une fois la majorité des deux tiers (⅔).
 \item
  Les personnes qui participent aux séances à huis clos sont tenues de préserver la confidentialité des dossiers traités.
 \item
  En plus des personnes présentes, seulement les directeur.trice.s qui n'ont pas assisté à la séance à huis clos peuvent être informés des discussions après coup.
 \item
  Seul le/la président.e du Conseil d\textquotesingle administration peut transmettre ces informations aux directeur.trice.s qui n'ont pas assisté à la séance à huis clos.
\end{enumerate}

\subsection{Les résolutions écrites}\label{les-ruxe9solutions-uxe9crites}

\begin{enumerate}
 \item
  Les directeur.trice.s doivent transmettre au/à la président.e du Conseil d\textquotesingle administration leur vote sur les résolutions dans un délai de trois jours ouvrables; cette exigence s'ajoute aux dispositions relatives aux résolutions écrites formulées dans la constitution.
 \item
  Les résolutions écrites peuvent être adoptées par des moyens électroniques jugés sécuritaires par le/la président.e du Conseil d\textquotesingle administration.
 \item
  Le/la président.e du Conseil d\textquotesingle administration doit annoncer le résultat de la résolution à la prochaine séance.
\end{enumerate}

