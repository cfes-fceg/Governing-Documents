\section{Le financement des voyages}\label{le-financement-des-voyages}

\subsection{L'allocation des fonds}\label{lallocation-des-fonds}

\begin{enumerate}
 \item
  Le budget présenté par le/la VPFA au CCLI comprendra une ligne budgétaire intitulée « Financement des voyages » qui comptabilisera tous les fonds à dépenser pour les voyages au cours de l\textquotesingle année fiscale de ce budget.
 \item
  Lors du SCIP, le/la VPFA présente une modification de la ligne budgétaire susmentionnée en une série de lignes qui allouent des fonds pour chaque activité de la FCÉG et pour tous les événements extérieurs auxquels l\textquotesingle exécutif national entrant a l\textquotesingle intention de participer.
 \item
  Ces fonds doivent être ventilés entre « voyages des membres du Conseil d\textquotesingle administration », « voyages des petites écoles » et « voyages des partenaires ».
 \item
  Une provision de 10 \% est également prévue pour les déplacements des membres du Conseil d\textquotesingle administration.
 \item
  Au moins le plus élevé des deux montants suivants, soit 5 000 \$ ou 40 \% du financement total des frais de voyage, doit être alloué au financement de l\textquotesingle adhésion au SDAI et au CCLI.
\end{enumerate}

\subsection{La réaffectation de fonds}\label{la-ruxe9affectation-de-fonds}

\begin{enumerate}
 \item
  Après la fin d\textquotesingle un événement pour lequel des fonds de voyage ont été alloués, le Conseil d\textquotesingle administration doit réaffecter les fonds de l\textquotesingle une des manières suivantes:

  \begin{enumerate}
   \item
    L'utilisation pour couvrir le déficit de lignes budgétaires similaires passées
   \item
    La réaffectation à un autre événement de nature similaire
   \item
    La prochaine opportunité de financement pour les membres la plus proche
   \item
    S\textquotesingle il n\textquotesingle y a pas de possibilités de financement ultérieures pour les membres, reporter à l\textquotesingle année suivante.
  \end{enumerate}
\end{enumerate}

\subsection{Les candidatures des officiers}\label{les-candidatures-des-officiers}

\subsubsection{La vue d\textquotesingle ensemble}\label{la-vue-densemble-2}

\begin{enumerate}
 \item
  En tant qu\textquotesingle administrateur.trice principal.e du fonds de voyage, le/la VPFA est chargé.e de coordonner la soumission des demandes pour les réunions des officiers, le Congrès canadien sur le leadership en ingénierie et le Sommet du développement des associations en ingénierie.
 \item
  Tous les agents doivent soumettre une demande, qu\textquotesingle ils soient candidat.e.s pour un financement de voyage.
 \item
  La demande comprendra une justification de l\textquotesingle intérêt pour le demandeur d\textquotesingle assister à la réunion.
 \item
  L\textquotesingle exécutif national présentera une recommandation au Conseil d\textquotesingle administration concernant l\textquotesingle allocation du Fonds et une recommandation distincte concernant les agents qui devraient être autorisés à assister à la réunion en question.
 \item
  Le Conseil d'administration est autorisé à approuver les recommandations telles qu\textquotesingle elles sont présentées ou à les modifier si nécessaire.
\end{enumerate}

\subsubsection{Les détails de la demande des officiers}\label{les-duxe9tails-de-la-demande-des-officiers}

\begin{enumerate}
 \item
  Les demandes de financement de voyage doivent comprendre au minimum les documents suivants :

  \begin{enumerate}
   \item
    Les documents justifiant le montant demandé. Il peut s\textquotesingle agir de reçus, du prix actuel des billets d\textquotesingle avion, de bus ou de train, d\textquotesingle un devis pour un véhicule de location ou d\textquotesingle un barème kilométrique pour un véhicule personnel, en suivant la procédure décrite à la section 11.3.3.
   \item
    La preuve que le financement de leur faculté a été épuisé;
   \item
    La preuve que les fonds de leur syndicat étudiant a été épuisé;
   \item
    La preuve que le financement de leur société d'ingénierie a été épuisé; et
   \item
    La preuve qu\textquotesingle ils ont choisi le mode de transport le moins coûteux possible et, s\textquotesingle ils voyagent par avion, qu\textquotesingle ils ont la possibilité d\textquotesingle effectuer des changements, même si cela inclut l\textquotesingle ajout de certains frais.
  \end{enumerate}
 \item
  Le financement peut être approuvé sur la base d\textquotesingle une estimation des coûts plutôt que sur la base d\textquotesingle un montant spécifique.
 \item
  Si le financement est approuvé sur la base des coûts estimés, il sera versé partiellement ou totalement, selon l\textquotesingle approbation du Conseil d\textquotesingle administration, une fois que la preuve du paiement aura été fournie.
 \item
  Le financement accordé ne dépassera le coût estimé fourni dans la demande que si les billets sont achetés dans les trois (3) jours suivant la réception par l\textquotesingle agent de la confirmation du financement.
\end{enumerate}

\subsubsection{Le remboursement des frais kilométriques}\label{le-remboursement-des-frais-kilomuxe9triques}

\begin{enumerate}
 \item
  Les demandes de fonds de voyage impliquant l\textquotesingle utilisation d\textquotesingle un véhicule personnel seront remboursées à un taux basé sur le calculateur des coûts de conduite de la CAA, à la discrétion du Conseil d\textquotesingle administration.
\end{enumerate}

\subsubsection{La rubrique d\textquotesingle évaluation de la demande des officiers}\label{la-rubrique-duxe9valuation-de-la-demande-des-officiers}

\begin{enumerate}
 \item
  Toutes les demandes sont évaluées sur la base des quatre critères suivants, pondérés de manière égale :

  \begin{enumerate}
   \item
    L'importance de la position de la personne demandant un financement lors de la réunion
   \item
    Les performances de la personne qui demande un financement au cours de la dernière l\textquotesingle année fiscale
   \item
    La preuve du besoin (la personne a-t-elle cherché d\textquotesingle autres sources de financement, etc.)
   \item
    Le caractère approprié du financement demandé (par exemple, le mode de transport est-il approprié, le coût est-il approprié, l\textquotesingle utilisation précédente du fonds de voyage, etc.)
  \end{enumerate}
\end{enumerate}

\subsection{Les candidatures des membres}\label{les-candidatures-des-membres}

\subsubsection{La vue d\textquotesingle ensemble}\label{la-vue-densemble-3}

\begin{enumerate}
 \item
  Seuls les membres qui bénéficient d\textquotesingle un financement à la date limite de dépôt des candidatures sont éligibles pour recevoir des fonds de voyage.
 \item
  Les fonds sont distribués selon l\textquotesingle ordre de priorité suivant:

  \begin{enumerate}
   \item
    Chaque membre recevra une proportion égale du montant de financement demandé pour un.e délégué.e.
   \item
    Chaque membre peut recevoir un financement pour un.e délégué.e supplémentaire si ce.tte délégué.e est un officier de la FCÉG, s\textquotesingle il a l\textquotesingle intention de se présenter à une élection en personne ou à une élection partielle, ou s\textquotesingle il présente une candidature à l\textquotesingle activité et s\textquotesingle il n\textquotesingle a pas l\textquotesingle intention d\textquotesingle être le mandataire des écoles.
   \item
    Chaque membre recevra un remboursement proportionnel de sa cotisation de délégué.e.
   \item
    Tout fonds supplémentaire doit être reporté à la prochaine opportunité de financement de l\textquotesingle adhésion ou au budget suivant.
  \end{enumerate}
\end{enumerate}

\subsubsection{Le financement des petites écoles}\label{le-financement-des-petites-uxe9coles}

\begin{enumerate}
 \item
  Les écoles peuvent demander le statut de petite école si elles comptent moins de 1 000 élèves ou disposent d\textquotesingle un budget annuel inférieur à 50 000 \$.
 \item
  Le statut de petite école est valable pour un (1) an suivant l\textquotesingle acceptation de la demande.
 \item
  Les demandes reçoivent une réponse dans un délai de deux (2) semaines.
 \item
  La demande comprend:

  \begin{enumerate}
   \item
    Un budget détaillé de la société candidate
   \item
    Une liste des occurrences où l\textquotesingle école a reçu des fonds en tant que petites écoles au cours des années précédentes
   \item
    Une explication détaillée de la manière dont le/la candidat.e a tenté de réunir le montant demandé auprès de son doyen, de ses anciens élèves et d\textquotesingle autres sources disponibles.
  \end{enumerate}
\end{enumerate}

\subsubsection{Le statut de financement spécial}\label{le-statut-de-financement-spuxe9cial}

\begin{enumerate}
 \item
  Les membres peuvent demander au Conseil d\textquotesingle administration de bénéficier d\textquotesingle un financement spécial s\textquotesingle ils se trouvent dans une situation financière imprévue.
 \item
  Le Conseil d\textquotesingle administration peut approuver leur statut spécial par un vote (⅔).
\end{enumerate}

\subsection{Le financement des demandes d\textquotesingle adhésion}\label{le-financement-des-demandes-dadhuxe9sion}

\begin{enumerate}
 \item
  Les membres ayant le statut de demandeur de financement doivent fournir une demande comprenant les éléments suivants :

  \begin{enumerate}
   \item
    Une explication des tentatives de financement
   \item
    Le montant demandé et coût prévu
   \item
    La preuve que le membre choisit une option de voyage fiscalement prudente
  \end{enumerate}
 \item
  La demande de financement pour un événement particulier est ouverte pendant au moins deux semaines.
 \item
  Les candidat.e.s doivent être informés du montant de leur financement au plus tard à l\textquotesingle ouverture de la préinscription.
 \item
  S\textquotesingle il n\textquotesingle y a pas de préinscription pour un événement, les candidat.e.s en seront informés au plus tard au début de l\textquotesingle inscription.
 \item
  Si les demandeurs sont informés tardivement, une prolongation leur est automatiquement accordée.
\end{enumerate}

\subsection{Frais de déplacement admissibles}\label{frais-de-duxe9placement-admissibles}

\subsubsection{Vols, trains et autobus}\label{vols-trains-et-autobus}

\begin{enumerate}
 \item
  Le financement des déplacements pour le CELC et les autres déplacements de plus de quatre jours peut inclure le financement d\textquotesingle un bagage enregistré ou d\textquotesingle un type de tarif incluant un bagage enregistré.
 \item
  Pour tous les autres déplacements, le financement d\textquotesingle un bagage à main ou d\textquotesingle un type de tarif incluant un bagage à main peut être envisagé.
 \item
  Le choix du siège, l\textquotesingle espace supplémentaire pour les jambes ou tout autre supplément facturé par le transporteur ne seront pas pris en compte pour le financement prioritaire.
 \item
  Les types de billets doivent être achetés comme suit :

\begin{enumerate}
 \item
  Plus de 90 jours avant le départ, un billet entièrement remboursable doit être acheté au minimum.
 \item
  Entre 30 et 90 jours avant le départ, un billet permettant des modifications, même moyennant des frais, doit être acheté au minimum.
 \item
  Moins de 30 jours avant le départ, aucune modification ni aucun remboursement n\textquotesingle est requis.
\end{enumerate}
\end{enumerate}

\subsubsection{Utilisation d\textquotesingle un véhicule de location}\label{utilisation-dun-vuxe9hicule-de-location}

\begin{enumerate}
 \item
  Les demandes de financement de voyage impliquant l\textquotesingle utilisation d\textquotesingle un véhicule de location peuvent être approuvées à la discrétion du conseil d\textquotesingle administration.
 \item
  La responsabilité du véhicule de location est entièrement assumée par la personne.
 \item
  Le remboursement peut inclure le coût du véhicule de location, du stationnement et des reçus de carburant.
\end{enumerate}

\subsubsection{Utilisation d\textquotesingle un véhicule personnel}\label{utilisation-dun-vuxe9hicule-personnel}

\begin{enumerate}
 \item
  Les demandes de financement de voyage impliquant l\textquotesingle utilisation d\textquotesingle un véhicule personnel peuvent être approuvées à la discrétion du conseil d\textquotesingle administration.
 \item
  La responsabilité du véhicule personnel est entièrement assumée par la personne.
 \item
  Le remboursement est calculé à raison de 0,35 \$ par kilomètre, réparti comme suit :

\begin{enumerate}
 \item
  0,15 \$/km pour les frais de carburant
 \item
  0,10 \$/km pour la dépréciation
 \item
  0,10 \$/km pour l\textquotesingle entretien
\end{enumerate}

   \item
    Le remboursement peut également inclure les frais de stationnement.
\end{enumerate}

\subsubsection{Taxis et covoiturage}\label{taxis-et-covoiturage}

\begin{enumerate}
 \item
  Les demandes de financement de déplacements en taxi ou en covoiturage peuvent être approuvées à la discrétion du conseil d\textquotesingle administration.
 \item
  Les taxis et le covoiturage sont déconseillés lorsqu\textquotesingle il existe des moyens de transport public viables.
 \item
  Lorsque des moyens de transport public viables sont disponibles, le financement des taxis ou du covoiturage peut être limité au coût qu\textquotesingle aurait représenté le transport public.
 \item
  La faisabilité des moyens de transport public est laissée à la discrétion du VPFA, qui tient compte de facteurs tels que la durée, la proximité et l\textquotesingle heure de la journée.
 \item
  Le covoiturage avec d\textquotesingle autres personnes est encouragé afin de réduire les coûts.
\end{enumerate}

\subsubsection{Déplacement vers et depuis le point de départ}\label{duxe9placement-vers-et-depuis-le-point-de-duxe9part}

\begin{enumerate}
 \item
  Les frais liés au déplacement vers et depuis le lieu de départ principal peuvent être approuvés par le conseil d\textquotesingle administration, conformément aux critères de remboursement applicables au mode de transport concerné.
\end{enumerate}

\subsubsection{Déplacement vers et depuis le point d\textquotesingle arrivée}\label{duxe9placement-vers-et-depuis-le-point-darrivuxe9e}

\begin{enumerate}
 \item
  Les frais liés au déplacement entre le point d\textquotesingle arrivée et les hébergements ou lieux locaux ne sont pas prioritaires pour le financement.
\end{enumerate}

\subsubsection{Repas}\label{repas}

\begin{enumerate}
 \item
  Les déplacements de plus de 9 heures, sans compter le temps d\textquotesingle attente, peuvent donner droit à des subventions pour les repas.
\end{enumerate}

\subsubsection{Frais de délégation}\label{frais-de-duxe9luxe9gation}

\begin{enumerate}
 \item
  Les frais de délégation à une conférence ne seront pas pris en compte pour le financement prioritaire.
\end{enumerate}

