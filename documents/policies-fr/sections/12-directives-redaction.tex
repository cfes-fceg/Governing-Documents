\section{Les directives sur la rédaction des documents}\label{les-directives-sur-la-ruxe9daction-des-documents}

\subsection{Les directives sur la traduction}\label{les-directives-sur-la-traduction}

\begin{enumerate}
 \item
  Comme règle générale, les documents suivants doivent être traduits avant d'être distribués:

  \begin{enumerate}
   \item
    La Constitution et le Manuel des politiques
   \item
    Les documents à caractère obligatoire
   \item
    Le plan stratégique
   \item
    Les règlements de la CCI
   \item
    L'information destinée aux étudiants faisant partie des associations membres (comme les trousses remises aux délégué.e.s, les rapports de sondage, les manuels d'information, les guides, et ainsi de suite)
   \item
    Toute communication diffusée par l'exécutif national auprès de l'ensembles de ses membres
   \item
    Toute documentation liée aux plaintes et aux procédures
   \item
    Le procès-verbal des assemblées générales
   \item
    Les rapports des officiers
   \item
    Les rapports officiels sur les enjeux importants
   \item
    Les documents du CCLI
   \item
    Le Cahier des positions
   \item
    Les procès-verbaux et tout autre document important pour les assemblées générales
  \end{enumerate}
 \item
  La Constitution et le Manuel de politique doivent être traduits au moins une fois par an.
 \item
  L'agenda et tout autre document pour l'assemblée générale pourront être distribués avant leur traduction.
 \item
  Le procès-verbal des assemblées générales et les rapports pourront aussi être distribués avant leur traduction.
 \item
  Les documents doivent être traduits dans leurs entiers au plus vite en Anglais et en Français.
 \item
  Tout membre peut faire la requête de la traduction de tout autre document.
\end{enumerate}

\subsection{La féminisation des textes en français}\label{la-fuxe9minisation-des-textes-en-franuxe7ais}

\begin{enumerate}
 \item
  Tout texte écrit en français ou traduit vers le français doit être féminisé ou désexisé, notamment en appliquant les lignes directrices énoncées dans le Guide de rédaction épicène publié par l'Office québécois de la langue française.
\end{enumerate}

\subsection{Le comité de gouvernance}\label{le-comituxe9-de-gouvernance}

\subsubsection{La vue d\textquotesingle ensemble}\label{la-vue-densemble-4}

\begin{enumerate}
 \item
  Le comité de gouvernance est chargé d\textquotesingle examiner et de garantir l\textquotesingle intégrité et l\textquotesingle efficacité des documents de gouvernance de l\textquotesingle organisation, y compris la constitution, les statuts et les politiques.
 \item
  Le comité présentera les changements proposés au Conseil d\textquotesingle administration et à l\textquotesingle assemblée générale le cas échéant, en veillant à ce que toutes les parties prenantes concernées soient informées et engagées dans le processus de gouvernance.
\end{enumerate}

\subsubsection{Les tâches et les responsabilités}\label{les-tuxe2ches-et-les-responsabilituxe9s}

\begin{enumerate}
 \item
  Les tâches et responsabilités du comité de gouvernance sont les suivantes:

  \begin{enumerate}
   \item
    Le comité se réunit chaque mois pour examiner systématiquement les sections des documents de référence, en veillant à ce que tous les documents de référence fassent l\textquotesingle objet d\textquotesingle un examen complet chaque année.
   \item
    Le comité prépare et présente les propositions de modifications au Conseil d\textquotesingle administration et à l\textquotesingle Assemblée générale si nécessaire.
   \item
    Le comité se tient informé des meilleures pratiques en matière de gouvernance afin de s\textquotesingle assurer que les documents de référence sont conformes aux normes et aux exigences réglementaires en vigueur.
  \end{enumerate}
\end{enumerate}

\subsubsection{La composition}\label{la-composition}

\begin{enumerate}
 \item
  Le comité de gouvernance est présidé par le/la président.e du Conseil d\textquotesingle administration et se compose des membres suivants.

  \begin{enumerate}
   \item
    Le/la président.e du Conseil d\textquotesingle administration
   \item
    Le/la président.e de la FCÉG
   \item
    Le/la VPFA
   \item
    Au moins un.e (1) conseiller.ère national.e
  \end{enumerate}
 \item
  Un nombre illimité d'officiers supplémentaires peut être sélectionné par le Conseil d\textquotesingle administration pour rejoindre le comité à tout moment au cours du mandat, en fonction de l\textquotesingle intérêt et de l\textquotesingle expertise pertinente.
 \item
  Ces officiers peuvent faire part de leur intérêt en contactant le/la président.e du Conseil d\textquotesingle administration.
 \item
  Le Conseil d\textquotesingle administration s\textquotesingle efforce de veiller à ce que le comité soit composé d\textquotesingle un groupe diversifié et bien informé, capable de contribuer efficacement à l\textquotesingle examen et à l\textquotesingle amélioration des documents de référence de l\textquotesingle organisation.
\end{enumerate}

