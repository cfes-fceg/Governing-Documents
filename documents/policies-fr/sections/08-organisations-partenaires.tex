\section{Les organisations partenaires}\label{les-organisations-partenaires}

\subsection{L'objectif}\label{lobjectif-1}

\begin{enumerate}
 \item
  Les partenaires sont des organisations associées qui partagent la vision et la mission de la FCÉG.
 \item
  La Fédération travaille avec ces groupes dans l\textquotesingle intérêt mutuel.
 \item
  Les nouveaux partenaires doivent être reconnus par une résolution du Conseil d\textquotesingle administration après que l'exécutif national s'assure du respect des dispositions énumérées ci-après.
\end{enumerate}

\subsection{Les partenaires étudiants}\label{les-partenaires-uxe9tudiants}

\subsubsection{L'objectif}\label{lobjectif-2}

\begin{enumerate}
 \item
  La Fédération compte deux types de partenaires étudiants: les partenaires étudiants à part entière et les partenaires étudiants permanents.
 \item
  Les contrats et/ou accords de partenariat avec ces organisations doivent inclure les dispositions, ou des dispositions similaires, contenues dans la présente section.
\end{enumerate}

\subsubsection{Les partenaires étudiants à part entière}\label{les-partenaires-uxe9tudiants-uxe0-part-entiuxe8re}

\begin{enumerate}
 \item
  Les avantages dont bénéficient les partenaires étudiants à part entière sont les suivants :

  \begin{enumerate}
   \item
    La participation au CCLI sans frais d\textquotesingle inscription
   \item
    La participation à d\textquotesingle autres événements de la FGÉG avec frais d'inscription, sauf dans le cas d'une décision contraire prise dans le cadre d\textquotesingle un accord ratifié.
   \item
    Utilisation du réseau de la FCÉG: Possibilité d\textquotesingle envoyer des courriels aux "membres-liens", logo sur le site web sous la rubrique "Partenaire étudiant", publicité pour les événements et les services.
  \end{enumerate}
 \item
  Les exigences auxquelles les partenaires étudiants à part entière doivent satisfaire sont les suivantes :

  \begin{enumerate}
   \item
    Disposer de documents de gestion et être incorporé dans la juridiction appropriée, qu\textquotesingle il s\textquotesingle agisse d\textquotesingle un pays, d\textquotesingle une province ou d\textquotesingle un État.
   \item
    Faire preuve d\textquotesingle un objectif et d\textquotesingle une orientation, par le biais d\textquotesingle un plan stratégique ou d\textquotesingle une autre forme de document de planification à long terme.
   \item
    Doit prouver qu\textquotesingle il y a des étudiants membres, par exemple par le biais de courriels.
   \item
    Doit avoir un logo
   \item
    Avoir une présence en ligne, comme un site web ou une page Facebook.
   \item
    Avoir un document proposé (c\textquotesingle est-à-dire un contrat) avec la FCÉG décrivant les attentes en matière de relations au moment où la proposition de partenariat est présentée au Conseil d\textquotesingle administration.
  \end{enumerate}
\end{enumerate}

\subsubsection{Les partenaires étudiants permanents}\label{les-partenaires-uxe9tudiants-permanents}

\begin{enumerate}
 \item
  Les avantages dont bénéficient les partenaires étudiants permanents sont les suivants:

  \begin{enumerate}
   \item
    La visibilité auprès des membres de la FCÉG
   \item
    Les avantages spécifiés par le Conseil d\textquotesingle administration
   \item
    Ne pas être identifié (sur le site web ou ailleurs) avec les partenaires étudiants à part entière.
  \end{enumerate}
 \item
  Les exigences auxquelles les partenaires étudiants permanents doivent satisfaire sont les suivantes :

  \begin{enumerate}
   \item
    La compatibilité avec la mission de la FCÉG
   \item
    Être une organisation étudiante
   \item
    Doit être assigné au commissaire aux relations internationales ou au commissaire aux relations d'entreprises, suivi et avancé par le/la président.e.
  \end{enumerate}
\end{enumerate}

\subsection{Les partenaires professionnels}\label{les-partenaires-professionnels}

\begin{enumerate}
 \item
  Les partenaires professionnels de la Fédération sont des organisations qui représentent les intérêts de la profession d\textquotesingle ingénieur.eure et avec lesquelles la FCÉG a conclu des accords reconnus.
 \item
  Bien qu\textquotesingle il n\textquotesingle y ait pas de conditions particulières pour obtenir ce statut en dehors d\textquotesingle une résolution du Conseil d\textquotesingle administration, les dispositions suivantes sont recommandées :

  \begin{enumerate}
   \item
    La mission compatible avec celle de la FCÉG
   \item
    Les services pertinents pour les membres
   \item
    Incorporé en tant que personne morale.
  \end{enumerate}
 \item
  Un partenaire professionnel proposé doit disposer d\textquotesingle un document (c\textquotesingle est-à-dire d\textquotesingle un contrat) avec la FCÉG décrivant les attentes en matière de relations au moment où la proposition de partenariat est présentée au Conseil d\textquotesingle administration.
\end{enumerate}

\subsection{La révision du partenariat}\label{la-ruxe9vision-du-partenariat}

\begin{enumerate}
 \item
  Un rapport officiel d\textquotesingle évaluation des partenariats doit être établi tous les deux ans pour les partenaires étudiants et professionnels actuels et en cours de développement de la FCÉG.
 \item
  Ce rapport sera dirigé par le/la VPE et soutenu par l\textquotesingle équipe de officier de la FCÉG.
 \item
  L\textquotesingle objectif de ce rapport d\textquotesingle examen des partenariats est de garantir que la FCÉG procède à un examen transparent et approfondi de la manière dont chaque partenaire (ou partenaire en développement) s\textquotesingle aligne sur les valeurs de la Fédération.
 \item
  Chaque étude de partenariat doit contenir des recherches et des informations sur les points suivants :

  \begin{enumerate}
   \item
    La révision de l\textquotesingle organisation

    \begin{enumerate}
     \item
      Le nom de l\textquotesingle organisation
     \item
      Le secteur d\textquotesingle activité
     \item
      Le domaine général de travail
     \item
      La mission et la vision
     \item
      La champ d\textquotesingle action
    \end{enumerate}
   \item
    La publicité et/ou les préoccupations

    \begin{enumerate}
     \item
      Les liens vers des sources
    \end{enumerate}
   \item
    La comparaison des valeurs

    \begin{enumerate}
     \item
      La diversité
    \end{enumerate}
   \item
    Le développement durable
   \item
    La qualité de l\textquotesingle éducation des étudiant.e.s
   \item
    La création d\textquotesingle espaces plus sûrs
   \item
    Le respect et reconnaissance de l\textquotesingle autochtonie dans l\textquotesingle ingénierie
   \item
    Le démantèlement des systèmes d\textquotesingle oppression
   \item
    Prendre soin des personnes et de l\textquotesingle environnement
   \item
    La capacité d\textquotesingle engagement et d\textquotesingle influence
   \item
    Le sommaire
   \item
    Les commentaires supplémentaires de l\textquotesingle évaluateur.trice
  \end{enumerate}
\end{enumerate}

