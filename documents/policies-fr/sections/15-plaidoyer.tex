\section{Le plaidoyer}\label{le-plaidoyer}

\subsection{L'apport des membres}\label{lapport-des-membres}

\begin{enumerate}
 \item
  Pour toute réunion externe à laquelle participe l\textquotesingle exécutif national, un dossier de réunion comprenant toutes les présentations que l\textquotesingle exécutif national prévoit de faire doit être envoyé au Conseil d\textquotesingle administration au moins 7 jours avant le début de la réunion pour qu\textquotesingle il y apporte sa contribution, ou dès qu\textquotesingle il est disponible.
 \item
  Les ambassadeurs.drices régionaux.ales peuvent alors consulter les membres sur le sujet de ces présentations, à leur discrétion.
 \item
  Le retour d\textquotesingle information sur la présentation respective sera recueilli au moyen d\textquotesingle un formulaire établi par l\textquotesingle exécutif national et complété par l\textquotesingle ambassadeur.drice régional.e.
\end{enumerate}

\subsection{Les déclarations officielles, les mémorandums, les lettres de soutien et les communiqués}\label{les-duxe9clarations-officielles-les-muxe9morandums-les-lettres-de-soutien-et-les-communiquuxe9s}

\begin{enumerate}
 \item
  Toute déclaration officielle de la Fédération doit être approuvée par le Conseil d\textquotesingle administration selon le processus suivant.
 \item
  Voici quelques exemples de raisons de faire des déclarations publiques:

  \begin{enumerate}
   \item
    Une demande d\textquotesingle adhésion pour la défense des intérêts locaux
   \item
    Une demande des partenaires
   \item
    Les évènements mondiaux
  \end{enumerate}
\end{enumerate}

\subsection{La déclaration conforme aux positions}\label{la-duxe9claration-conforme-aux-positions}

\begin{enumerate}
 \item
  La déclaration est accompagnée d\textquotesingle un document démontrant son lien avec le Cahier de positions de la FCÉG.

  \begin{enumerate}
   \item
    Les ambassadeur.drice.s régionaux.ales doivent distribuer les deux documents pour obtenir un retour d\textquotesingle information dans un délai minimum de 48 heures.
   \item
    Le Conseil d\textquotesingle administration peut alors organiser un vote pour approuver la déclaration, après lequel elle peut être diffusée.
  \end{enumerate}
\end{enumerate}

\subsection{Les déclarations qui ne correspondent pas explicitement aux prises de positions}\label{les-duxe9clarations-qui-ne-correspondent-pas-explicitement-aux-prises-de-positions}

\begin{enumerate}
 \item
  La déclaration est préparée avec un document d\textquotesingle accompagnement qui explique la motivation de la déclaration.
 \item
  Les ambassadeur.drice.s régionaux.ales doivent distribuer les deux documents pour obtenir un retour d\textquotesingle information dans un délai minimum de 14 jours.
 \item
  Les ambassadeur.drice.s régionaux.ales peuvent, à leur discrétion, demander des modifications à la déclaration sur la base des commentaires reçus.
 \item
  La déclaration modifiée doit être distribuée avec une réponse au retour d\textquotesingle information.
 \item
  Le délai de réponse aux déclarations modifiées est de 7 jours.
 \item
  Le Conseil d\textquotesingle administration peut alors organiser un vote pour approuver la déclaration, après lequel elle peut être diffusée publiquement.
\end{enumerate}

