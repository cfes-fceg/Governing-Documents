\section{Le financement et l'adhésion}\label{le-financement-et-ladhuxe9sion}

\subsection{Les candidatures à l\textquotesingle adhésion}\label{les-candidatures-uxe0-ladhuxe9sion}

\begin{enumerate}
 \item
  Les demandes d\textquotesingle adhésion à la Fédération ne peuvent être examinées que par l\textquotesingle assemblée générale.
 \item
  Conformément aux lignes directrices énoncées dans les statuts, les demandes d\textquotesingle adhésion doivent contenir (sans s\textquotesingle y limiter) les informations suivantes:

  \begin{enumerate}
   \item
    Le nom de l\textquotesingle organisation
   \item
    Le but ou les objectifs de l\textquotesingle organisation
   \item
    Les dirigeants actuels et membres du Conseil d\textquotesingle administration
   \item
    La date de création de l\textquotesingle organisation
   \item
    Le nombre d\textquotesingle étudiants dans la faculté (ou équivalent)
   \item
    La preuve que l\textquotesingle organisation candidate représente des étudiants de programmes d\textquotesingle ingénierie accrédités ou en cours d\textquotesingle accréditation par le BCAPG.
   \item
    La preuve du soutien de l'association des étudiants en ingénierie (motion adoptée dans le procès-verbal de la réunion publique)
  \end{enumerate}
 \item
  Toute société qui assiste à une session de l\textquotesingle assemblée générale avec l\textquotesingle intention de devenir membre à cette occasion sera considérée comme délégué et non comme observateur.
\end{enumerate}

\subsection{Le budget}\label{le-budget}

\begin{enumerate}
 \item
  La FCÉG fonctionne sur la base d\textquotesingle un budget annuel approuvé au CCLI.
 \item
  Toute modification de ce budget approuvé de plus de 25 \% d\textquotesingle un poste dépassant à l\textquotesingle origine 5 000 \$ doit être approuvée d\textquotesingle abord par le Conseil d\textquotesingle administration, puis ratifiée par l\textquotesingle assemblée générale.
\end{enumerate}

\subsection{Les excédents et les déficits d\textquotesingle activité}\label{les-excuxe9dents-et-les-duxe9ficits-dactivituxe9}

\subsubsection{Les fonds d\textquotesingle activités}\label{les-fonds-dactivituxe9s}

\begin{enumerate}
 \item
  Cette section a été créée afin de réglementer la distribution des excédents et le remboursement des déficits des activités et services de la FCÉG.
 \item
  Le fonds d\textquotesingle activités sera utilisé pour conserver tous les surplus et payer tous les déficits en cas d\textquotesingle urgence, par exemple en cas de manque de commandites.
 \item
  Le fonds d\textquotesingle activités devrait être augmenté de manière à conserver au moins 2 000 \$ pour chaque activité, y compris le Sommet du développement des associations en ingénierie, le Congrès sur la diversité en ingénierie, la Compétition canadienne d\textquotesingle ingénierie et le Congrès sur le développement durable en ingénierie, et 10 000 \$ pour le Congrès canadien sur le leadership en ingénierie, avec un total supplémentaire de 35 000 \$ à utiliser comme fonds de démarrage pour toutes les activités.
 \item
  Les montants des fonds de démarrage seront alloués comme suit :

  \begin{enumerate}
   \item
    Le Sommet du développement des associations en ingénierie : 5 000 \$.
   \item
    Le Congrès sur la diversité en ingénierie : 7 000 \$
   \item
    Le Congrès canadien sur le leadership en ingénierie : 9 000 \$
   \item
    Le Congrès sur le développement durable en ingénierie : 7 000 \$
   \item
    La Compétition canadienne d\textquotesingle ingénierie : 7 000 \$.
  \end{enumerate}
 \item
  La valeur totale minimale du fonds d\textquotesingle activité devrait être de 53 000 \$.
 \item
  Si, après l\textquotesingle accomplissement de son mandat, un compte de service ou d\textquotesingle activité de la FCÉG présente un surplus ou un déficit, le/la VPFA est responsable d\textquotesingle assurer la procédure pour soit réaffecter l\textquotesingle excédent ou payer le déficit. Ceci doit aussi être suivie conformément à la section suivante et à la convention d\textquotesingle activité correspondante, dans un délai d\textquotesingle un mois à compter de la fin de l\textquotesingle activité.
\end{enumerate}

\subsubsection{L'utilisation des fonds pour couvrir les excédents d\textquotesingle activité}\label{lutilisation-des-fonds-pour-couvrir-les-excuxe9dents-dactivituxe9}

\begin{enumerate}
 \item
  En cas d\textquotesingle excédent d\textquotesingle activité, les étapes suivantes seront suivies :

  \begin{enumerate}
   \item
    Créer un comité de transition composé du/de la VPFA et des responsables d\textquotesingle activités entrant.s et sortant.s.
   \item
    Vérifier la présence d\textquotesingle un excédent ou d\textquotesingle un déficit
  \end{enumerate}
\end{enumerate}

\subsubsection{L'utilisation des fonds pour couvrir les déficits d\textquotesingle activité}\label{lutilisation-des-fonds-pour-couvrir-les-duxe9ficits-dactivituxe9}

\begin{enumerate}
 \item
  En cas de transfert d\textquotesingle un excédent, celui-ci est transféré dans le fonds d\textquotesingle activité jusqu\textquotesingle à ce qu\textquotesingle il soit épuisé.
 \item
  Dans le cas d\textquotesingle une activité déficitaire, la FCÉG paiera le déficit impayé par l\textquotesingle intermédiaire du fonds d\textquotesingle activité, et la société membre qui a signé la convention d\textquotesingle activité ne sera plus en règle avec la FCÉG.
 \item
  Les étapes suivantes seront suivies afin que la société d\textquotesingle ingénierie puisse retrouver son statut de membre en règle de la FCÉG :

  \begin{enumerate}
   \item
    Créer le comité de transition désigné par le Conseil d\textquotesingle administration
   \item
    Vérifier la présence d\textquotesingle un déficit
   \item
    La commission examinera la demande, qui doit comprendre les éléments suivants :

    \begin{enumerate}
     \item
      Une lettre du comité demandeur
     \item
      Le montant demandé et l\textquotesingle urgence de la demande
     \item
      Le budget proposé et le budget réel
     \item
      La trésorerie et les équivalents de trésorerie à jour
     \item
      Les raisons des écarts et des déficits
     \item
      Les plans d\textquotesingle urgence
     \item
      Les conséquences d\textquotesingle un refus de financement
     \item
      La preuve que le/la président.e du comité d\textquotesingle organisation est au courant de la situation financière
     \item
      La preuve qu\textquotesingle ils ont épuisé les fonds de leur faculté
     \item
      La preuve qu\textquotesingle il a épuisé les fonds de son association d'étudiants en ingénierie
     \item
      La preuve qu\textquotesingle ils ont épuisé le financement de leur société d\textquotesingle ingénierie
     \item
      D'autres informations justificatives, le cas échéant
    \end{enumerate}
  \end{enumerate}
 \item
  Tout financement supplémentaire obtenu par le comité organisateur dans le but de combler le déficit sera versé au fonds d\textquotesingle activité.
 \item
  Le comité organisateur doit obtenir l\textquotesingle approbation du Conseil d\textquotesingle administration pour solliciter des fonds auprès de sources non spécifiées ci-dessus dans le but de combler le déficit.
 \item
  Une fois que les documents soumis ont été examinés par le comité, le responsable d\textquotesingle activité doit présenter le dossier de financement en temps réel (c\textquotesingle est-à-dire par conférence téléphonique ou en personne).
 \item
  Après la présentation, le/la responsable d\textquotesingle activité doit répondre à toutes les questions du comité.
 \item
  Le comité doit être créé au maximum deux semaines après l\textquotesingle approbation du/de la président.e de la FCÉG.
 \item
  La décision du comité doit être présentée au Conseil d\textquotesingle administration lors de la réunion qui suit la décision.
 \item
  Toute la documentation doit être archivée afin d\textquotesingle assurer la préséance pour les cas futurs.
\end{enumerate}

\subsection{L'utilisation du fonds d\textquotesingle activité}\label{lutilisation-du-fonds-dactivituxe9}

\subsubsection{Les fonds excédentaires}\label{les-fonds-excuxe9dentaires}

\begin{enumerate}
 \item
  Chaque année, au cours de la première semaine de janvier, un membre, un/une commissaire peut présenter au Conseil d\textquotesingle administration une proposition de distribution de 5 \% des fonds excédentaires du fonds d\textquotesingle activité.
 \item
  Si une proposition est soumise, le Conseil d\textquotesingle administration créera une commission chargée d\textquotesingle évaluer les propositions et les fonds disponibles, qui ne pourra faire qu\textquotesingle une recommandation au Conseil d\textquotesingle administration.
 \item
  La composition de la commission est déterminée par le Conseil d\textquotesingle administration :
 \item
  Les possibilités par lesquelles les fonds peuvent être alloués sont les suivantes :

  \begin{enumerate}
   \item
    Le fonds d\textquotesingle activité de dons pour les activités et services de la FCÉG
   \item
    Le fonds d\textquotesingle activités spéciales pour les initiatives scolaires
   \item
    Le fonds de voyage pour les petites écoles
  \end{enumerate}
 \item
  Tous les documents doivent être archivés afin de servir de référence pour les cas futurs.
\end{enumerate}

\subsubsection{Les prêts du Fonds d'activité}\label{les-pruxeats-du-fonds-dactivituxe9}

\begin{enumerate}
 \item
  Au cas où le besoin de liquidités d'une activité serait supérieur au solde de son compte de dépôt à ce moment, le/la responsable d'activité peut demander un prêt au responsable du fonds d'activités pour couvrir les dépenses à court terme.
 \item
  Il faut joindre les éléments suivants à la demande :

  \begin{enumerate}
   \item
    Une lettre du comité demandeur
   \item
    Le montant demandé et le degré de priorité de la demande
   \item
    Le budget proposé et définitif
   \item
    La valeur à jour des espèces et des quasi-espèces
  \end{enumerate}
 \item
  Si le montant de l'avance en espèces est inférieur à 10 000 \$, le/la VPFA a le pouvoir d'approuver la demande en envoyant un avis écrit au Conseil d\textquotesingle administration.
 \item
  Au cas où le montant serait supérieur, la demande serait soumise directement à l'approbation du Conseil d\textquotesingle administration.
 \item
  L'avance en espèces accordée ne portera pas intérêt.
 \item
  Les capitaux avancés seront exigibles à la clôture des états financiers de l'activité à la date précisée dans l'entente d'activité spécifique.
\end{enumerate}

\subsection{Les exigences bancaires}\label{les-exigences-bancaires}

\subsubsection{Les autorités de signature}\label{les-autorituxe9s-de-signature}

\begin{enumerate}
 \item
  Le/la président.e peut détenir jusqu\textquotesingle à deux droits de signature sur tout compte bancaire de la FCÉG.
 \item
  Le/la VPFA peut détenir jusqu\textquotesingle à un droit de signature sur tout compte bancaire de la FCÉG; toutefois, toute transaction d\textquotesingle une valeur supérieure à 3 000 \$ doit être signée par le/la président.e en plus du/de la VPFA.
 \item
  Tous les autres directeur.trice.s sont limités aux privilèges « Tous à signer ».
\end{enumerate}

\subsubsection{La répartition des pouvoirs de signature actuels}\label{la-ruxe9partition-des-pouvoirs-de-signature-actuels}

\begin{enumerate}
 \item
  Les exigences énoncées à la section 8.6.1 sont respectées dans le cadre de la procédure décrite dans le document intitulé « Procédures de pouvoir de signature de la FCÉG », qui sera ajouté en annexe au manuel des politiques de la FCÉG.
\end{enumerate}

\subsection{Perception des frais d\textquotesingle inscription des délégués}\label{perception-des-frais-dinscription-des-duxe9luxe9guuxe9s}

\begin{enumerate}
 \item
  Les frais d\textquotesingle inscription des délégués pour toutes les activités seront perçus par l\textquotesingle intermédiaire des comptes CFES.
 \item
  L\textquotesingle activité facturera tous les frais d\textquotesingle inscription des délégués à l\textquotesingle aide du logiciel fourni par le CFES.
 \item
  Les factures seront envoyées aux membres au plus tard une (1) semaine après la date limite d\textquotesingle inscription.
 \item
  Le délai de paiement des factures ne sera pas inférieur à trente (30) jours après l\textquotesingle envoi de la facture.
 \item
  L\textquotesingle activité suivra l\textquotesingle état des paiements de tous les frais d\textquotesingle inscription à l\textquotesingle aide d\textquotesingle un logiciel fourni par la CFES.
 \item
  La CFES pourra transférer les montants des factures payées sur le compte de l\textquotesingle activité.
 \item
  Après l\textquotesingle activité, le montant total des frais d\textquotesingle inscription facturés sera versé à l\textquotesingle activité, quel que soit le statut de paiement de la facture.

  \begin{enumerate}
   \item
    Pour les comptes d\textquotesingle activité, les fonds seront versés dans les 30 jours suivant la confirmation du montant final des frais d\textquotesingle inscription.
   \item
    Pour les activités dont les comptes financiers sont gérés par une organisation externe, les fonds seront versés dans les 30 jours suivant la réception de la facture.
  \end{enumerate}
 \item
  La CFES sera chargée de s\textquotesingle assurer que le paiement des factures impayées est bien reçu.
\end{enumerate}

\subsection{Comptes bancaires des activités de la FCEG Structure}\label{comptes-bancaires-des-activituxe9s-de-la-fceg-structure}

\subsubsection{Structure}\label{structure}

\begin{enumerate}
 \item
  La CFES fournira un compte bancaire désigné, ci-après appelé « compte d\textquotesingle activité », pour les activités qui choisissent de gérer leurs finances de conférence par l\textquotesingle intermédiaire de la CFES.
 \item
  Le compte d\textquotesingle activité sera un compte d\textquotesingle exploitation d\textquotesingle entreprise à but non lucratif CIBC.
 \item
  Le compte d\textquotesingle activité sera géré par l\textquotesingle intermédiaire de CIBC SmartBanking™ for Business, ci-après appelé « SmartBanking ».
\end{enumerate}

\subsubsection{Pouvoirs de signature}\label{pouvoirs-de-signature}

\begin{enumerate}
 \item
  Les comptes d\textquotesingle activité étant des sous-comptes de la CFES, les seules personnes habilitées à signer sur les comptes d\textquotesingle activité seront le vice-président des finances et le président de la CFES.
\end{enumerate}

\subsubsection{Accès}\label{accuxe8s}

\begin{enumerate}
 \item
  Le ou les présidents d\textquotesingle activité et le ou les vice-présidents des finances d\textquotesingle activité seront ajoutés en tant qu\textquotesingle utilisateurs du compte d\textquotesingle activité sur SmartBanking.
 \item
  Toutes les transactions effectuées à partir du compte d\textquotesingle activité devront être approuvées en deux étapes.
 \item
  Le ou les présidents d\textquotesingle activité du compte d\textquotesingle activité disposeront des autorisations « créer ou approuver ».
 \item
  Le ou les vice-présidents financiers d\textquotesingle activité du compte d\textquotesingle activité disposeront des autorisations « créer ou approuver » ou « consulter ou créer », à la discrétion du ou des présidents d\textquotesingle activité.
\end{enumerate}

\subsubsection{Fonctionnalité}\label{fonctionnalituxe9}

\begin{enumerate}
 \item
  Les virements électroniques, également appelés transferts électroniques de fonds (TEF), doivent être effectués à l\textquotesingle aide de la fonction Interac e-Transfer® for Business de SmartBanking.
 \item
  Les dépôts directs, également appelés transferts électroniques de fonds (TEF), doivent être effectués à l\textquotesingle aide de la fonction Interac e-Transfer® for Business de SmartBanking, en utilisant le numéro de routage du compte.

  \begin{enumerate}
   \item
    Lorsque le numéro de routage du compte n\textquotesingle est pas disponible, les dépôts directs doivent être effectués à l\textquotesingle aide de la fonction CMO.
  \end{enumerate}
 \item
  Tous les chèques doivent être émis par le vice-président aux finances de la CFES à partir du compte chèques principal de la CFES
\end{enumerate}

\subsubsection{Frais et limites}\label{frais-et-limites}

\begin{enumerate}
 \item
  L\textquotesingle activité est responsable de tous les frais bancaires engagés sur le compte de l\textquotesingle activité.
 \item
  L\textquotesingle activité est responsable de tous les frais liés aux virements électroniques, aux paiements par carte de crédit ou à d\textquotesingle autres frais de service de transaction, qu\textquotesingle ils soient entrants ou sortants.
 \item
  La CFES approvisionnera le compte de l\textquotesingle activité avec le solde minimum, tel que défini par la CIBC, afin d\textquotesingle éviter les frais mensuels.

  \begin{enumerate}
   \item
    Si le solde du compte de l\textquotesingle activité tombe en dessous du solde minimum, l\textquotesingle activité doit réapprovisionner ce montant et couvrir les frais bancaires mensuels encourus.
  \end{enumerate}
 \item
  Le compte d\textquotesingle activité dispose de 30 transactions gratuites par mois, y compris les transactions Interac e-Transfer, conformément aux offres de la CIBC.

  \begin{enumerate}
   \item
    L\textquotesingle activité sera responsable de tous les frais bancaires encourus si le seuil de 30 transactions gratuites est dépassé.
   \item
    Le compte d\textquotesingle activité aura une limite de 25 000 \$ par transaction et de 300 000 \$ par jour pour la fonction Interac e-Transfer® pour les entreprises.
  \end{enumerate}
\end{enumerate}
