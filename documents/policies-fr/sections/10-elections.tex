\section{Les élections}\label{les-uxe9lections}

\subsection{Les élections annuelles du CCLI}\label{les-uxe9lections-annuelles-du-ccli}

\begin{enumerate}
 \item
  Le/la président.e, le/la VPA, le/la VPS, le/la VPFA ainsi que deux conseiller.ère.s nationaux sont élus dans le cadre du CCLI en janvier de chaque année ; ils entrent en fonction le 1er avril.
 \item
  Les écoles hôtes pour les activités se font également élire à ce moment-là.

  \begin{enumerate}
   \item
    La période de présentation des demandes des écoles candidates à l'organisation du CCLI est de 24 mois.
   \item
    La période de présentation des demandes des écoles candidates à l'organisation de la CCI est de 26 mois.
   \item
    La période de présentation des demandes des écoles candidates à l'organisation du CDI est de 22 mois.
   \item
    La période de présentation des demandes des écoles candidates à l'organisation du CDDI est ouvert 25 mois à l\textquotesingle avance ;
   \item
    La période de présentation des demandes des écoles candidates à l'organisation du Sommet du développement des associations en ingénierie est de 21 mois.
  \end{enumerate}
\end{enumerate}

\subsubsection{Invitations aux appels d\textquotesingle offres pour des activités}\label{invitations-aux-appels-doffres-pour-des-activituxe9s}

\begin{enumerate}
 \item
  Toute société membre peut choisir de se porter candidate à une activité de la FCÉG après en avoir informé le/la VPS et le/la président du CA une semaine avant la tenue du SCIP, du SDAI ou du CCLI
 \item
  Aucune invitation de la part du/de la VPS n\textquotesingle est nécessaire pour exprimer son intérêt pour une activité de la FCÉG.
 \item
  Trois écoles membres seront invitées par le/la VPS par courriel à présenter une offre ou une contre-offre pour un nombre choisi de congrès deux mois avant une AG organisée par la FCÉG (les SDAI, CCLI et SCIP).

  \begin{enumerate}
   \item
    Une candidature est une présentation de 10 minutes expliquant pourquoi l\textquotesingle école membre devrait accueillir cette activité.
   \item
    Une anti-candidature est une présentation plus courte de 5 minutes sur les raisons pour lesquelles l\textquotesingle école membre ne peut pas accueillir une activité.

    \begin{enumerate}
     \item
      Il peut s\textquotesingle agir de la preuve d\textquotesingle un manque de soutien de la part du corps enseignant, de la société d\textquotesingle ingénierie, d\textquotesingle un manque d\textquotesingle engagement des étudiants ou d\textquotesingle un manque de ressources financières.
    \end{enumerate}
   \item
    Le/la VPS met à la disposition de tous les candidats une base de données sur les ressources en matière de candidature.
  \end{enumerate}
 \item
  Les trois invitations sont envoyées sur la base d\textquotesingle un calendrier d\textquotesingle appels d\textquotesingle offres interne.
 \item
  Les écoles membres qui n\textquotesingle ont pas organisé d\textquotesingle événement depuis 2020 sont placées en début de file d\textquotesingle attente et les écoles membres qui ont organisé des événements récemment sont placées en fin de file d\textquotesingle attente.

  \begin{enumerate}
   \item
    Seules les écoles membres proposant des programmes d\textquotesingle ingénierie en 4 ans peuvent être invitées à présenter une offre ou une anti-offre selon le calendrier des appels d\textquotesingle offres.
  \end{enumerate}
\end{enumerate}

\subsection{Le/la directeur.e général.e des élections}\label{lela-directeur.e-guxe9nuxe9ral.e-des-uxe9lections}

\begin{enumerate}
 \item
  Le/la directeur.e général.e des élections est responsable de la tenue de l'élection annuelle dans le cadre du CCLI.
 \item
  Habituellement, c'est le/la président.e du Conseil d\textquotesingle administration qui remplit cette fonction, mais le Conseil d\textquotesingle administration peut décider de nommer une autre personne par une résolution adoptée à la majorité des deux tiers (2/3), notamment dans le cas où le/la président.e actuel.le du Conseil d\textquotesingle administration se présente à l'élection.
 \item
  Le/la directeur.e général.e des élections est chargé.e de résoudre les conflits ou les ambiguïtés et de définir la méthode électorale.
 \item
  Toutes ses décisions sont définitives.
\end{enumerate}

\subsection{La mise en candidature}\label{la-mise-en-candidature}

\begin{enumerate}
 \item
  La période de mise en candidature consistera en deux séances qui auront lieu pendant des journées qui se suivent, selon ce qu'auront décidé le directeur général des élections et le/la président.e du CCLI.
 \item
  Les tours de mise en candidature consistent, pour le/la directeur.e général.e des élections, à demander aux membres de proposer des candidat.e.s à chacun des postes.
 \item
  Tout.e étudiant.e inscrit dans une école membre peut proposer une candidature.
 \item
  Au premier tour, une personne désignée peut accepter, refuser ou reporter sa mise en candidature.
 \item
  Au second tour, après que l'appel de candidatures est terminé, toute personne qui avait reporté sa mise en candidature doit l'accepter ou la refuser.
\end{enumerate}

\subsection{Les documents de campagne}\label{les-documents-de-campagne}

\begin{enumerate}
 \item
  Tous les candidat.e.s seront autorisés à produire un document dévoilant leur plateforme, lequel sera distribué aux membres.
 \item
  Le format, la longueur et la date de dépôt de ce document facultatif seront déterminés par le/la directeur.trice général.e des élections et annoncés tout au moins avant la première période de mise en candidature.
 \item
  Le document sera traduit en anglais et en français dès que possible, mais pourra être distribué avant la traduction.
 \item
  Aucun autre document de campagne ne sera recueilli ni distribué par la FCÉG; le/la candidat.e en aura la responsabilité exclusive.
\end{enumerate}

\subsection{Les discours}\label{les-discours}

\begin{enumerate}
 \item
  Tous les candidat.e.s seront autorisé.e.s à présenter aux membres une allocution de deux à cinq minutes, selon ce qu'aura déterminé le/la directeur.trice général.e des élections.
 \item
  Tous les candidat.e.s d'une école hôte seront autorisé.e.s à présenter aux membres un diaporama d'un maximum de dix minutes.
 \item
  Le/la directeur.trice général.e des élections doit déterminer l'ordre dans lequel les candidats prendront la parole.
 \item
  Les allocutions ont lieu entre la seconde période de mise en candidature et la séance de vote de l'assemblée générale; le moment exact sera déterminé par le directeur général des élections et le/la président.e du CCLI.
 \item
  Les discours prendront fin au moins 12 heures avant le scrutin.
 \item
  Une traduction simultanée doit être disponible durant les discours, au minimum à la tous les membres votants.
 \item
  Si la traduction se fait par l'entremise d'une compagnie externe, le service de traduction sera fourni à un prix au comité organisateur de l'activité durant laquelle les élections ont lieu.
\end{enumerate}

\subsection{La période de questions}\label{la-puxe9riode-de-questions}

\begin{enumerate}
 \item
  Une période de questions sera organisée après que tous les candidat.e.s à un poste ou à un événement donné se seront exprimés, sous la médiation du/de la directeur.trice général.e.
 \item
  Chaque question doit être posée dans les deux langues officielles, en alternant la langue posée en premier.
 \item
  Chaque candidat.e aura l'occasion de répondre à toutes les questions dans un ordre déterminé par le/la directeur.trice général.e des élections.
 \item
  Les candidats doivent répondre aux questions suivantes:

  \begin{enumerate}
   \item
    Leur niveau de compréhension des deux langues officielles; et,
   \item
    Leur plan de communication avec les délégué.e.s dont la première langue ne correspond pas à la leur.
  \end{enumerate}
 \item
  Une traduction simultanée doit être disponible durant les périodes de questions, au minimum à la tous les membres votants. La traduction peut se faire à l'interne où par l'entremise d'une compagnie externe.
 \item
  Dans ce deuxième cas, le service de traduction sera fourni à un prix au comité organisateur de l'activité durant laquelle les élections ont lieu.
\end{enumerate}

\subsection{Le scrutin}\label{le-scrutin}

\subsubsection{La procédure}\label{la-procuxe9dure}

\begin{enumerate}
 \item
  L'élection doit se faire au scrutin secret.
 \item
  Pour être élu, un.e candidat.e doit obtenir la voix de la majorité absolue des membres votants (50 \% +1), exception faite des abstentions.
 \item
  Chaque membre déposera un bulletin de vote comportant une liste de candidat.e.s à classer.
 \item
  Le membre pourra classer autant de candidat.e.s qu'il le désire selon son ordre de préférence.

  \begin{enumerate}
   \item
    Pour les postes du/de la président.e, du/de la VPA, du/de la VPS et du/de la VPFA, le membre votant pourra indiquer un seul premier choix.
   \item
    Si aucun.e candidat.e n'obtient la majorité absolue des votes de premier choix, la méthode de scrutin par élimination uninominal entrera en vigueur.
  \end{enumerate}
 \item
  Pour les postes de conseiller.ère national.e

  \begin{enumerate}
   \item
    Il sera possible d'indiquer jusqu'à deux premiers choix.
   \item
    On pourra classer un.e candidat.e à tous les rangs autres que le premier.
   \item
    Si moins de deux candidat.e.s obtiennent la majorité absolue des votes de premier choix, la méthode de scrutin par élimination binominal entrera en vigueur.
  \end{enumerate}
 \item
  Un membre de l'exécutif national en fonction, sauf s'il se présente à l'élection, déposera un bulletin comportant une liste de candidat.e.s à classer.

  \begin{enumerate}
   \item
    Il doit assigner un rang à chacun.e des candidat.e.s.
   \item
    Ce bulletin de vote ne sera consulté qu'en cas d'égalité, comme il est indiqué ci-dessous.
  \end{enumerate}
 \item
  C'est le/la directeur.trice général.e des élections qui comptera les votes, sous la supervision d'un.e scrutateur.trice, lequel est nommé par l'assemblée générale.
\end{enumerate}

\subsubsection{\texorpdfstring{La méthode de scrutin par élimination uninominal }{La méthode de scrutin par élimination uninominal }}\label{la-muxe9thode-de-scrutin-par-uxe9limination-uninominal}

\begin{enumerate}
 \item
  Le nombre de premiers choix pour chaque candidat.e est compté.
 \item
  Si un.e candidat.e reçoit plus de 50\% du vote, il/elle est déclaré vainqueur et l'élection est terminée.
 \item
  Dans le cas où aucun.e candidat.e ne reçoit plus de 50\% du vote, celui ayant reçu le moins de votes est éliminé.
 \item
  Si plus d'un.e candidat.e reçoit le même nombre de votes et que ce nombre représente le moins de votes pour un.e candidat.e, tous les candidat.e.s à égalité sont éliminés, sauf si, ce faisant, il ne reste aucun.e candidat.e éligible.

  \begin{enumerate}
   \item
    Dans ce cas, le/la candidat.e éliminé.e sera celui dont le rang est le plus bas sur le bulletin de vote du membre de l'exécutif national.
   \item
    Un.e seul.e candidat.e est éliminé.e dans ce cas.
  \end{enumerate}
 \item
  Sur tous les bulletins de vote où le/la candidat.e éliminé.e est classé.e premier.ère, le/la candidat.e classé.e deuxième est traité.e comme ayant été classé.e premier.ère, à moins que le/la candidat.e du deuxième choix ait déjà été éliminé.e.

  \begin{enumerate}
   \item
    Dans ce cas, le/la candidat.e classé.e au rang suivant deviendra premier.ère, et ainsi de suite.
  \end{enumerate}
 \item
  Le nombre de candidat.e.s de premier choix est recompté.e.s.
 \item
  S'il ne reste qu'un.e seul.e candidat.e et que ce candidat.e n'obtient pas plus de 50 \% des voix, le vote est déclaré nul en raison de l'incertitude et on reprend l'élection.
 \item
  Les étapes 2 à 6 sont répétées.
\end{enumerate}

\subsubsection{La méthode de scrutin par élimination binominal}\label{la-muxe9thode-de-scrutin-par-uxe9limination-binominal}

\begin{enumerate}
 \item
  Le nombre de premiers choix pour chaque candidat.e est compté.
 \item
  Si deux candidat.e.s ou plus ont obtenu plus de 50 \% des voix, les deux candidat.e.s ayant obtenu le plus grand nombre de voix sont déclaré.e.s gagnant.e.s et l'élection est terminée.
 \item
  Dans le cas où aucun.e. candidat.e ne reçoit plus de 50\% du vote, celui/celle ayant reçu le moins de votes est éliminé.e.
 \item
  Si plus d'un.e candidat.e reçoit le même nombre de votes et que ce nombre représente le moins de votes pour un.e candidat.e, tous les candidat.e.s à égalité sont éliminé.e.s, sauf si, ce faisant, il ne reste aucun.e candidat.e éligible.

  \begin{enumerate}
   \item
    Dans ce cas, le/la candidat.e éliminé.e sera celui dont le rang est le plus bas sur le bulletin de vote du membre de l\textquotesingle exécutif national
   \item
    Un.e seul.e candidat.e est éliminé.e dans ce cas.
  \end{enumerate}
 \item
  Sur tous les bulletins de vote où le/la candidat.e éliminé.e est classé.e premier.ère, le/la candidat.e classé.e deuxième est traité.e comme ayant été classé.e premier.ère, à moins que le/la candidat.e du deuxième choix ait déjà été éliminé.e.

  \begin{enumerate}
   \item
    Dans ce cas, le/la candidat.e classé.e au rang suivant deviendra premier.ère, et ainsi de suite.
  \end{enumerate}
 \item
  Le nombre de candidat.e.s de premier choix est recompté.
 \item
  Si seulement un.e candidat.e obtient plus de 50 \% des voix alors qu'il reste deux candidat.e.s, cette personne est élue et l'autre poste de conseiller.ère national.e est rouvert.e.
 \item
  Si aucun des candidat.e.s n'obtient plus de 50 \% des voix, alors qu'il reste deux candidat.e.s, personne n'est élue et on reprend l'élection.
 \item
  Les étapes 2 à 6 sont répétées.
\end{enumerate}

\subsection{La ratification}\label{la-ratification}

\begin{enumerate}
 \item
  Les candidat.e.s élu.e.s seront présenté.e.s à l'assemblée générale pour être ratifié.e.s par les membres votants.
\end{enumerate}

\subsection{Les élections partielles}\label{les-uxe9lections-partielles}

\subsubsection{Les élections partielles pour un poste de membre de l'exécutif national ou de conseiller.ère national.e}\label{les-uxe9lections-partielles-pour-un-poste-de-membre-de-lexuxe9cutif-national-ou-de-conseiller.uxe8re-national.e}

\begin{enumerate}
 \item
  Si un poste reste vacant après l'élection annuelle ou qu'il est libéré pendant l'exercice, le/la président.e du Conseil d\textquotesingle administration agit à titre de directeur.trice général.e des élections.
 \item
  L'élection a lieu pendant une réunion du Conseil d\textquotesingle administration ou d\textquotesingle une réunion ultérieure des membres; un mécanisme électoral supplémentaire est mis en place à la discrétion du/de la président.e du Conseil d\textquotesingle administration en sa qualité de directeur.trice général.e des élections.
\end{enumerate}

\subsubsection{Les élections partielles pour les écoles hôtes}\label{les-uxe9lections-partielles-pour-les-uxe9coles-huxf4tes}

\paragraph{La vue d\textquotesingle ensemble}\label{la-vue-densemble-1}

\begin{enumerate}
 \item
  Si un poste reste vacant après l'élection annuelle, le/la président.e du Conseil d\textquotesingle administration agit à titre de directeur.trice général.e des élections.
 \item
  L'élection a lieu pendant une réunion du Conseil d\textquotesingle administration ou une réunion ultérieure des membres; un mécanisme électoral supplémentaire est mis en place à la discrétion du/de la président.e du Conseil d\textquotesingle administration en sa qualité de directeur.trice général.e des élections.
\end{enumerate}

\paragraph{Les postes vacants dans les écoles hôtes}\label{les-postes-vacants-dans-les-uxe9coles-huxf4tes}

\begin{enumerate}
 \item
  Dans le cas où il n'y a pas d'école hôte pour le SDAI ou le CCLI qui a été élue dans le cadre de la réunion de printemps de l'année fiscale précédente, l'exécutif national entrant devra assumer le rôle du/de la président.e de l'activité et en faire l'organisation.

  \begin{enumerate}
   \item
    Le lieu de l'activité doit être déterminé par un vote majoritaire du Conseil d\textquotesingle administration; il sera choisi de façon à réduire les coûts pour les écoles membres.
   \item
    L'exécutif national peut recruter les étudiant.e.s qui le souhaitent pour agir à titre de responsables d'activité .
   \item
    Les deux (2) événements seront menés à bien de façon à réduire les coûts; il sera possible d'accepter seulement un.e représentant.e par école membre et certains officiers pour qui il serait important d'assister à l'activité.
  \end{enumerate}
 \item
  Dans le cas où il n'y pas d'école hôte pour le Congrès sur la diversité en ingénierie, le Congrès sur le développement durable en ingénierie ou la Compétition canadienne d\textquotesingle ingénierie n'a pas été élue dans le cadre de la réunion de printemps de l'année fiscale précédente, l'activité ne sera pas tenue et les membres en seront avisés au plus tard le 1er mai.
\end{enumerate}

