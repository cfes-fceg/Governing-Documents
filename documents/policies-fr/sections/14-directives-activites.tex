\section{Les directives sur les activités}\label{les-directives-sur-les-activituxe9s}

\subsection{L'objectif}\label{lobjectif-3}

\begin{enumerate}
 \item
  Pour mieux servir ses membres, il est essentiel que les activités de la FCÉG ont lieu dans des espaces sécuritaires qui encouragent la discussion et l'apprentissage.
 \item
  À cette fin, la FCÉG a comme politique de favoriser les points suivants:

  \begin{enumerate}
   \item
    La mise en place d'un code de conduite applicable aux activités
   \item
    La conception d'un système de notification des incidents, faisant intervenir le comité d'intervention d'incidents dans le cadre de chaque activité
   \item
    La tenue d'une séance sur l'inclusivité dans le cadre du CCLI chaque année
   \item
    La communication d'information sur la planification d'activités inclusives aux responsables des activités au moment de la ratification
   \item
    Inclusion dans les paquets d'informations pré-délégués sur les éléments d\textquotesingle accessibilité disponibles sur le.s site.s principal.aux de toute activité.
   \item
    Une séance d\textquotesingle information sur les attentes de l\textquotesingle activité pour tous les délégué.e.s avant l\textquotesingle activité.
  \end{enumerate}
\end{enumerate}

\subsection{\texorpdfstring{Le code de conduite pour les activités }{Le code de conduite pour les activités }}\label{le-code-de-conduite-pour-les-activituxe9s}

\begin{enumerate}
 \item
  Le code de conduite pour toute activité de la FCÉG doit être distribué au moins un mois avant chaque activité et sera comme suit:
\end{enumerate}

\emph{La Fédération canadienne des étudiant.e.s en génie (FCÉG) vise à mettre en valeur des milieux inclusifs et respectueux où tous les étudiant.e.s se sentent en sécurité, accueilli.e.s chaleureusement et valorisé.e.s. À cette fin, la FCÉG s'attend à ce que les étudiant.e.s se comportent d'une façon inclusive et respectueuse, digne d'un.e futur.e ingénieur.e.}

\emph{En assistant à} une activité \emph{de la FCÉG, vous vous engagez à :}

\begin{itemize}
 \item
  \emph{Accorder de l'importance aux expériences des autres ;}
 \item
  \emph{Incarner un esprit de respect et de fraternité ;}
 \item
  \emph{Entretenir une communication respectueuse indépendamment des désaccords ;}
 \item
  \emph{Garder l'esprit ouvert aux idées et aux informations nouvelles ;}
 \item
  \emph{Vous abstenir de vous comporter d'une manière dégradante, insultante ou nuisible pour les autres ;}
 \item
  \emph{Soutenir vos pairs ;}
 \item
  \emph{Apprendre et à vous développer sur les plans professionnel, académique et social ;}
 \item
  \emph{Respecter la vie privée des autres et observer les convenances de l'hôtel.}
\end{itemize}

\emph{La FCÉG s'appuie sur les efforts, le dévouement et l'engagement de l'ensemble de ses étudiant.e.s. En vous conformant au présent Code de conduite, vous contribuez activement à bâtir une FCÉG meilleure et à amener un climat constructif dans la Fédération.}

\emph{Je, (nom) \_\_\_\_\_\_\_\_\_\_\_\_\_\_\_\_\_\_\_\_\_\_\_\_\_\_\_\_\_\_\_\_\_\_\_\_ de (école) \_\_\_\_\_\_\_\_\_\_\_\_\_\_\_\_\_\_\_\_\_\_\_\_\_\_\_\_ ai lu le Code de conduite pour les} activité \emph{de la FCÉG. Je comprends le Code et, de plus, je me rends compte de son importance pour mon expérience personnelle et celle des autres. J'accepte de me conformer au Code de conduite pour le (}activité\emph{) \_\_\_\_\_\_\_\_\_\_\_\_\_\_\_\_\_\_\_\_\_\_\_\_\_\_\_\_\_\_\_, en plus d'encourager et d'aider les autres à faire de même.}

\emph{Délégué : Signé le \_\_\_\_\_\_\_\_\_\_\_ devant le/la chef.fe de ma délégation \_\_\_\_\_\_\_\_\_\_\_\_\_\_ à titre de témoin.}

\emph{Chef.fe de délégation ou officier de la FCÉG :}

\emph{Signé le \_\_\_\_\_\_\_\_\_\_\_\_ devant l'officier de la FCÉG \_\_\_\_\_\_\_\_\_\_\_\_ à titre de témoin.}

\emph{Si vous êtes témoin d'une infraction au Code de conduite, vous devez les signaler par le truchement du Système de notification des incidents. Le non-respect du Code pourra entraîner des conséquences allant de la réprimande verbale à l'expulsion des événements actuels et futurs de la FCÉG. Nous pourrons également communiquer avec l'établissement d'enseignement du participant.}

\subsection{L'information sur l'accessibilité}\label{linformation-sur-laccessibilituxe9}

\subsubsection{Le contenu du dossier des pré-délégués}\label{le-contenu-du-dossier-des-pruxe9-duxe9luxe9guuxe9s}

\begin{enumerate}
 \item
  Afin de s'assurer que tous les délégué.e.s soient au courant de tout obstacle à leur participation à un activité, le dossier d'informations pour les délégué.e.s doit contenir les informations suivantes:

  \begin{enumerate}
   \item
    La présence d'aménagements et d'obstacles accessibles aux personnes à mobilité réduite
   \item
    Toute restriction concernant les animaux d'assistance pouvant exister sur le.s lieu.x principal.aux de l'activité
   \item
    La disponibilité ou le manque de disponibilité d'aliments répondant aux exigences des restrictions diététiques les plus courantes, y compris les options végétariennes, végétaliennes, sans gluten, sans lactose, halal et casher
   \item
    La disponibilité ou l\textquotesingle absence de ressources de traduction telles que définies dans la section 13.4.
   \item
    Une liste des mots courants et des abréviations officielles avec leur signification lors des activités.
   \item
    Toute autre information importante déterminées par le responsable d'activité concerné ou le/la VPS.
  \end{enumerate}
\end{enumerate}

\subsubsection{La responsabilités pendant l\textquotesingle activité}\label{la-responsabilituxe9s-pendant-lactivituxe9}

\begin{enumerate}
 \item
  Le/la VPS et le comité organisateur doivent se réunir au moins trois fois avant l'activité afin de s\textquotesingle assurer qu'à la section 13.3 est respectée.

  \begin{enumerate}
   \item
    La première réunion doit avoir lieu au moins 6 mois avant l'activité.
   \item
    La deuxième réunion doit avoir lieu au moins un mois avant l\textquotesingle envoi du dossier d'informations pour les délégué.e.s.
   \item
    La troisième réunion doit avoir lieu au moins 3 semaines avant le début de l\textquotesingle activité.
  \end{enumerate}
 \item
  Le/la VPS et le comité organisateur de l'activité concernée doivent se réunir après l'activité pour déterminer si les attentes en matière d\textquotesingle accessibilité ont été satisfaites.
 \item
  Le Conseil d\textquotesingle administration attribue un statut d\textquotesingle avertissement à l\textquotesingle hôte de l'activité examinée s\textquotesingle il manque de professionnalisme et s\textquotesingle il s\textquotesingle agit de sa première activité.
 \item
  Le statut d\textquotesingle avertissement doit être retiré si le même hôte organise une activité réussie.
 \item
  Lorsque l\textquotesingle hôte averti planifie une autre activité avec le statut d\textquotesingle avertissement et que la activité pose toujours problème, le Conseil d\textquotesingle administration doit placer l\textquotesingle école membre en mauvaise posture.
 \item
  Si le rapport fait état d\textquotesingle un manque de planification de la part du comité organisateur de l\textquotesingle activité concernée, les membres peuvent charger le Conseil d\textquotesingle administration de mener une enquête.
 \item
  Le Conseil d\textquotesingle administration doit partager les conclusions et les résultats de l\textquotesingle enquête avec les écoles membres dans un délai d\textquotesingle un mois à compter de la notification des résultats à l\textquotesingle école membre ayant fait l\textquotesingle objet de l\textquotesingle enquête.
\end{enumerate}

\subsection{Les directives sur les services de traduction}\label{les-directives-sur-les-services-de-traduction}

\begin{enumerate}
 \item
  Les éléments suivants sont des ressources qui peuvent être disponibles lors des activités:

  \begin{enumerate}
   \item
    Des PowerPoints, panneaux et documents bilingues réalisés pour l'activité
   \item
    Des présentateurs bilingues
   \item
    La traduction simultanée: sous-titres, volontaires/délégué.e.s bilingues avec un.e présentateur.e unilingue
   \item
    La traduction du script de présentation
   \item
    La traduction du procès-verbal de la session
   \item
    Des cartes avec des mots courants sont distribuées aux délégué.e.s
   \item
    Fournir des ressources ou une formation immersive aux membres/agents clés : enseigner le vocabulaire de base, les mots pertinents et les termes clés.
   \item
    Un.e ou des délégué.e.s au bilinguisme facilement identifiable
  \end{enumerate}
\end{enumerate}

