\section{Elections}\label{elections}

\subsection{Annual Elections at CELC}\label{annual-elections-at-celc}

\begin{enumerate}
 \item
  The President, Vice President Academic, Vice President Services, Vice President Finance \& Administration, and two National Councilors are elected at CELC in January of each year to take office on April 1st.
 \item
  Bidding for host Schools is also elected at this time.

  \begin{enumerate}
   \item
    The bidding for the CELC host school is opened 24 months in advance;
   \item
    The bidding for the CEC host school is opened 26 months in advance;
   \item
    The bidding for the CDE host school is opened 22 months in advance;
   \item
    The bidding for the CSE host school is opened 25 months in advance;
   \item
    The bidding for the Summit on Development of Engineering Societies host school is open 21 months in advance.
  \end{enumerate}
\end{enumerate}

\subsubsection{Activity Bid Invitations}\label{activity-bid-invitations}

\begin{enumerate}
 \item
  Any Member Society can choose to bid for a CFES Activity upon notice given to the VPS and Board Chair one week prior to CESS, SDES, or CELC.
 \item
  No invitation from the VPS is needed to express interest in bidding for a CFES activity.
 \item
  There will be 3 member schools invited by the VPS via email to bid or anti-bid for a select number of conferences 2 months prior to a GA hosted by the CFES (SDES, CELC, and CESS).

  \begin{enumerate}
   \item
    A bid is a 10 min presentation of why the member school should host this activity.
   \item
    An anti-bid is a shorter 5 min presentation on why the member school cannot host an activity.

    \begin{enumerate}
     \item
      This can include proof of lack of support from faculty, from engineering society, lack of student engagement, or lack of financial resources.
    \end{enumerate}
   \item
    A bidding resources database is provided to all candidates by the VPS.
  \end{enumerate}
 \item
  The 3 invitations are sent based on an internal bidding schedule.
 \item
  This includes member schools that have not hosted since 2020 at the start of the queue, and member schools that have hosted recently in the back of the queue.

  \begin{enumerate}
   \item
    Only member schools with 4-year engineering programs can be invited to bid/anti-bid based on the bidding schedule.
  \end{enumerate}
\end{enumerate}

\subsection{Chief Returning Officer}\label{chief-returning-officer}

\begin{enumerate}
 \item
  The Chief Returning Officer (CRO) is responsible for the conducting of the annual elections at CELC.
 \item
  The CRO will normally be the Board Chair, but the Board may resolve to appoint another individual by two-thirds (2/3) resolution, including in the event the current Chair of the Board stands for election.
 \item
  The CRO is responsible for the resolution of any conflicts or ambiguities, and interpretations of the electoral procedures.
 \item
  All decisions of the CRO are final.
\end{enumerate}

\subsection{Nominations}\label{nominations}

\begin{enumerate}
 \item
  The nomination period will consist of two nomination sessions on consecutive days as decided between the CRO and CELC Chair.
 \item
  Nomination rounds consist of the CRO going through each of the positions and asking for nominations from the membership.
 \item
  Any student who is enrolled at a member school may nominate someone for a position.
 \item
  During the first nomination round, a candidate may either accept, decline, or defer the nomination.
 \item
  During the second nomination round, after the call has been completed, anyone who deferred their nomination must either accept or decline.
\end{enumerate}

\subsection{Campaign Material}\label{campaign-material}

\begin{enumerate}
 \item
  All candidates for individual positions will be allowed to produce a document outlining their platforms which will be distributed to the members.
 \item
  The format, length, and timeline for the optional document shall be determined by the CRO and announced prior to the first nomination period, at a minimum.
 \item
  The document will be translated in both English and French as soon as possible, but may be distributed prior to translation.
 \item
  No other campaign material will be collected or distributed from the CFES and shall be the sole responsibility of the candidate.
\end{enumerate}

\subsection{Speeches}\label{speeches}

\begin{enumerate}
 \item
  All candidates for individual positions will be allowed to present a speech to the members of a length determined by the CRO, no less than 2 minutes, and no more than 5 minutes.
 \item
  All candidates for host schools will be allowed to present a slideshow to the members for up to 10 minutes.
 \item
  The CRO shall determine the order in which candidates speak.
 \item
  Speeches will take place between after the second nomination period and the General Assembly voting session; the exact time will be determined by the CRO and CELC Chair.
 \item
  Speeches will end at least twelve four hours before voting.
 \item
  During speeches, simultaneous translation services must be available, at minimum to all members, whether they be professionally outsourced or member-led.
 \item
  If outsourced, these will be provided at a cost to the organizing committee at which the election takes place.
\end{enumerate}

\subsection{Question Period}\label{question-period}

\begin{enumerate}
 \item
  A question period will be held after all candidates for a given position or event have spoken and will be mediated by the CRO.
 \item
  Each question must be asked in both official languages, alternating which language is asked first.
 \item
  Each candidate will be given the opportunity to answer any and all questions in an order determined by the CRO.
 \item
  All candidates must be asked:

  \begin{enumerate}
   \item
    their level of comprehension of both official languages; and,
   \item
    their plan to communicate with delegates whose first language does not match their own.
  \end{enumerate}

 \item
  During question period, simultaneous translation services must be available, at minimum to all members, whether they be professionally outsourced or member-led.
 \item
  If outsourced, these will be provided at a cost to the organizing committee at which the election takes place.
\end{enumerate}

\subsection{Voting}\label{voting}

\subsubsection{Procedure}\label{procedure}

\begin{enumerate}
 \item
  The election must be held by means of a secret ballot.
 \item
  To be elected, a candidate must receive the absolute majority of the members voting (50\% +1), abstentions excluded.
 \item
  Each Member shall submit a ballot comprised of a ranked list of candidates.
 \item
  The member is able to rank as many or as few of the candidates as desired.

  \begin{enumerate}
   \item
    For the President, Vice President Academic, Vice President Services, and Vice President Finance \& Administration positions: one candidate may be ranked as a first choice.
   \item
    Should less than one candidate receive the majority of first-choice votes outright, the single candidate runoff procedure shall be enacted.
  \end{enumerate}

 \item
  For the National Councillor positions:

  \begin{enumerate}
   \item
    up to two distinct candidates may be ranked equally as first choices.
   \item
    One candidate may be ranked for each rank other than first.
   \item
    Should fewer than two candidates receive majorities of first-choice votes outright, the dual candidate runoff procedure shall be enacted.
  \end{enumerate}

 \item
  The current National Executive, excluding any National Executive member who is standing for election, shall submit a ballot comprised of a ranked list of candidates.

  \begin{enumerate}
   \item
    The National Executive must rank all candidates, each receiving a unique rank
   \item
    This ballot will only be viewed in the event of a tie, as outlined below.
  \end{enumerate}

 \item
  Votes will be counted by the CRO, under the supervision of a scrutineer, who is appointed by the General Assembly.
\end{enumerate}

\subsubsection{Single Candidate Runoff
 Procedure}\label{single-candidate-runoff-procedure}

\begin{enumerate}
 \item
  The number of first‐choice votes for each candidate is counted.
 \item
  If one candidate has received more than 50\% of the vote, this candidate is declared the winner, and the procedure is complete.
 \item
  In the event that no candidate has received more than 50\% of the vote, the candidate who received the lowest number of first choice votes is dropped from the running.
 \item
  In the event that two or more candidates are tied in receiving the lowest number of first-place votes, all tied candidates are dropped from the running, except in the case when this would reduce the remaining eligible candidates to zero.

  \begin{enumerate}
   \item
    In such a case, the candidate to be dropped will be the candidate with the lower rank on the National Executive ballot.
   \item
    Only one candidate will be dropped in this case.
  \end{enumerate}

 \item
  On all ballots with the dropped candidate ranked as a first choice, the candidate ranked second on the ballot is treated as having been ranked first, unless the second choice candidate has already been dropped.

  \begin{enumerate}
   \item
    In this case, the next choice will be counted as first, etc.
  \end{enumerate}

 \item
  The number of first choice votes for each candidate is counted again.
 \item
  If there is only one candidate remaining and this candidate does not have more than 50\% of the vote, the vote is declared void due to no confidence, and the elections are held again.
 \item
  Repeat steps 2--6 until one winner is declared.
\end{enumerate}

\subsubsection{Dual Candidate Runoff
 Procedure}\label{dual-candidate-runoff-procedure}

\begin{enumerate}
 \item
  The number of first‐choice votes for each candidate is counted.
 \item
  If two or more candidates have each received more than 50\% of the vote, the top two candidates are declared the winners, and the procedure is complete.
 \item
  In the event that there is only one candidate with more than 50\% of the vote, or zero candidates with more than 50\% of the vote, the candidate who received the lowest number of first choice votes is dropped from the running.
 \item
  In the event that two or more candidates are tied in receiving the lowest number of first-place votes, all tied candidates are dropped from the running, except in the case when this would reduce the remaining eligible candidates to one.

  \begin{enumerate}
   \item
    In such a case, the candidate to be dropped will be the candidate with the lower rank on the National Executive ballot.
   \item
    Only one candidate will be dropped.
  \end{enumerate}

 \item
  On all ballots with the dropped candidate ranked as a first choice, the candidate ranked second on the ballot is treated as having been ranked first, unless the second choice candidate has already been dropped.

  \begin{enumerate}
   \item
    In this case, the next choice will be counted as first, etc.
  \end{enumerate}

 \item
  The number of first choice votes for each candidate is counted again.
 \item
  If only one candidate receives more than 50\% of the vote when there are two candidates remaining, that person is elected, and the remaining National Councillor position is reopened. If no candidates receive more than 50\% of the vote when there are two candidates remaining, no one is elected, and the elections are held again.
 \item
  Repeat steps 2--6 until two winners are declared.
\end{enumerate}

\subsection{Ratification}\label{ratification}

\begin{enumerate}
 \item
  The elected candidates will be put forth to the General Assembly for ratification by the members.
\end{enumerate}

\subsection{By-Elections}\label{by-elections}

\subsubsection{By-Elections for National Executive and National Councillor Positions}\label{by-elections-for-national-executive-and-national-councillor-positions}

\begin{enumerate}
 \item
  Should a position be vacant after the Annual Elections, or vacated during the year, the Board Chair acts as the CRO.
 \item
  Elections are held during a meeting of the Board or subsequent meetings of the members and additional election procedures are established at the discretion of the Board Chair in their capacity as the CRO.
\end{enumerate}

\subsubsection{By-Elections for Host Schools}\label{by-elections-for-host-schools}

\paragraph{Overview}\label{overview-1}

\begin{enumerate}
 \item
  Should a position be vacant after the Annual Elections the Board Chair acts as the CRO.
 \item
  Elections are held during a meeting of the Board or subsequent meetings of the members and additional election procedures are established at the discretion of the Board Chair in their capacity as the CRO.
\end{enumerate}

\paragraph{Vacant Host School
 Positions}\label{vacant-host-school-positions}

\begin{enumerate}
 \item
  In the event that a host school for Summit on Development of Engineering Societies or CELC is not elected by the Spring Meeting of the prior fiscal year, the incoming National Executive will be required to take on the role of conference chair and host the conference.
 \item
  The location of the conference is to be determined by a majority vote of the Board and should be selected to reduce costs to member schools
 \item
  The National Executive may enlist the help of interested students to act as conference organizers
 \item
  Both events will be run as to minimize costs associated with the event and may be limited to only allow one representative from each member school and pertinent Board Members to attend
 \item
  In the event that a host school for the Conference on Diversity in Engineering, Conference on Sustainability in Engineering, or Canadian Engineering Competition is not elected by the Spring Meeting of the prior fiscal year, the Activities shall not run and members shall be notified by May 1.
\end{enumerate}

