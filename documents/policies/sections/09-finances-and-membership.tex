\section{Finances and Membership}\label{finances-and-membership}

\subsection{Membership Fees}\label{membership-fees}

\begin{enumerate}
 \item
  Starting in the 2021-2022 year, membership fees are due on the final day of CELC held during the previous fiscal year (Membership fees for 202X are due in Fiscal Year 202(x-1)).
 \item
  Invoices are issued on September 1st of the year prior to the membership fee due date.
 \item
  The following requirements are in effect regarding membership fees:

  \begin{enumerate}
   \item
    A penalty of 5\% per month, up to a total of 20\% of the total amount outstanding, applies if a school member is late with its fee.
   \item
    Paying the CFES membership fee maintains an engineering student society's active membership status, which includes:
    \begin{enumerate}
     \item
      Eligibility for participation in CFES activities and services
     \item
      Eligibility to bring forward motions at meetings of the General Assembly
     \item
      Voting rights at meetings of the General Assembly
    \end{enumerate}

   \item
    A late and outstanding membership fee will affect the membership status of the respective engineering student society within the Federation.
   \item
    The school must pay observer fees at CFES events, and will have no voting rights, until such time that membership fees, including all applicable penalties, are paid.
   \item
    An engineering student society that did not attend CFES Summit on Development of Engineering Societies or CELC, and did not pay its membership fee on time must be contacted by the relevant Regional Ambassador to assess its future as a member of CFES.
   \item
    A member shall be considered for expulsion in the event that it does not pay its CFES member fee for two consecutive years.
   \item
    The President will ensure that the society understands the upcoming membership proceedings and the potential change in status.
   \item
    Membership fees must be received by the CFES prior to the commencement of their regional engineering competition for members to be eligible to attend CEC.
  \end{enumerate}
\end{enumerate}

\subsection{Applications for Membership}\label{applications-for-membership}

\begin{enumerate}
 \item
  Applications for membership in the Federation may only be considered by the General Assembly.
 \item
  Further to the guidelines outlined in the Constitution, applications for membership should contain (but are not limited to) the following information:

  \begin{enumerate}
   \item
    Name of organization
   \item
    Purpose or objectives of the organization
   \item
    Current executive team and membership of governing board
   \item
    Date of establishment of the organization
   \item
    Number of students in the faculty (or equivalent)
   \item
    Proof that the applying organizations represents students in accredited engineering program(s), or program(s) in the process of accreditation, under the CEAB
   \item
    Proof of support from the engineering society (a passed motion in public meeting minutes)
  \end{enumerate}

 \item
  Any societies that attend a General Assembly session with the eligibility and intention of becoming a member at that event will be considered a delegate instead of an observer.
\end{enumerate}

\subsection{Budget}\label{budget}

\begin{enumerate}
 \item
  The CFES operates with an annual budget approved at CELC.
 \item
  A change to this approved budget by more than 25\% of a line item originally in excess of \$5,000 must be approved first by the Board and then ratified by the General Assembly.
\end{enumerate}

\subsection{Activity Surpluses and
 Deficits}\label{activity-surpluses-and-deficits}

\subsubsection{Activity Fund}\label{activity-fund}

\begin{enumerate}
 \item
  This section has been created in order to regulate the distribution of surplus, and paying off deficits, from CFES activities and services.
 \item
  The Activity Fund will be used to hold all surpluses and pay off all deficits in case of an emergency such as shortage of sponsorship.
 \item
  The Activity Fund should be grown to keep at least \$2,000 for each activity including Summit on Development of Engineering Societies, Conference on Diversity, Canadian Engineering Competition and Conference on Sustainability in Engineering and \$10,000 for Canadian Engineering Leadership Conference with an additional \$35,000 total to be used as the seed funds for all activities.
 \item
  The seed fund amounts will be allocated as follows:

  \begin{enumerate}
   \item
    Summit on Development of Engineering Societies: \$5,000
   \item
    Conference on Diversity in Engineering: \$7,000
   \item
    Canadian Engineering Leadership Conference: \$9,000
   \item
    Conference on Sustainability in Engineering: \$7,000
   \item
    Canadian Engineering Competition: \$7,000
  \end{enumerate}

 \item
  The total minimum value of the Activity Fund should be \$53,000.
 \item
  If, after its mandate is fulfilled, a CFES service or activity account has a surplus or deficit, the VPFA shall be responsible for ensuring the procedure for either reallocating the surplus or paying the deficit is followed as prescribed in the following section, and the corresponding Activity Agreement, within one month of that activity ending.
\end{enumerate}

\subsubsection{Use of Funds to Cover Activity Surpluses}\label{use-of-funds-to-cover-activity-surpluses}

\begin{enumerate}
 \item
  In the event of an Activity Surplus, the following steps will be followed:

  \begin{enumerate}
   \item
    Create the transition committee consisting of: VP Finance and Administration, Outgoing Activity Manager, and Incoming Activity Manager.
   \item
    Verify the presence of surplus or deficit
  \end{enumerate}
\end{enumerate}

\subsubsection{Use of Funds to Cover Activity Deficits}\label{use-of-funds-to-cover-activity-deficits}

\begin{enumerate}
 \item
  In the event of a surplus transfer, the surplus is transferred into the Activity Fund until the surplus is depleted.
 \item
  In the event of an activity running a deficit, the CFES will pay the outstanding deficit through the Activity Fund, and the member society who signed the Activity Agreement will no longer be in good standing with the CFES.
 \item
  The following steps will be followed such that the Engineering Society can regain Good Standing with the CFES:

  \begin{enumerate}
   \item
    Create the transition committee as designated by the Board
   \item
    Verify the presence of deficit
   \item
    The committee will review the request, which must include the following:

    \begin{enumerate}
     \item
      A letter from the requesting committee
     \item
      Amount requested and the urgency of the request
     \item
      Proposed and actual budget
     \item
      Up-to-date cash and cash equivalents
     \item
      Reasons for discrepancies and shortfall
     \item
      Contingency plans
     \item
      Consequences of being denied funding
     \item
      Proof that dean/director of the organizing committee is aware of the financial situation
     \item
      Proof they have exhausted funding from their Faculty
     \item
      Proof they have exhausted funding from their Student Union
     \item
      Proof they have exhausted funding from their Engineering Society
     \item
      Other supporting information as necessary

    \end{enumerate}
  \end{enumerate}
 \item
  Any additional funding obtained by the Organizing Committee in an effort to pay off the deficit will be paid to the Activity Fund.
 \item
  The Organizing Committee must obtain approval from the Board to solicit funding from sources not specified above in an effort to pay off the deficit.
 \item
  After the submitted documents have been reviewed by the committee, the Activity Manager must present the case for funding in real-time (i.e.~conference call, in person).
 \item
  After the presentation, the Activity Manager must answer all questions from the committee.
 \item
  The committee must be created maximum two weeks after acknowledgement by the CFES President.
 \item
  The decision of the committee must be presented to the Board at the next meeting following the decision.
 \item
  All documentation must be archived for purposes of precedence for future cases.

\end{enumerate}

\subsection{Use of Funds from the Activity
 Fund}\label{use-of-funds-from-the-activity-fund}

\subsubsection{Excess Funds}\label{excess-funds}

\begin{enumerate}
 \item
  Each year in the first week of January a proposal for the distribution of 5\% of excess funds from the activity fund may be presented to the Board by any member or Board Member.
 \item
  If a proposal is submitted the Board will create a committee to evaluate the available proposals and available funds which may only make a recommendation to the Board.
 \item
  The committee composition shall be determined by the Board.
 \item
  The avenues by which the accrued funds could be allocated are as follows:

  \begin{enumerate}
   \item
    The Activity Fund Donation for CFES Activities and Services
   \item
    Special Activity Fund for School Initiatives
   \item
    Member School funding Travel Fund
  \end{enumerate}

 \item
  All documentation must be archived for purposes of precedence for future cases.
\end{enumerate}

\subsubsection{Activity Fund Loans}\label{activity-fund-loans}

\begin{enumerate}
 \item
  In the event of an activity needing cash greater than their current cash account balance, the Activity can request a loan from the Activity Fund to cover the immediate expenses.
 \item
  The request must include the following:

  \begin{enumerate}
   \item
    A letter from the requesting committee
   \item
    Amount requested and the urgency of the request
   \item
    Proposed and actual budget
   \item
    Up-to-date cash and cash equivalents
  \end{enumerate}

 \item
  If the amount of the cash advance is less than \$10 000, it may be approved at the discretion of the VP Finance and Administration with a written notification to the Board.
 \item
  In the case that the request exceeds this amount, the cash advance shall go to the Board for approval.
 \item
  The cash advance shall be granted with a 0\% interest rate.
 \item
  The principal of the cash advance will be due upon the closing of the Activity's financial statements by the dates specified in the respective Activity Agreements.
\end{enumerate}

\subsection{Banking Requirements}\label{banking-requirements}

\subsubsection{Signing Authorities}\label{signing-authorities}

\begin{enumerate}
 \item
  The President may hold up to ``two to sign'' privileges on any CFES bank account.
 \item
  The VPFA may hold up to ``one to sign'' privileges on any CFES bank account; however any transaction with a value over \$3000 must be signed by the President in addition to the VPFA.
 \item
  All other Board Members be limited to holding ``All to sign'' Privileges.
\end{enumerate}

\subsubsection{Current Signing Authority
 Breakdown}\label{current-signing-authority-breakdown}

\begin{enumerate}
 \item
  The requirements in 8.6.1 are met in procedure laid out in the document ``CFES Signing Authority Procedures'', which will be added as an appendix to the CFES Policy Manual
\end{enumerate}

\subsection{Delegate Fee Collection}\label{delegate-fee-collection}

\begin{enumerate}
 \item
  The delegate fees for all Activities shall be collected through the CFES accounts.
 \item
  The Activity shall invoice all delegate fees through software provided by the CFES.
 \item
  Invoices shall be issued to the membership no later than one (1) week after the registration deadline.
 \item
  The payment deadline for invoices shall be no earlier than thirty (30) days after the invoice is distributed.
 \item
  The Activity shall track the payment status of all delegate fees through software provided by the CFES.
 \item
  The CFES may transfer the amounts of any paid invoices to the Activity Account.
 \item
  After the Activity, the complete amount of invoiced delegate fees shall be remitted to the Activity, regardless of the payment status of the invoice.

  \begin{enumerate}
   \item
    For Activity Accounts, funds shall be remitted within 30 days of the confirmation of the final delegate fee amount.
   \item
    For Activities with financial accounts managed by an external organization, funds shall be remitted within 30 days of receiving an invoice.
  \end{enumerate}
 \item
  The CFES will be responsible for ensuring that payment for unpaid invoices is received.
\end{enumerate}

\subsection{CFES Activity Bank
 Accounts}\label{cfes-activity-bank-accounts}

\subsubsection{Structure}\label{structure-1}

\begin{enumerate}
 \item
  The CFES shall provide a designated bank account, hereafter referred to as an ``Activity Account'', for activities opting to manage their conference finances through the CFES.
 \item
  The Activity Account will be a CIBC Not-for-Profit Business Operating Account.
 \item
  The Activity Account will be managed through CIBC SmartBanking™ for Business, hereafter referred to as ``SmartBanking''.
\end{enumerate}

\subsubsection{Signing Authorities}\label{signing-authorities-1}

\begin{enumerate}
 \item
  As Activity Accounts are sub-accounts of the CFES, the only signing authorities on Activity Accounts will be the CFES VPFA and President.
\end{enumerate}

\subsubsection{Access}\label{access}

\begin{enumerate}
 \item
  The Activity Chair(s) and Activity VP Finance(s) will be added as users to the Activity Account on SmartBanking.
 \item
  Two-step approval will be required for all transactions from the Activity Account.
 \item
  The Activity Chair(s) on the Activity Account shall have ``create or approve'' permissions.
 \item
  The Activity VP Finance(s) on the Activity Account shall have ``create or approve'' or ``view or create'' permissions, at the discretion of the Activity Chair(s).
\end{enumerate}

\subsubsection{Functionality}\label{functionality}

\begin{enumerate}
 \item
  E-Transfers, also known as Electronic Funds Transfers (EFTs), shall be done through the Interac e-Transfer® for Business feature on SmartBanking.
 \item
  Direct Deposits, also known as Electronic Funds Transfers (EFTs), shall be done through the Interac e-Transfer® for Business feature on SmartBanking, using account number routing.

  \begin{enumerate}
   \item
    Where account number routing is not possible, Direct Deposits shall be done through the CMO feature.
  \end{enumerate}
 \item
  Any cheques shall be written by the CFES VPFA from the main CFES Chequings Account.
\end{enumerate}

\subsubsection{Fees and Limits}\label{fees-and-limits}

\begin{enumerate}
 \item
  The Activity is responsible for all bank fees incurred on the Activity Account.
 \item
  The Activity is responsible for any fees incurred by Wire Transfers, Credit Card payments, or other transaction service charges, either incoming or outgoing.
 \item
  The CFES will fund the Activity Account the minimum balance, as outlined by CIBC, to avoid monthly account fees.

  \begin{enumerate}
   \item
    If the Activity Account balance falls below the minimum balance, the Activity must replenish this amount and cover the costs of any incurred monthly bank fees.
  \end{enumerate}
 \item
  The Activity Account has 30 free transactions per month, including Interac e-Transfer transactions, as per CIBC offerings.

  \begin{enumerate}
   \item
    The Activity will be responsible for any bank fees incurred if the 30 free transaction threshold is exceeded.
  \end{enumerate}
 \item
  The Activity Account shall have a \$25,000 per transaction and \$300,000 per day limit on the Interac e-Transfer® for Business feature.
\end{enumerate}

\subsection{Audit Costs}\label{audit-costs}

\begin{enumerate}
 \item
  Effective 2020-2021 fiscal year, in the years that the CFES is deemed required to have an audit, review engagement, or other financial review completed as per relevant Not-for-profit law, the cost of the audit shall be distributed among the following parties in the following manner, beginning in 2019-2020, increasing each year thereafter with the Consumer Price Index (CPI):

  \begin{enumerate}
   \item
    Summit on Development of Engineering Societies shall be responsible for 10\% of the audit fee, up to \$1000
   \item
    Conference on Diversity in Engineering shall be responsible for 15\% of the audit fee, up to \$1500
   \item
    CELC shall be responsible for 30\% of the audit fee, up to \$3000
   \item
    Conference on Sustainability in Engineering shall be responsible for 15\% of the audit fee, up to \$1500
   \item
    Canadian Engineering Competition shall be responsible for 30\% of the audit fee, up to \$3000
  \end{enumerate}

 \item
  The CFES shall be responsible for the remainder of the audit fee, including audit fees for Activities that use an incorporated Board's financial accounts as approved by Board and for audit fees exceeding \$10000.
 \item
  In the case that the actual cost of the audit is below the amount contributed above, the difference will be contributed to the Activity Fund.
\end{enumerate}

\subsection{Financial Deliverables}\label{financial-deliverables}

\subsubsection{Board Meetings}\label{board-meetings}

\begin{enumerate}
 \item
  Presenting detailed financial statements at Board Meetings include:

  \begin{enumerate}
   \item
    an accounting of all outgoing transactions of less than \$1000
   \item
    changes to all account balances since the previous meeting.
  \end{enumerate}
\end{enumerate}

\subsubsection{General Meetings}\label{general-meetings}

\begin{enumerate}
 \item
  Presenting detailed financial statements at meetings of the General Assembly which include:

  \begin{enumerate}
   \item
    an accounting of all outgoing transactions of less than \$1000
   \item
    changes to all account balances since the previous meeting.
  \end{enumerate}
\end{enumerate}

\subsubsection{Financial Reports}\label{financial-reports}

\begin{enumerate}
 \item
  A Financial Report must be provided by the VPFA before the final day of each month detailing:

  \begin{enumerate}
   \item
    The previous month's actuals
   \item
    Progress on the initiatives described in the VPFA's Action Plan
   \item
    The status of all investments, whether existing or new, along with an approximate value of each investment
   \item
    In the April, July, October, and January Financial Reports, the quarterly financials
   \item
    In the June Financial Report, the unaudited financial statements for the previous fiscal year
   \item
    The results of any tax returns in the month after their assessment by the CRA
   \item
    Any audited statements in the month after their finalization
  \end{enumerate}
\end{enumerate}

