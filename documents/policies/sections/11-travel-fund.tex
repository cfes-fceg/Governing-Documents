\section{Travel Fund}\label{travel-fund}

\subsection{Structure}\label{structure-2}

\begin{enumerate}
 \item
  The VPFA, on behalf of the National Executive, is responsible for the coordination of applications, allocations, and reimbursements of the Travel Fund.
 \item
  The Board is empowered to approve the recommendations as presented, or to make amendments as necessary.
 \item
  All travel funding allocations must be approved by the Board.
 \item
  Any requested reimbursements exceeding Board's approval must be amended at a later Board Meeting.
 \item
  For travel funding where an external funding source has committed to a certain amount or fraction, the reimbursement shall not exceed the total cost subtracted by that external commitment.
\end{enumerate}

\subsection{Funds Allocation}\label{funds-allocation}

\begin{enumerate}
 \item
  The budget presented by the VPFA at CELC will include a budget line entitled ``Travel Funding'' that will account for all funds to be spent on travel for the fiscal year of that budget.
 \item
  These funds should be broken down further into ``Board Member Travel'', ``Member School funding Travel'', and ``Partner Travel''.
 \item
  At CESS the Incoming VPFA shall present an amendment to the aforementioned budget line into a series of lines that allocate funds for each CFES activity and all external events the incoming National Executive intends to attend.
 \item
  There shall also be a 10 percent contingency for Board Member Travel.
 \item
  At least the larger of \$5,000 or 40\% of the total travel funding must be allocated for membership funding to SDES and CELC.
\end{enumerate}

\subsection{Fund Reallocation}\label{fund-reallocation}

\begin{enumerate}
 \item
  After an event with travel funding has passed, the Board must reallocate surplus funds to an upcoming member funding opportunity.
 \item
  Surplus travel funding remaining after Spring Meeting shall be carried over into the SDES Small School Travel Fund.
 \item
  Surplus travel funding remaining after SDES shall be carried over into the CELC Small School Travel Fund.
\end{enumerate}

\subsection{Board Travel Fund
 Applications}\label{board-travel-fund-applications}

\begin{enumerate}
 \item
  Board Members may be eligible to apply for travel funding for:

  \begin{enumerate}
   \item
    Internal Meetings
   \item
    CELC
   \item
    SDES
  \end{enumerate}

 \item
  All Board Members must submit an application regardless of whether they are applying for travel funding.
 \item
  Applications for travel funding must include the following documentation at a minimum:

  \begin{enumerate}
   \item
    Documentation supporting the amount requested. This could include receipts, current flight/bus/train ticket prices, rental vehicle estimate, or personal vehicle mileage rate, following the procedure outlined in Section 11.3.3.
   \item
    Proof they have exhausted funding from their Faculty;
   \item
    Proof they have exhausted funding from their Student Union;
   \item
    Proof they have exhausted funding from their Engineering Society;
   \item
    Proof they have chosen the least expensive feasible travel mode, and if traveling by air, that allows for changes to be made, even if for a fee.
  \end{enumerate}

 \item
  Funding may be approved based on estimated costs rather than a specific dollar amount.
 \item
  If funding is approved based on estimated costs, funding will be issued partially or fully as approved by the Board once proof of payment is provided.
 \item
  All applications shall be assessed by the following four criteria, weighted equally:

  \begin{enumerate}
   \item
    The relevance of the position of the person applying for funding at the meeting
   \item
    The performance of the person applying for funding over the last fiscal year
   \item
    Evidence of need (has the person looked for alternative funding, etc.)
   \item
    The appropriateness of the funding being requested (e.g.~is the mode of transportation appropriate, is the cost appropriate, previous use of the travel fund, etc.)
  \end{enumerate}
\end{enumerate}

\subsection{Small School Travel Fund
 Allocations}\label{small-school-travel-fund-allocations}

\subsubsection{Overview}\label{overview-2}

\begin{enumerate}
 \item
  Only members who hold Member School funding Funding Status at the application deadline are eligible to receive member travel funds.
 \item
  Each member is to receive an equal proportion of their requested funding amount for 1 delegate.
 \item
  The funding shall be for the delegate that will represent their Engineering Society in the General Assembly
\end{enumerate}

\subsubsection{Small School Funding
 Status}\label{small-school-funding-status}

\begin{enumerate}
 \item
  Schools may apply for Small School Funding Status if they have:

  \begin{enumerate}
   \item
    less than 1000 students, or
   \item
    an annual budget of less than \$50,000.
  \end{enumerate}

 \item
  Small School Funding Status shall last one year after the application is accepted.
 \item
  Applications shall be responded to within two weeks.
 \item
  The application shall include:

  \begin{enumerate}
   \item
    Detailed budget of the applicant society
   \item
    List of times the school has been a recipient of the Small School Travel Fund in previous years
   \item
    Detailed explanation of how the applicant has attempted to raise the amount being requested from their Dean, alumni and other available sources.
  \end{enumerate}
\end{enumerate}

\subsubsection{Special Funding Status}\label{special-funding-status}

\begin{enumerate}
 \item
  Members may speak to the Board for special funding status if the member is undergoing some unforeseen financial situation.
 \item
  The Board may approve them for special status with a (⅔) vote.
\end{enumerate}

\subsubsection{Membership Funding
 Applications}\label{membership-funding-applications}

\begin{enumerate}
 \item
  Members holding status requesting funding shall provide an application including the following:

  \begin{enumerate}
   \item
    An explanation of fundraising attempts
   \item
    Amount requested and expected cost
   \item
    Proof that the member is choosing a fiscally prudent travel option
  \end{enumerate}

 \item
  Application for funding for a particular event shall be open for at least two weeks.
 \item
  Applicants are to be informed of their funding amount no later than then the opening of pre-registration.
 \item
  If there is no pre-registration for an event the applicants will be informed no later than the start of registration.
 \item
  If the applicants are informed late then they shall be automatically granted an extension.
\end{enumerate}

\subsection{Eligible Travel Expenses}\label{eligible-travel-expenses}

\subsubsection{Flights, Trains, and
 Buses}\label{flights-trains-and-buses}

\begin{enumerate}
 \item
  Travel funding for CELC, and other travel greater than four days, may consider funding for a checked bag, or a fare type that includes a checked bag.
 \item
  For all other travel, travel funding for a carry-on may be considered, or a fare type that includes a carry-on.
 \item
  Seat selection, extra leg room, or any other carrier surcharges will not be considered for priority funding.
 \item
  Fare types should be purchased as follows:

  \begin{enumerate}
   \item
    Greater than 90 days before, a fully refundable ticket shall be purchased at minimum.
   \item
    Between 30 and 90 days before, a ticket that allows for changes, even with a fee, at minimum.
   \item
    Less than 30 days before, no changes or refund is required.
  \end{enumerate}
\end{enumerate}

\subsubsection{Rental vehicle usage}\label{rental-vehicle-usage}

\begin{enumerate}
 \item
  Travel funding applications involving rental vehicle use may be approved at the discretion of the Board.
 \item
  The liability of the rental vehicle is completely assumed by the individual.
 \item
  Reimbursement may include the cost of the rental vehicle, parking, and fuel receipts.
\end{enumerate}

\subsubsection{Personal vehicle usage}\label{personal-vehicle-usage}

\begin{enumerate}
 \item
  Travel funding applications involving personal vehicle use may be approved at the discretion of the Board.
 \item
  The liability of the personal vehicle is completely assumed by the individual.
 \item
  Reimbursement shall be calculated at a rate of \$0.35 per kilometre, allocated as follows:

  \begin{enumerate}
   \item
    \$0.15/km for fuel costs
   \item
    \$0.10/km for depreciation
   \item
    \$0.10/km for maintenance
  \end{enumerate}
 \item
  Reimbursement may also include parking costs.
\end{enumerate}

\subsubsection{Taxis and Rideshares}\label{taxis-and-rideshares}

\begin{enumerate}
 \item
  Travel funding applications involving taxis and rideshares may be approved at the discretion of the Board.
 \item
  Taxes and rideshares are discouraged when there are feasible public transportation methods available.
 \item
  Where feasible public transportation methods are available, funding for taxis or rideshares may be limited to the cost that the public transportation would have been.
 \item
  The feasibility of public transportation methods is at the discretion of the VPFA, with factors of duration, proximity, and time of day being considered.
 \item
  Taking rideshares with other individuals is encouraged to reduce costs.
\end{enumerate}

\subsubsection{Travel to and from the Point of
 Departure}\label{travel-to-and-from-the-point-of-departure}

\begin{enumerate}
 \item
  Costs associated with travel to and from the primary departure location may be approved by the Board, pursuant to the reimbursement criteria applicable for the relevant mode of transportation.
\end{enumerate}

\subsubsection{Travel to and from the Point of Arrival}\label{travel-to-and-from-the-point-of-arrival}

\begin{enumerate}
 \item
  Costs associated with travel between the point of arrival and local accommodations or venues are not prioritized for funding
\end{enumerate}

\subsubsection{Meals}\label{meals}

\begin{enumerate}
 \item
  Travel that exceeds 9 hours, not including time spent waiting, may be eligible for meal subsidies.
\end{enumerate}

\subsubsection{Delegate Fees}\label{delegate-fees}

\begin{enumerate}
 \item
  Delegate fees to a conference will not be considered for priority funding.
\end{enumerate}

