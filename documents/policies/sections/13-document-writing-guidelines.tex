\section{Document Writing Guidelines}\label{document-writing-guidelines}

\subsection{Translation Guidelines}\label{translation-guidelines}

\begin{enumerate}
 \item
  The following items require translation before distribution:

  \begin{enumerate}
   \item
    Constitution and Policy Manual
   \item
    Binding documents
   \item
    Strategic Plan
   \item
    CEC Rules
   \item
    Information directed at students of members (such as delegate packages, surveys, information manuals, guides, etc.)
   \item
    Text of any member-wide communications from the National Executive
   \item
    All documentation related to legal complaints and proceedings
   \item
    Previous Assembly minutes
   \item
    Board reports
   \item
    Official reports on major issues
   \item
    CELC documents
   \item
    Document of Stances
   \item
    Agendas and supporting documents for the General Assembly
  \end{enumerate}

 \item
  The Constitution and Policy Manual should each be translated at least once every year.
 \item
  The agenda and other documents for the General Assembly may be distributed before translation.
 \item
  Minutes of General Assemblies and reports may also be distributed before translation.
 \item
  They must be fully translated and distributed in both English and French as quickly as possible.
 \item
  Members may request the translation of any other documents.
\end{enumerate}

\subsection{Feminization of French
 Text}\label{feminization-of-french-text}

\begin{enumerate}
 \item
  Any text written in or translated into the French language must be feminized or gender-neutralized, namely by applying the guidelines laid out in the guide op epicene writing published by the Quebec Board of the French Language.
\end{enumerate}

\subsection{Governance Committee}\label{governance-committee}

\subsubsection{Overview}\label{overview-3}

\begin{enumerate}
 \item
  The Governance Committee is responsible for reviewing and ensuring the integrity and effectiveness of the organization\textquotesingle s governing documents, including the Constitution, bylaws, and policies.
 \item
  The committee will present proposed changes to the Board and the General Assembly when applicable, ensuring that all relevant stakeholders are informed and engaged in the governance process.
\end{enumerate}

\subsubsection{Tasks and
 Responsibilities}\label{tasks-and-responsibilities-1}

\begin{enumerate}
 \item
  The Committee shall meet monthly to systematically review sections of the governing documents, ensuring that all governing documents are comprehensively reviewed annually.
 \item
  The Committee shall prepare and present proposed changes to the Board and to the General Assembly as necessary.
 \item
  The Committee shall stay informed about best practices in governance to ensure that the governing documents align with current standards and regulatory requirements.
\end{enumerate}

\subsubsection{Composition}\label{composition}

\begin{enumerate}
 \item
  The Governance Committee shall be chaired by the Board Chair, and shall be composed of the following members of the CFES Team.

  \begin{enumerate}
   \item
    The Board Chair
   \item
    The President
   \item
    The Vice President Finance and Administration
   \item
    A minimum of one (1) National Councillor
  \end{enumerate}

 \item
  An unlimited number of additional members of the CFES Team may be selected by the Board to join the Committee at any point throughout the term based on interest and relevant expertise.
 \item
  Such individuals may demonstrate interest by contacting the Board Chair.
 \item
  The Board shall strive to ensure that the Committee is composed of a diverse and knowledgeable group that can effectively contribute to the review and enhancement of the organization\textquotesingle s governing documents.
\end{enumerate}

