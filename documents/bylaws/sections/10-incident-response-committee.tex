\section{Incident Response Committee}
\label{incident-response-committee}

\subsection{Purpose}
\label{purpose-2}

\begin{enumerate}
 \item
  The Incident Reporting Form will be the way that the Incident Response Committee receives complaints of violations of the Code of Conduct.
 \item
  To ensure that CFES Activities are a safe and welcoming space for everyone, an Incident Response Committee shall be formed to investigate and attempt to resolve any reported issues.
 \item
  The goal of the Incident Response Committee shall be to ensure that a safe and inclusive environment is maintained at CFES Activities.
\end{enumerate}

\subsection{Incident Reporting Form}
\label{incident-reporting-form}

\subsubsection{Structure}
\label{structure}

\begin{enumerate}
 \item
  The Incident Response Form shall be a form created by the Incident Response Committee and made accessible to all delegates at a CFES Activity during the Activity and up to 3 days afterwards.
 \item
  The form shall include the following fields:
\end{enumerate}

\begin{enumerate}
 \item
  Name of Reporting Individual
 \item
  School of Reporting Individual
 \item
  Name(s) of the Individual(s) or Organization Involved in the Incident (or description(s))
 \item
  Incident Date and Time
 \item
  Incident Details
 \item
  Incident Witness(es), if appropriate
 \item
  Contact Information (Optional)
 \item
  Whether or not the Reporter is comfortable being contacted by the IRC
 \item
  A description of any personal relationships or potential conflicts of interest with IRC members, with regards to both the reporting individual or anyone involved in the complaint
 \item
  A brief description of IRC procedure, which includes clarifying that all complaints submitted through the Incident Reporting Form will go to the Chair of the IRC and that if complainants wish to submit complaints outside of the form, it must be done through the contact information on this form.
\end{enumerate}

\subsubsection{Confidentiality of Reports}
\label{confidentiality-of-reports}

\begin{enumerate}
 \item
  Any identifying contents of Incident Reports, and identifying information discovered during the investigation, shall be kept confidential by the members of the Incident Response Committee (IRC) subject only to Bylaw 10.3.6: Release of Information.
\end{enumerate}

\subsubsection{Incident Response Timeline}
\label{incident-response-timeline}

\begin{enumerate}
 \item
  Should a report be submitted between 8am and 8pm, the Incident Response Committee shall begin to address the report within four hours.
 \item
  Should a report be submitted outside of these times, but during the conference, the IRC shall begin to address the report by the following noon.
 \item
  The IRC is not permitted to work through meal time, and thus any report open during that time has its timeline extended.
 \item
  Should a report be submitted with less than 4 hours remaining in the conference, or within 72 hours after midnight of the last day of the conference, the IRC shall begin to address the report within 36 hours.
\end{enumerate}

\subsection{Incident Response Committee}
\label{incident-response-committee-1}

\subsubsection{Structure}
\label{structure-1}

\begin{enumerate}
 \item
  An Incident Response Committee (IRC) shall be formed for each CFES Activity.
\end{enumerate}

\subsubsection{Committee Membership}
\label{committee-membership}

\begin{enumerate}
 \item
  An IRC shall consist of up to six (6) members including:
\end{enumerate}

\begin{enumerate}
 \item
  Up to two non voting members of the Board who shall serve as IRC Chair(s).
 \item
  The proposed IRC Chair(s) shall be responsible for choosing an IRC committee and presenting it to the Board for approval.
 \item
  The committee can consist of one or a combination of the following:

  \begin{enumerate}
   \item
    Four members of the OC, chosen by the Activity Manager(s) with consultation of the Board
  \end{enumerate}
\end{enumerate}

\begin{enumerate}
 \item
  These OC members can maintain their usual responsibilities if possible
 \item
  A separate OC hire (1-4 individuals) conducted by the Activity Manager(s) can be considered if the Activity Manager(s) expresses reasonable concern over workload.
 \item
  The individuals will be welcome to attend sessions if applicable for the Activity.

  \begin{enumerate}
   \item
    Four non-member delegates selected by the Board.

    \begin{enumerate}
     \item
      CELC may have two separate IRCs, each consisting of an IRC chair and four non-member delegates for a total of 2 IRC chairs and eight non-member delegates.
     \item
      In the case of two IRC chairs, they shall throughout the entirety of the Activity share duties.
     \item
      For each case, only one of the two chairs will act as a voting chair while the other performs as a consultant chair without a vote.
     \item
      The consultant chair may sit in for any interviews and case discussions, but does not have a final vote.
     \item
      Chairs will alternate as voting chairs between cases to balance workload.
     \item
      Co-chairs may distribute cases between them as they see fit.
     \item
      The IRC membership will:
    \end{enumerate}
  \end{enumerate}
\end{enumerate}

\begin{enumerate}
 \item
  Undergo IRC introductions with the National Councillors
 \item
  Undergo Psychological First Aid Training paid for by the CFES.
 \item
  Have no more than one member of the Board.
 \item
  Be ratified by the Board no later than 2 weeks before the start of the Activity.
 \item
  Be kept confidential, unless necessary for the IRC to complete their duties, for the duration of the conference

  \begin{enumerate}
   \item
    When selecting members of the IRC, the Board shall make an effort to ensure that the composition of the IRC reflects a diverse range of perspectives and backgrounds.
   \item
    Additionally, the Board shall make an effort to ensure that members do not serve on multiple consecutive IRCs when possible.
   \item
    To mitigate the psychological risks associated with this role, IRC members will be required to submit emotional safety plans to the Board no later than three (3) days before the beginning of the activity.
   \item
    Emotional safety plans must include, at a minimum, the following;
  \end{enumerate}
\end{enumerate}

\begin{enumerate}
 \item
  A recognition of the emotional labour associated with the role and a willingness to participate in such labour
 \item
  An outline of professional supports available to the IRC member.
 \item
  An outline of personal supports available to the IRC member.
 \item
  Contact information for individuals identified by the IRC member as key-emotional supports.

  \begin{enumerate}
   \item
    In the case of extreme emotional duress, IRC members may use the professional \& personal supports outlined in their emotional safety plans to discuss concerns relating to their work as an IRC member.
   \item
    IRC members are not permitted to discuss their role as an IRC member with other attending delegates and are subject to Bylaw 10.3.6 in all other matters relating to their role.
   \item
    The Organizing Committee, with consultation of the IRC and National Executive, is expected to gather mental health resources to have ready for the Activity in order to direct delegates to them for professional support.
   \item
    These can be services offered by the Activity, or external resources that may be offered in the city/province of the Activity.
   \item
    If there are no mental health resources available, the IRC Chair must let the Board know so other precautions can take place.
  \end{enumerate}
\end{enumerate}

\subsubsection{Committee Authority}
\label{committee-authority}

\begin{enumerate}
 \item
  Decisions made by the IRC shall be considered final for the duration of the Activity, and may range from verbally reprimanding an offending delegate to expulsion from the conference.
 \item
  After the conclusion of the Activity, the respondent may appeal decisions made by the IRC.
 \item
  Delegates may not be expelled by any other authority of the CFES.
 \item
  The IRC may not make decisions that affect the delegate's ability to contribute to the CFES in the future, such as conditionally or completely banning from further conferences, but may recommend these actions to the Appeal Board.
\end{enumerate}

\subsubsection{Committee Response Procedure}
\label{committee-response-procedure}

\begin{enumerate}
 \item
  All complaints submitted through the Incident Reporting Form will first be examined by the Chair of the IRC, and the Chair of the IRC will examine the form to determine whether any members of the IRC should not have access to the complaint due to conflict of interest as per Bylaw 10.3.5.
 \item
  Once this is determined, the Chair of the IRC shall share the form responses with all relevant members of the IRC and begin their investigation.
 \item
  If a complaint is submitted to another member of the IRC due to a conflict of interest with the IRC Chair, the member who the complaint was submitted to shall conduct the same process before sharing the complaint with other IRC members.
 \item
  To ensure that the IRC has a complete picture of the Incident, and the impact it has had on the involved parties, they should contact the following parties and discuss the Incident and appropriate responses:
\end{enumerate}

\begin{enumerate}
 \item
  The Respondent(s)
 \item
  The Affected Person(s)
 \item
  The individual who made the report, if appropriate
 \item
  Any witnesses to the Incident, if absolutely necessary

  \begin{enumerate}
   \item
    Note: The Respondent(s) and the Affected Persons' will be invited to have an additional individual for support present during any conversations.
   \item
    After these discussions, the IRC shall use their best judgment on what response is appropriate or if further investigation is needed.
   \item
    All IRC decisions require a majority vote, and the chair of the IRC shall only vote in the event of a tie.
   \item
    Any official communications should be made by the Chair of the IRC, with the knowledge and approval of all members of the IRC except in cases where the Chair of the IRC has a conflict of interest as described in Bylaw 10.3.5.
   \item
    In this case, the IRC shall appoint one of their members to make official communications for the complaint in question.
   \item
    Responses to any official communication should be shared with all IRC members.
   \item
    The IRC should ensure that there are at least two IRC members present for any conversation with anyone involved with the Incident.
   \item
    The IRC should be directing those involved to the mental health resources mentioned in 8.4.2.4.
  \end{enumerate}
\end{enumerate}

\subsubsection{Conflict of Interest}
\label{conflict-of-interest}

\begin{enumerate}
 \item
  A conflict of interest shall be defined as a strong positive or negative personal relationship between a member of the IRC and the Respondent, Victim, or other individuals involved in the Incident that unfairly biases their actions or decisions.
 \item
  A member of the IRC should remove themselves from the IRC if they believe they have a perceived or real conflict of interest.
 \item
  Should anyone believe that an IRC member has a conflict of interest, it should be presented to the IRC.
 \item
  If one member of the IRC believes that a perceived or real conflict of interest exists, the appropriate member shall be removed from the IRC.
 \item
  If three or more members of the IRC are required to step down due to conflict of interest, the remaining members of the IRC will be responsible for choosing a replacement member on a temporary or permanent basis, as appropriate, and notifying the Appeal Board of the change.
 \item
  The remaining members of the IRC shall release changes of membership to the Activity delegates if there are permanent changes, and these changes shall be reflected on the Incident Response Form.
 \item
  In the case of two IRC chairs, if one of the chairs has a conflict of interest, the other chair must take over the case without consultation from the other chair.
\end{enumerate}

\subsubsection{Release of Information}
\label{release-of-information}

\begin{enumerate}
 \item
  The IRC is expected to keep all identifying information for anyone involved other than the Respondent permanently confidential except in the following situations:
\end{enumerate}

\begin{enumerate}
 \item
  The IRC or anyone involved with the Incident decides that the Police should be notified, or;
 \item
  The IRC has obtained written consent of everyone involved to release the information.

  \begin{enumerate}
   \item
    The IRC is expected to keep confidential any information that may identify the Respondent, except in the following situations:
  \end{enumerate}
\end{enumerate}

\begin{enumerate}
 \item
  The IRC or anyone involved with the Incident decides that the Police should be notified, or;
 \item
  The IRC has obtained written consent of everyone involved to release the information, or;
 \item
  The IRC recommends that the Board notify the Respondent's school i.e. The Dean's office) the nature of the complaint.

  \begin{enumerate}
   \item
    The IRC recommends that the Board limit or end an individual\textquotesingle s involvement in CFES activities.
   \item
    If the Board (after the appeal has been finalized) decides to uphold this recommendation the member society shall be notified of the individual\textquotesingle s status but not of the reasoning for the decision.
   \item
    Should the IRC decide to take action against a delegate they are to inform the head delegate of the decision.
   \item
    If the action is being taken against a head delegate the delegate\textquotesingle s Regional Ambassador is to be informed.
   \item
    If an IRC member breaches these confidentiality rules, they may be subject to discipline by the Board.
  \end{enumerate}
\end{enumerate}

\subsubsection{Best Practices}
\label{best-practices}

\begin{enumerate}
 \item
  The IRC be briefed by the National Councillors on best practises as per the Incident Response Committee Best Practises document.
 \item
  The Incident Response Best Practises should be maintained by the National Executive and IRC Chairs.
\end{enumerate}

\subsection{Appeal of Decision}
\label{appeal-of-decision}

\subsubsection{Purpose}
\label{purpose-3}

\begin{enumerate}
 \item
  Should anyone feel that a decision made by the Incident Response Committee (IRC), or that the process followed by the IRC, was unfair they may Appeal the decision by informing the IRC Chair.
\end{enumerate}

\subsubsection{Appeal Board}
\label{appeal-board}

\begin{enumerate}
 \item
  The Board shall hear all appeals resulting from decisions made by the IRC.
 \item
  Any Board members who were members of the IRC should attend the meeting where an appeal is heard to provide information, but should not participate in any decision made regarding the appeal.
 \item
  The Board may uphold or overturn decisions made by the IRC, and may accept the recommendation of the IRC to limit an individual\textquotesingle s involvement with the CFES for a period of time, or until certain conditions are met.
 \item
  All decisions made by the Board are final.
\end{enumerate}

\subsubsection{Appeal Timeline}
\label{appeal-timeline}

\begin{enumerate}
 \item
  Appeals will be accepted by the IRC Chair for 72 hours after either the close of the Activity, or notice of a decision made by the IRC, whichever happens last.
 \item
  A meeting will be scheduled by the Board Chair 14 days of the Appeal being made to hear from the individual who brings forward the Appeal and, separately, the Chair of the IRC
\end{enumerate}

\subsubsection{Conflict of Interest}
\label{conflict-of-interest-1}

\begin{enumerate}
 \item
  A member of the Board should remove themselves from the Appeal Process if they believe they have conflict of interest as defined in Bylaw 10.3.5.
\end{enumerate}

\subsection{Recording of Information}
\label{recording-of-information}

\begin{enumerate}
 \item
  In the case that a school is notified of the IRC's decision, or that a decision to limit or end an individual's involvement is made, the following should be recorded and kept by the National Councillors for future iterations of the IRC:
\end{enumerate}

\begin{enumerate}
 \item
  Name of Respondent.
 \item
  Decision that was reached.
 \item
  Relevant copies of correspondence with the School and/or Individual.

  \begin{enumerate}
   \item
    Otherwise, the following information should be recorded and kept by the National Councillors for future IRCs:
  \end{enumerate}
\end{enumerate}

\begin{enumerate}
 \item
  High level summary of the incident names, schools, or any other identifying information.
 \item
  Decision that was reached.
\end{enumerate}
