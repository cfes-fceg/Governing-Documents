\section{Membership}
\label{membership}

\subsection{Membership Fees}
\label{membership-fees}

\begin{enumerate}
 \item
  The membership fee for each school shall be \$0.65 per full-time engineering undergraduate student beginning in 2019-2020, in addition to any student who has opted into CFES membership as specified in Bylaw 6.4, increasing each year thereafter with the Consumer Price Index (CPI).
 \item
  The number of members shall be determined by the enrollment during the previous academic year.
 \item
  Each member is responsible for ensuring its membership fee is paid in full and on time.
 \item
  All members are responsible for reporting the total number of eligible members to the Vice President Finance and Administration before October 31st of each given fiscal year
\end{enumerate}

\subsection{Member Types}
\label{member-types}

\subsubsection{Regular Member in Good Standing}
\label{regular-member-in-good-standing}

\begin{enumerate}
 \item
  A Regular Member in Good Standing of the Federation is defined as having paid its annual membership fees as outlined in Bylaw 6.1.
 \item
  A Regular Member in Good Standing of the Federation is defined as having paid all Delegate Fees for students registered to attend CFES affiliated activities, regardless of whether or not they attend the activity, on time or within an approved time period.
 \item
  A Regular Member in Good Standing of the Federation is defined as having resolved any deficits and outstanding wrap up associated with activities it was responsible for hosting within the appropriate timelines
 \item
  Only Regular Members in Good Standing have voting rights in the Federation.
\end{enumerate}

\subsubsection{Regular Member Not in Good Standing}
\label{regular-member-not-in-good-standing}

\begin{enumerate}
 \item
  A Regular Member Not in Good Standing does not have voting rights in the Federation.
 \item
  Students who are represented by Members Not in Good Standing may qualify to attend CFES Activities or utilize CFES Services, provided the member has contacted the Board with intent to pay their membership fees, and the appropriate governing body for the Activity or Service approves their participation.
\end{enumerate}

\subsubsection{Observer}
\label{observer}

\begin{enumerate}
 \item
  An observer does not have voting rights in the Federation.
\end{enumerate}

\subsection{Member Status}
\label{member-status}

\subsubsection{Suspension of Members}
\label{suspension-of-members}

\begin{enumerate}
 \item
  The Board may, by resolution adopted by two thirds (2/3) of Board present at a meeting convened to this end, suspend any member which infringes upon a resolution or by-law of the Federation or whose conduct or activities are deemed detrimental to the wellbeing or functioning of the Federation, or who has not paid their membership fees for the current year, as outlined in Policy 8.1.1., or who has not paid their delegate fees for an activity..
 \item
  Such a suspended member will be deemed ``Member Not in Good Standing''.
 \item
  Within two business days after the suspension of a member, the President shall inform all members of the suspension, with reasoning provided.
 \item
  Members who are considered Members Not in Good Standing due to failure to pay their membership fees or delegate fees for a CFES activity shall be reinstated upon full payment of any outstanding fees.
\end{enumerate}

\subsubsection{Expulsion of Members}
\label{expulsion-of-members}

\begin{enumerate}
 \item
  A General Assembly may, by resolution adopted by 2/3 of members present at a member's assemble, expel any member which infringes upon the CFES constitution, or whose conduct or activities are deemed detrimental to the wellbeing or functioning of the CFES.
 \item
  Observers will automatically be expelled if they do not apply to become a regular member as per the Constitution, apply for a one-time extension as per Bylaw 6.3.4, or withdraw membership as per Article 6.3.5 within 30 days of the end of their observer status.
\end{enumerate}

\subsubsection{Application to be granted Observer Status}
\label{application-to-be-granted-observer-status}

\begin{enumerate}
 \item
  To apply for observer status an organization must submit a written application to the Vice-President Finance and Administration, who must immediately inform the Board of such an application.
 \item
  The Board must then make a decision regarding the organization admission, and this no later than the Board meeting following the organization's submission of the application.
 \item
  Following a decision from the Board, the members shall be immediately notified.
 \item
  A society may only apply for observer status if at any point in the past 12 months they have neither been a regular member nor held observer status.
\end{enumerate}

\subsubsection{Application to extend Observer Status}
\label{application-to-extend-observer-status}

\begin{enumerate}
 \item
  An observer may apply for a one-time extension at the end of their first year as an observer.
 \item
  To extend observer an organization must submit a written application to the Vice-President Finance and Administration, who must immediately inform the Board of such an application.
 \item
  The application must include the following:
  \begin{enumerate}
   \item
    A letter from the President of the society explaining the reasons for the decision to extend observer status and evaluating the CFES' progress on areas of improvement put forward in the original observer status application letter
   \item
    Proof of support for the extension from the engineering society through a passed motion in public meeting minutes.
   \item
    It is recommended that a member of the Board be invited to this meeting to represent the CFES.
  \end{enumerate}

 \item
  An observer status extension is valid for one year after an extension application is approved after which point the society in question must choose to apply to become a member as per Article 4.3 or withdraw membership as per Bylaw 6.3.5.

\end{enumerate}

\subsubsection{Membership Withdrawal Procedure}
\label{membership-withdrawal-procedure}

\begin{enumerate}
 \item
  Any member may withdraw from the Federation by delivering a written resignation to the Vice-President Finance and Administration of the Federation.
 \item
  The withdrawal shall take effect the first day of the month immediately following reception of the notice.
\end{enumerate}

\subsubsection{Grace Period}
\label{grace-period}

\begin{enumerate}
 \item
  The withdrawn member can immediately return to the membership status they held before the withdrawal occurred without needing to go through the formal membership application process, provided that by Summit on Development of Engineering Societies of the fiscal year following the withdrawal the member:
  \begin{enumerate}
   \item
    Notifies the Board with written proof of support from the engineering society (a passed motion in public meeting minutes).
   \item
    Pays the membership fees to be up to date with dues
   \item
    Before withdrawal they were a Regular Member In Good Standing or a Regular Member Not In Good Standing.
   \item
    Has not entered a grace period within the last 12 months
  \end{enumerate}
\end{enumerate}

\subsubsection{Appeal of Suspension or Expulsion}
\label{appeal-of-suspension-or-expulsion}

\begin{enumerate}
 \item
  A member may request the return of their membership fees before June 1st of that fiscal year.
 \item
  If a refund is granted, the member is suspended as per Bylaw 6.3.1 until such a time that they pay their full fee for the financial year.
 \item
  The Board may, by resolution adopted by two-thirds (2/3) of the Board present at a meeting convened to this end, grant a full refund in fees.
 \item
  The Board may ask for documentation, including financial documentation, from the member in order to properly assess the need for a refund.
\end{enumerate}

\subsection{Member Student Opt-In}
\label{member-student-opt-in}

\begin{enumerate}
 \item
  Member societies may decide to opt in all students they represent who are not either enrolled in an accredited engineering program or enrolled in a program in the process of accreditation into the CFES membership.
 \item
  Societies who wish to opt-in additional students to the CFES must submit this request to the Vice-President Finance and Administration before July 1st of the year prior to the membership fee due date.
 \item
  The CFES' responsibility to members opted in this way is limited to providing access to the activities outlined in Article 12 of the constitution.
 \item
  If a request is not submitted before this date, students who are not either enrolled in an accredited engineering program or enrolled in a program in the process of accreditation shall not be permitted to attend CFES activities, gain from the services it provides, or be eligible for positions on the CFES Team unless an exception is granted by the Board..
\end{enumerate}
