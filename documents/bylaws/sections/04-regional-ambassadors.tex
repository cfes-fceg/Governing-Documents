\section{Regional Ambassadors}
\label{regional-ambassadors}

\subsection{Purpose}
\label{purpose}

\begin{enumerate}
 \item
  The Regional Ambassadors are chosen by their respective region to represent on the Board.
 \item
  There is one Regional Ambassador per region, and they have tasks to accomplish at both the national and regional levels.
\end{enumerate}

\subsection{Responsibilities}
\label{responsibilities}

\begin{enumerate}
 \item
  The regional ambassadors shall be responsible for representing their respective region's views to the Federation and facilitating the communication between their respective regional associations and the federation and fill any other mandate which is entrusted to them by the general assembly or the Board.
\end{enumerate}

\subsection{Regional Agreements}
\label{regional-agreements}

\begin{enumerate}
 \item
  The CFES shall maintain contracts with the legal entities corresponding to the regions of the CFES as defined in the Constitution.
 \item
  These contracts will be known as Regional Agreements.
 \item
  Regional Agreements should be reviewed annually, and address the following areas at a minimum:
  \begin{enumerate}
   \item
    Regional Ambassador responsibilities: the Regional Agreements shall state that governing documents of the relevant regional organization and the CFES establish the duties of the Regional Ambassador for the region in question; no duties shall be contained in the Regional Agreements themselves
   \item
    Regional Ambassador elections: the Regional Agreements shall establish procedures for the election of the region in question's Regional Ambassador, including the timeline for ratification by the CFES General Assembly or Board
   \item
    Regional engineering competitions: the Regional Agreements shall require the chair of the relevant region's engineering competition to assist in an advisory capacity with the Canadian Engineering Competition
  \end{enumerate}
\end{enumerate}

\subsection{Tasks and Responsibilities to the CFES}
\label{tasks-and-responsibilities-to-the-cfes}

\begin{enumerate}
 \item
  The Regional Ambassadors' tasks and responsibilities include:
  \begin{enumerate}
   \item
    Maintaining open lines of communication between CFES and members in their region, and bringing their region's perspective forward in Board meetings.
   \item
    Facilitating the use of proxy votes as necessary.
   \item
    Encouraging the members of their region to participate in and/or organize CFES events
   \item
    Encouraging students to run for CFES positions.
   \item
    Assisting the VPFA in the collection of membership fees.
   \item
    Supporting the Administrative Commissioner in providing the VPS with updated contact information for members in their region.
   \item
    Conducting CFES introduction and update sessions for members in their region.
   \item
    Ensuring CFES is represented at major regional activities, such as AGMs, regional competitions and regional meetings.
   \item
    Managing regional meetings at CELC.
   \item
    Meet with their constituencies before all conferences to keep them updated on the progress of the Federation's mandates and inform all delegates of the conference expectations. Monthly timelines are as follows:

    \begin{enumerate}
     \item
      August or September to prepare for Summit on Development of Engineering Societies
     \item
      November to prepare for CELC
    \end{enumerate}
   \item
    Meet with their constituencies after all general meetings to review the outcomes. Monthly timelines are as follows:

    \begin{enumerate}
     \item
      September or October to review Summit on Development of Engineering Societies
     \item
      January to review CELC
    \end{enumerate}
   \item
    Creating a summary transition document by April 30th to be stored in the electronic CFES archive
   \item
    Liaising on behalf of their region to assist in designating individuals as representatives on CFES bodies, such as the Advocacy Working Group
   \item
    Providing written updates on tasks completed at a frequency determined by the National Executive and at minimum every two months
   \item
    Other duties as outlined in the relevant regional organization's governing documents.
  \end{enumerate}
\end{enumerate}

\subsection{Regional Ambassadors}
\label{regional-ambassadors-1}

\begin{enumerate}
 \item
  The Regional Ambassadors are the following 4 positions, each chosen by their respective regional organization and ratified at CELC:
  \begin{enumerate}
   \item
    West Ambassador
   \item
    Ontario Ambassador
   \item
    Quebec Ambassador
   \item
    Atlantic Ambassador
  \end{enumerate}
\end{enumerate}

\subsection{Vacancy}
\label{vacancy-2}

\begin{enumerate}
 \item
  In case of a vacancy, this vacancy shall be communicated to the relevant regional membership and the relevant regional organization within five (5) days.
 \item
  A Regional Ambassador position shall be filled by a mail-in ballot of the members of the relevant region.
 \item
  This shall be facilitated by the relevant regional organization, and ratified by the Board.
 \item
  A nomination period must be announced to the relevant members and remain open for a minimum of two (2) weeks, unless otherwise specified by the relevant regional organization.
\end{enumerate}

\subsection{Dismissal}
\label{dismissal-1}

\begin{enumerate}
 \item
  The Regional Ambassadors may be dismissed by a special resolution of the members of the relevant region.
 \item
  This shall be facilitated by the relevant regional organization.
\end{enumerate}
